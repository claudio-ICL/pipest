\documentclass[10pt, article,table]{article}
\include{header}
\usepackage{mathrsfs}
\usepackage{amsfonts}
\usepackage{xfrac}
%\usepackage{mathbbol}
%\usepackage{dsfont}
%\usepackage{bbm}

\usepackage{amsthm}
\usepackage{amsmath}
\usepackage{amssymb}





\newsavebox{\fminipagebox}
\NewDocumentEnvironment{fminipage}{m O{\fboxsep}}
{\par\kern#2\noindent\begin{lrbox}{\fminipagebox}
		\begin{minipage}{#1}\ignorespaces}
		{\end{minipage}\end{lrbox}%
	\makebox[#1]{%
		\kern\dimexpr-\fboxsep-\fboxrule\relax
		\fbox{\usebox{\fminipagebox}}%
		\kern\dimexpr-\fboxsep-\fboxrule\relax
	}\par\kern#2
}



\newcommand{\R}{\mathbb{R}}
\newcommand{\Rd}{\R^{d}}
\newcommand{\Rn}{\R^{n}}
\newcommand{\Rm}{\R^{m}}
\newcommand{\RN}{\R^{N}}
\newcommand{\Rnn}{\R^{n\times n}}
\newcommand{\N}{\mathbb{N}}
\newcommand{\Z}{\mathbb{Z}}
\newcommand{\C}{\mathbb{C}}
\newcommand{\X}{\mathbb{X}}
\newcommand{\sigmatwo}{\sigma^{2}}
\newcommand{\phalf}{\frac{p}{2}}
\newcommand{\half}{\frac{1}{2}}
\newcommand{\inverse}{^{-1}}
\newcommand{\symmetricPart}{\text{sym}}
\newcommand{\antisymmetricPart}{\text{antisym}}
\newcommand{\abs}[1]{\left\lvert {#1} \right\rvert}

\newcommand{\id}[1][ ]{\mathtt{id}_{#1}}

\newcommand{\transpose}{^{\mathsf{T}}}
\newcommand{\squared}{^{2}}

\newcommand{\subscriptij}{_{i,j}}
\newcommand{\tzero}{t_{0}}

\newcommand{\suchThat}{\text{ s.t. }}
\newcommand{\argmin}{\mathrm{argmin}}
\newcommand{\partition}{\pi}
\newcommand{\timeHorizon}{T}
\newcommand{\timeWindow}{{[0,\timeHorizon]}}

\newcommand{\hurstExponent}{H}
\newcommand{\volterraKernel}{{K}_{\hurstExponent}}
\newcommand{\volterraProcess}{\zeta}
\newcommand{\covarianceKernel}{{R}_{\hurstExponent}}

\newcommand{\zeroSleqTtimeHorizon}{0\leq s\leq t \leq \timeHorizon}
\newcommand{\zerostTimeHorizon}{0\leq s, t \leq \timeHorizon}
\newcommand{\simplex}{\lbrace (s,t) \in \R\squared:\,  \zeroSleqTtimeHorizon\rbrace}
\newcommand{\timesSleqUleqT}{s \leq u \leq t}

\newcommand{\convexHull}{\mathrm{Conv}}

\newcommand{\restrictedto}[1]{\arrowvert_{#1}}

\newcommand{\trace}{\mathtt{trace}}

\newcommand{\positivePartOfMinimum}{\text{min}_{+}}

\newcommand{\dotEta}{\dot{\eta}}

\newcommand{\intzerot}[1][t]{\int_{0}^{#1}}

%notation for derivatives
\newcommand{\derivative}{^{\prime}}
\newcommand{\gradient}{\nabla}
\newcommand{\Fprime}{F\derivative}
\newcommand{\partialij}{\partial^{2}_{i,j}}
\newcommand{\gradx}{\nabla_{x}}
\newcommand{\gradz}{\nabla_{z}}
\newcommand{\Hessianx}{\nabla^{2}_{xx}}
\newcommand{\Hessianz}{\nabla^{2}_{zz}}





%norms
\newcommand{\norm}[1][\cdot]{\left\lVert {#1}\right\rVert}
\newcommand{\tripleNorm}[1]{\vert \vert \vert {#1}\vert \vert \vert }
\newcommand{\supNorm}[1][\cdot]{\lVert #1 \rVert_{\infty}}
\newcommand{\HoelNorm}[2][\cdot]{\lVert {#1} \rVert_{{#2}\text{-H\"ol}}}
\newcommand{\pvarNorm}[2][\cdot]{\lVert {#1} \rVert_{{#2}\text{-var}}}
\newcommand{\pvarNormInterval}[3][\cdot]{\lVert {#1} \rVert_{{#2}\text{-var}, {#3}}}

%Geometric notation
\newcommand{\oneforms}[2]{ {\Omega}^{{1}} ({#1},{#2})}
\newcommand{\alphatilde}{\tilde{\alpha}}
\newcommand{\sectionsTM}[1][TM]{\Gamma(#1)}
\newcommand{\TxM}{T_x M}
\newcommand{\TmM}{T_m M}
\newcommand{\Homomorphisms}{\text{Hom}}
\newcommand{\manifold}{\mathcal{M}}

%notation functional analysis
\newcommand{\banachSpace}{\mathcal{B}}
\newcommand{\pairing}[2]{\langle{#1},\, {#2} \rangle }
\newcommand{\Lone}{L^{1}}
\newcommand{\Ltwo}{L\squared}
\newcommand{\vectorSpace}{\boldsymbol{V}}
\newcommand{\smoothFunctions}[2]{C^{\infty}(#1,#2)}
\newcommand{\smoothCompactlySupportedFunctions}[2]{C^{\infty}_{c}(#1,#2)}
\newcommand{\Cpvar}[1][p]{C^{{#1}\text{-var}}}
\newcommand{\Contpvar}[3][p]{C^{{#1}\text{-var}}({#2},{#3})}
\newcommand{\diracDelta}{\delta}
\newcommand{\HoelderPaths}[1][\alpha]{C^{{#1}-\text{H\"ol}}}
\newcommand{\approxAdditivepVariation}{AA_{p\text{-var}}}
\newcommand{\semigroupP}{\mathtt{P}}
\newcommand{\semigroupT}{\mathtt{T}}
\newcommand{\generatorL}{{\mathsf{L}}}
\newcommand{\generatorA}{\mathsf{A}}
\newcommand{\CalphaHoelderLoc}{C^{\alpha\text{-H\"ol}}_{\text{loc}}}
\newcommand{\ContFunctionsOfEllipticPDEregularity}{\mathcal{C}^{\alpha}}

%notation Probability
\newcommand{\Prob}{{{P}}}
\newcommand{\probabilityLaw}{\text{Law}}
\newcommand{\probabilityQ}{{{Q}}}
\newcommand{\probabilityDensity}{{p}}
\newcommand{\Expectation}{{{E}}}
\newcommand{\Variance}{\mathrm{Var}}
\newcommand{\CoVariance}{\mathrm{Cov}}
\newcommand{\correlation}{\mathrm{corr}}
\newcommand{\likelihood}{\mathcal{L}}
\newcommand{\sigmaAlgebra}{\mathfrak{F}}
\newcommand{\measurableSpace}{\big(\Omega,\sigmaAlgebra \big)}
\newcommand{\probabilitySpace}{\big(\Omega,\sigmaAlgebra, \Prob \big)}
\newcommand{\filteredMeasurableSpace}{\big(\Omega,\sigmaAlgebra, (\sigmaAlgebra_t)_t \big)}
\newcommand{\stochasticBase}{\big(\Omega,\sigmaAlgebra = (\sigmaAlgebra_t)_t , \Prob  \big)}
\newcommand{\filtrationF}{\mathfrak{F}}
\newcommand{\iid}{\overset{\text{i.i.d.}}{\sim}}
\newcommand{\gaussian}[2]{\mathcal{N}({#1},{#2})}
\newcommand{\normalPDF}[3]{p_{\gaussian{#2}{#3}}\left( #1 \right)}
\newcommand{\WienerMeasure}[1][ ]{\mu_{#1}}
\newcommand{\brownianMotion}{W}
\newcommand{\geometricBrownianMotion}{X}
\newcommand{\compensator}{\Lambda}
\newcommand{\EDFfun}[1][e]{\hat{F}_{#1}}
\newcommand{\dirichletLaw}[1][\dirparam]{\Prob_{#1}}
\newcommand{\dirichletDensity}[1][\dirparam]{f_{#1}}
\newcommand{\targetDensity}{f}
\newcommand{\proposalDensity}{g}

%Indicator function%
\def\one{\mbox{1\hspace{-4.25pt}\fontsize{12}{14.4}\selectfont\textrm{1}}}


%notation rough paths
\newcommand{\control}{\omega}
\newcommand{\roughX}[1][X]{\mathbf{{#1}}}
\newcommand{\secondOrderX}[1][X]{\mathbb{\MakeUppercase{#1}}}
\newcommand{\secondOrderXst}[1][X]{\mathbb{\MakeUppercase{#1}}_{s,t}}
\newcommand{\roughPair}[1][X]{\mathbf{{#1}}=(#1,\mathbb{{#1}})}
\newcommand{\roughBracket}[1][X]{\left[\roughX[#1]\right]}
\newcommand{\roughBracketst}[1][X]{\left[\roughX[#1]\right]_{s,t}}
\newcommand{\roughBracketsu}[1][X]{\left[\roughX[#1]\right]_{s,u}}
\newcommand{\roughBracketij}[1][X]{\left[\roughX[#1]\right]^{i,j}}
\newcommand{\roughBracketijst}[1][X]{\left[\roughX[#1]\right]^{i,j}_{s,t}}
\newcommand{\formintegral}[2][x]{{^{\mathbf{\MakeUppercase{#1}}}}\! \! \!  \int {#2}}
\newcommand{\roughSpaceC}[3]{{\mathscr{C}}^{\sfrac{1}{#1}}([0,{#2}],{#3})}
\newcommand{\geomRoughSpace}[2]{{\mathscr{G}}^{\sfrac{1}{#1}}({#2})}
\newcommand{\weakgeomRoughSpace}[2]{\mathscr{W}\geomRoughSpace{#1}{#2}}
\newcommand{\geomRoughSpaceInterval}[3]{{\mathscr{G}}^{\sfrac{1}{#1}}({#2},{#3})}
\newcommand{\weakgeomRoughSpaceInterval}[3]{\mathscr{W}\geomRoughSpaceInterval{#1}{#2}{#3}}
\newcommand{\xs}{x_{s}}
\newcommand{\xt}{x_{t}}	
\newcommand{\xst}{x_{s,t}}
\newcommand{\ys}{y_{s}}
\newcommand{\yt}{y_{t}}
\newcommand{\yst}{y_{s,t}}
\newcommand{\Xs}{X_{s}}
\newcommand{\Xt}{X_{t}}	
\newcommand{\Xst}{X_{s,t}}
\newcommand{\Xuv}{X_{u,v}}
\newcommand{\Xsti}{X_{s,t}^{i}}
\newcommand{\Xstj}{X_{s,t}^{j}}
\newcommand{\Xsu}{X_{s,u}}
\newcommand{\Xut}{X_{u,t}}
\newcommand{\Ys}{Y_{s}}
\newcommand{\Yt}{Y_{t}}
\newcommand{\Yst}{Y_{s,t}}
\newcommand{\Xist}{\Xi_{s,t}}
\newcommand{\aTensorbeqXst}[1][x]{a\otimes b = \Xst[#1]}
\newcommand{\roughDistance}[3][p]{\rho_{\sfrac{1}{#1}}\left(\roughX[#2],\roughX[#3]\right)}
\newcommand{\controlledPaths}[2][X]{\mathcal{D}_{#1}^{#2}}

%notation PP16 Superhedging approach to stochastic integration
\newcommand{\lambdaAdmissible}[1][\lambda]{\mathcal{H}_{#1}}
\newcommand{\superhedgingP}{\bar{P}}
\newcommand{\superhedgingQ}{\bar{Q}}
\newcommand{\simpleCapitalProcess}[1][t]{\mathfrak{K}_{#1}}
\newcommand{\positiveCapitalProcess}[1][t]{\mathfrak{G}_{#1}}

%Finance 
\newcommand{\pricep}{\mathfrak{p}}
\newcommand{\portfolioValue}{V}
\newcommand{\discountedPortfolioValue}{\tilde{\portfolioValue}}
\newcommand{\DeltaHedge}{\mathtt{Delta}}
\newcommand{\GammaHedge}{\mathtt{Gamma}}
\newcommand{\interestRate}{r}
\newcommand{\risklessAsset}{S^{0}}
\newcommand{\pricePath}{S}
\newcommand{\enhancedPricePath}{\roughX[\pricePath]}
\newcommand{\priceVolatilityEnhancer}{\roughBracket[\pricePath]}
\newcommand{\enhancedPricePair}{\roughPair[\pricePath]}
\newcommand{\discountedPricePath}{\tilde{S}}
\newcommand{\discountedPriceVolatilityEnhancer}{\roughBracket[\DiscountedPricePath]}
\newcommand{\volatilityCoefficient}{\sigma}
\newcommand{\volatilityOperator}{\generatorA}
\newcommand{\pricingMeasure}{\probabilityQ}
\newcommand{\pricingExpectation}{\Expectation_{\pricingMeasure}}
\newcommand{\linearDrag}{\ell}
\newcommand{\slopeLinearDrag}{\dot{\linearDrag}}
\newcommand{\deceptiveArithmeticBM}{Z}
\newcommand{\BSsolutionInForwardPrices}{w}
\newcommand{\BSsolutionForPortfolioFunction}{v}
\newcommand{\benchmarkMarkovianModel}{\left( \left(\Omega,\sigmaAlgebra, \pricingMeasure, (\sigmaAlgebra_t)_t \right), \volatilityOperator \right)}


%Notation for 'pipest'
\newcommand{\numEventTypes}{d_{E}}
\newcommand{\numStates}{d_{S}}
\newcommand{\arrivalTimes}[1][e]{T^{#1}}
\newcommand{\event}[1][n]{E_{#1}}
\newcommand{\countingProc}[1][e]{N_{#1}}
\newcommand{\multiCountingProc}{N}
\newcommand{\groundProc}{N_{\mathfrak{g}}}
\newcommand{\hybridHawkes}{\tilde{N}}
\newcommand{\hybridIntensity}{\tilde{\intensity[]}}
\newcommand{\intensity}[1][e]{\lambda_{#1}}
\newcommand{\baseRate}{\nu}
\newcommand{\hawkesKernel}{\kappa}
\newcommand{\parametricHawkesKernel}{k}
\newcommand{\impCoef}{\alpha}
\newcommand{\decCoef}{\beta}
\newcommand{\dirichletParam}{\gamma}
\newcommand{\dirparam}{\dirichletParam}
\newcommand{\eone}{e\derivative}
\newcommand{\subscriptee}{_{\eone,\, e}}
\newcommand{\xone}{x\derivative}
\newcommand{\subscriptexe}{_{\eone,\, \xone,\, e}}
\newcommand{\stateSpace}{\mathcal{S}}
\newcommand{\stateVariable}{X}
\newcommand{\transProb}{\phi}
\newcommand{\lplus}{\ell_{+}}
\newcommand{\lminus}{\ell_{-}}
\newcommand{\ESSE}{S}
\newcommand{\ESSEone}{S^{(1)}}
\newcommand{\ESSEtwo}{S^{(2)}}
\newcommand{\ESSEthree}{S^{(3)}}

% %Notation peculiar to StaticVsDynamic
% \newcommand{\spaceInventoryTrajectories}{\mathcal{Q}}
% \newcommand{\fundamentalPrice}{S}
% \newcommand{\executionPrice}{S}
% \newcommand{\liquiditySignal}{I}
% \newcommand{\inventory}{q}
% \newcommand{\optimalInventory}{\hat{\inventory}}
% \newcommand{\initialInventory}{\mathfrak{\inventory_0}}
% \newcommand{\initialInventoryAtTimeT}{\mathfrak{\inventory}_t}
% \newcommand{\liquidationTarget}{\mathfrak{\inventory}_{\mathtt{T} } }
% \newcommand{\inventoryRate}{\dot{\inventory}}
% \newcommand{\revenues}{C}
% \newcommand{\expectedRAR}{\Expectation \mathtt{RAR}}
% \newcommand{\spaceExecutionRates}{\mathcal{R}}
% \newcommand{\coeffMarketImpact}{c_{1}}
% \newcommand{\coeffRiskAversion}{c_{2}}
% \newcommand{\coeffPermanentImpact}{c_{4}}
% \newcommand{\lagrangian}{L}
% \newcommand{\normInventoryTrjectories}[1]{\lVert{#1}\rVert_{\coeffMarketImpact, \coeffRiskAversion}}
% \newcommand{\ratioAversionOverImpact}{c_{3}}
% \newcommand{\momentumOfExecution}{p}
% \newcommand{\costFunctional}{J}
% \newcommand{\spaceInventoryTrajectoriesPathwise}{\spaceInventoryTrajectories_{\text{pw}}}
% \newcommand{\spaceInventoryTrajectoriesFuel}{\spaceInventoryTrajectories_{\text{fuel}}}
% \newcommand{\spaceInventoryTrajectoriesStatic}{\spaceInventoryTrajectories_{\text{static}}}
% \newcommand{\spaceExecutionRatesMarkov}{\spaceExecutionRates_{\text{markov}}}
% \newcommand{\spaceInventoryTrajectoriesFuelMarkov}{\spaceInventoryTrajectories_{\text{fuel, markov}}}
% \newcommand{\spaceUnbiasedInventoryTrajectories}{\mathcal{U}}
% \newcommand{\spaceUnbiasedInventoryTrajectoriesInitialConstraint}{\spaceUnbiasedInventoryTrajectories^{0,\initialInventory} }
% % \newcommand{\stateVariable}{X}
% \newcommand{\actualPricePath}{x }
% \newcommand{\implementedInventory}{\inventory^{\actualPricePath} }
% \newcommand{\tildeImplementedInventory}{\tilde{\inventory}^{\actualPricePath} }
% \newcommand{\coeffPenalisationOutstandingInventory}{c_{5}}
% \newcommand{\ratioTerminalPenalisationOverImpact}{c_{6}}
% 
% \newcommand{\LTIF}{\mathtt{LTIF}}
% \newcommand{\LTIP}{\mathtt{LTIP}}
% \newcommand{\martLTIF}{\text{mart-}\mathtt{LTIF} }
% \newcommand{\martLTIP}{\text{mart-}\mathtt{LTIP} }
% \newcommand{\detLTIF}{\text{det-}\mathtt{LTIF} }
% \newcommand{\detLTIP}{\text{det-}\mathtt{LTIP} }
% \newcommand{\semimartLTIP}{\text{semimart-}\mathtt{LTIP}}
% \newcommand{\semimartLTIF}{\text{semimart-}\mathtt{LTIF}}
% \newcommand{\markovLTIP}{\text{markov-}\mathtt{LTIP}}
% \newcommand{\markovLTIF}{\text{markov-}\mathtt{LTIF}}
% \newcommand{\LTI}{\mathtt{LTI} }
% \newcommand{\randomLTIP}{\text{random-}\mathtt{LTIP}}
% \newcommand{\randomLTIF}{\text{random-}\mathtt{LTIF}}


% notation peculiar to "Sectional cost of Trade Execution"
% Notation for Limit Order Book
\newcommand{\price}{P}
\newcommand{\tickSizeOfLOB}{\tau}
\newcommand{\midprice}{\midPrice}
\newcommand{\midPrice}{\price^{m}}
\newcommand{\volume}{V}
\newcommand{\bidPriceLevelN}[1][1]{\price^{b,{#1}}}
\newcommand{\askPriceLevelN}[1][1]{\price^{a,{#1}}}
\newcommand{\bidVolumeLevelN}[1][1]{\volume^{b,{#1}}}
\newcommand{\askVolumeLevelN}[1][1]{\volume^{a,{#1}}}
\newcommand{\bestBidPrice}{\price^{b}}
\newcommand{\nthBestBidPrice}[1][1]{\price^{b,{#1}}}
\newcommand{\bestAskPrice}{\price^{a}}
\newcommand{\nthBestAskPrice}[1][1]{\price^{a,{#1}}}
\newcommand{\bestBidVolume}{\volume^{b}}
\newcommand{\nthBestBidVolume}[1][1]{\volume^{b,{#1}}}
\newcommand{\bestAskVolume}{\volume^{a}}
\newcommand{\nthBestAskVolume}[1][1]{\volume^{a,{#1}}}
\newcommand{\volumeImbalance}{I}
\newcommand{\queueImb}{\volumeImbalance}
\newcommand{\OFI}{\text{OFI}}
\newcommand{\orderFlowImbalance}{\OFI}
\newcommand{\quantityToLiquidate}{Q}
\newcommand{\quantityToExecute}{X}
\newcommand{\costOfImpact}{C^{\text{impact}}}
\newcommand{\costOfSplitting}{C^{\text{split}}}
\newcommand{\discountedCostOfImpact}{\tilde{C}}
\newcommand{\coeffOfAversionToMarketImpact}{\kappa}
\newcommand{\coeffCostOfDepthConsumption}{c_1}
\newcommand{\coeffCostOfVolumeImbalance}{c_2}
\newcommand{\coeffCostOfSplitting}{c_3}
\newcommand{\sectionalCostOfExecution}{C}
\newcommand{\sectionalCostOfExecutionFunction}{H}
\newcommand{\LOBspread}{\phi}


   
\begin{document} 
\numberwithin{equation}{section}
\maketitle
\tableofcontents

\section{Introduction}
Include brief discussion on LOB inspired by \citealp[Chapter 6]{FPR13mar} and \citealp{GPWMFH13lim}.

Make the point that non-average-based measures of  price impact require market simulators and thus a large component of our model is in fact a high-frequency market simulator. Some comments on the fact that price impact is largely regarded as an average phenomenon can be found in \citealp{BMBB19imp}. 

\section{Preliminaries on the theory of counting processes} 
In this section we introduce the adopted notation by reviewing basic concepts from the theory of counting processes. Our main reference is \citealp[Chapter 14]{DVJ08int}.

Let $\numEventTypes$ be a positive integer. For each $e$ ranging from $1$ to $\numEventTypes$, let $\arrivalTimes_{j}$, $j=1, 2, \dots$, be a strictly increasing sequence of positive random times, and assume that $\arrivalTimes_{j}\neq\arrivalTimes[e\derivative]_{j\derivative}$ if $(e,j)\neq (e\derivative,j\derivative)$. Then, 
\begin{equation*}
 \countingProc(t):= \sum_{j} \one\left\lbrace\arrivalTimes_{j}\leq t \right\rbrace, \qquad t\geq 0,
\end{equation*}
is a non-decreasing right-continuous process; we call $\countingProc$ the counting process associated with the sequence $(\arrivalTimes_{j})_j$. Notice that $(\arrivalTimes_{j})_j$ can be retrieved from $\countingProc$ by 
\begin{equation*}
 \arrivalTimes_{j} = \inf \left\lbrace t>0: \, \countingProc(t)\geq j\right\rbrace;
\end{equation*}
hence, there is a one-to-one correspondence between $\countingProc$ and $(\arrivalTimes_{j})_j$. 

For $t>0$ we define 
\begin{equation*}
 \delta\countingProc(t):= \lim_{h\downarrow 0} \Big( \countingProc(t) - \countingProc(t-h) \Big),
\end{equation*}
and we notice that $\delta \countingProc(t) = 1$ if and only if $t=\arrivalTimes_{j}$ for some $j$, otherwise $\delta \countingProc(t) = 0$.

The $\numEventTypes$-dimensional vector $\multiCountingProc(t) = (\countingProc[1](t), \dots , \countingProc[\numEventTypes](t))$ is referred to as multivariate counting process associated with the $\numEventTypes$ sequences $(\arrivalTimes_{j})_j$, $e=1,\dots, \numEventTypes$. Let $\groundProc(t) := \countingProc[1](t)+ \dots + \countingProc[\numEventTypes](t)$ be the ground process of $\multiCountingProc$, and let 
\begin{equation*}
 \arrivalTimes[ ]_{n} := \inf \left\lbrace t>0: \, \groundProc(t) \geq n\right\rbrace, \qquad n=1,2,\dots, 
\end{equation*}
be the ordered sequence of random times stemming from the union $\lbrace \arrivalTimes_{j}: \, j=1,2,\dots; \,\, e=1,\dots,\numEventTypes\rbrace$. By defining for $n=1,2, \dots$, 
\begin{equation*}
 \event := \sum_{e=1}^{\numEventTypes} e \, \one\left\lbrace \delta \groundProc(\arrivalTimes[ ]_{n}) = \delta\countingProc(\arrivalTimes[ ]_{n}) \right\rbrace,
\end{equation*}
we have that the pair $(\arrivalTimes[ ]_{n}, \event[n])$ equivalently characterises the multivariate counting process, because 
\begin{equation}\label{eq.NTE_counting_proc}
 \countingProc(t) = \sum_{n} \one \left\lbrace \arrivalTimes[ ]_{n}\leq t, \, \event[n] = e\right\rbrace,
\end{equation}
for all $t>0$ and all $e=1,\dots, \numEventTypes$.

We can interpret this construction by saying that the index $e$ labels $\numEventTypes$ types of events that occur in time, and $\countingProc(t)$ counts the number of events of type $e$ that have occurred by time $t$. 

\begin{example}[``Poisson process'']
 Let $\tau_e^j$, $j=1,2,\dots$, $\, \, e=1,\dots, \numEventTypes$ be independent random variables such that $\tau_e^j$ is exponentially distributed with parameter $\intensity>0$, $\, e=1,\dots, \numEventTypes$. Let $\arrivalTimes_{j} := \sum_{k\leq j} \tau_e^{k}$, and notice that $\arrivalTimes_{j}$ has probability density function 
 \begin{equation*}
  f_{e,j}(t) = \frac{\intensity^j}{(j-1)!}t^{j-1} e^{-\intensity t} \one\lbrace t>0\rbrace.
 \end{equation*}
 Then, the multivariate counting process $\multiCountingProc$ associated with the arrival times $\arrivalTimes_{j}$ is called $\numEventTypes$-dimensional Poisson process of rates $\intensity[1], \dots \intensity[\numEventTypes]$. This name is justified as follows. 
 Since $\lbrace \countingProc(t)\geq j\rbrace = \lbrace \arrivalTimes_{j} \leq t\rbrace$, we have that $\frac{d}{dt} \Prob(\countingProc(t) \geq j) = f_{e,j} (t)$. On the other hand, if we define 
 \begin{equation*}
  S_{e,j}(t) := \sum_{k\geq j} \frac{(\intensity t)^k}{k!}e^{-\intensity t},
 \end{equation*}
we also have that $\frac{d}{dt} S_{e,j}(t) = f_{e,j}(t)$, by telescopic sum. Since $S_{e,j}(0) = \Prob (\countingProc(t)\geq 0)$, we deduce that $\Prob(\countingProc(t) \geq j) = S_{e,j}(t)$, and that
\begin{equation*}
 \Prob\left( \countingProc(t) = j\right) = 
 \frac{(\intensity t)^j}{j!} \exp\left( -\intensity t\right).
\end{equation*}
Therefore for every $t$, $\countingProc(t)$ is a Poisson random variable of parameter $\intensity t$, and the ground process $\groundProc$ of $\multiCountingProc$ is such that for every $t$, $\groundProc(t) \sim \text{Pois}(\intensity[1]t + \dots + \intensity[\numEventTypes]t)$.
\end{example}

The minimal filtration to which a multivariate counting process $\multiCountingProc$ is adapted -- and such that it satisfies the usual conditions of completeness and right-continuity -- is called the internal history of $\multiCountingProc$. Any other filtration to which $\multiCountingProc$ is adapted is called a history of $\multiCountingProc$, and it must be a superset of the internal history. 

\begin{defi}\label{def.compensator}
 Let $\stochasticBase$ be a filtered probability space where the multivariate counting process $\multiCountingProc$ is defined, and assume that $\filtrationF$ is a history of $\multiCountingProc$. We say that the $\numEventTypes$-dimensional stochastic process $\compensator = (\compensator_1, \dots , \compensator_{\numEventTypes})$ is an $\filtrationF$-compensator for $\multiCountingProc$ if: (o) $\compensator(0)=0$ and $\compensator$ is of finite variation; (i) $\compensator$ is $\filtrationF$-predictable; (ii) $\compensator$ is right-continuous; (iii) $\multiCountingProc - \compensator$ is a local martingale. 
\end{defi}

Since a counting process $\multiCountingProc$ is non-decreasing, as soon as an $\filtrationF$-compensator $\compensator$ for $\multiCountingProc$ exists, not only it is of finite variation as required by Definition \ref{def.compensator}, but in fact it must be non-decreasing too. Furthermore, because of requirements (i) and (ii), the discontinuities of $\compensator$ -- if any -- are typically associated with the occurrence of deterministic events, such as fixed atoms in the distribution of $\multiCountingProc$. In the following, our concern will be on continuous compensators.

Given the counting process $\multiCountingProc$ and a history $\filtrationF$, the $\filtrationF$-compensator is unique up to an evanescent set, and it is equivalently characterised as the $\filtrationF$-predictable projection of $\multiCountingProc$, namely as the $\filtrationF$-predictable non-decreasing process $\compensator$ such that 
\begin{equation}\label{eq.compensators_and_predict_projections}
 \Expectation \left[ \int_{\R_+} Y d\multiCountingProc\right]
 = 
 \Expectation \left[ \int_{\R_+} Y d\compensator\right]
\end{equation}
for all non-negative $\filtrationF$-predictable processes $Y$ (see \citealp[Proposition 14.2.II]{DVJ08int}). 

Because of requirement \textit{(o)} in Definition \ref{def.compensator}, as soon as a compensator exists and is continuous, it can be written as 
\begin{equation*}
 \compensator (t) = \intzerot \intensity[ ](s)ds,
\end{equation*}
for some $\filtrationF$-predictable process $\intensity[ ]=(\intensity[1],\dots,\intensity[\numEventTypes])$, which is called intensity of the counting process $\multiCountingProc$. Combining this with equation \eqref{eq.compensators_and_predict_projections}, one obtains the formula 
\begin{equation*}
 \Expectation\left[ \countingProc(t)-\countingProc(s) \vert \filtrationF_s \right]
 =
 \Expectation\left[ \int_{s}^{t} \intensity(u) du
  \vert \filtrationF_s \right],
  \qquad  s\leq t,
\end{equation*}
which allows to interpret $\intensity(t)$ as a measure of the ``instantaneous risk'' of a jump at time $t$ in the $e$-th component of the counting process $\multiCountingProc$. Notice that this ``risk'' evolves in time and it varies depending on the information available up to time $s$. 

Compensators are crucial in the following time-change result, which will be used to perform goodness-of-fit diagnostics.

\begin{thm}[{\citealp{Mey71dem}}]\label{thm.meyer1971}
Let $\multiCountingProc$ be a $\numEventTypes$-dimensional counting process with arrival times $\arrivalTimes$. Assume that $\multiCountingProc$ has continuous compensator $\compensator$ such that $\compensator_{e}(t) \rightarrow \infty$ as $t\rightarrow \infty$ for all $e=1, \dots, \numEventTypes$. Then, the random sequences $\lbrace \compensator(\arrivalTimes_{j}): \, j=1,2,\dots \rbrace$, $e=1,\dots, \numEventTypes$ are the arrival times of a $\numEventTypes$-dimensional unit-rate Poisson process, namely the time-changed inter-arrival times
\begin{equation}\label{eq.time-changed_intertimes}
 \tau_e^j := \compensator(\arrivalTimes_{j}) - \compensator(\arrivalTimes_{j-1})
\end{equation}
are all independent exponentially distributed random variables for $j=1,2,\dots$ and $e=1,\dots,\numEventTypes$.  
\end{thm}
For a proof of Theorem \ref{thm.meyer1971}, see \citealp{BN88sim}. 

Testing the distribution of a multivariate counting process satisfying the assumptions of Theorem \ref{thm.meyer1971} can therefore be reduced to testing that the variables in equation \eqref{eq.time-changed_intertimes} are i.i.d samples from a unit rate exponential distribution. The latter can be based on EDF statistics (see e.g. \citealp[Chapter 4]{DS86goo}, as explained in the rest of the section. 

Given $\numEventTypes$ samples, each of $n_e$ data points $\arrivalTimes_{j}$, $j=1,\dots,n_e$, make the null hypothesis that they are generated by a multivariate counting process satisfying the assumptions of Theorem \ref{thm.meyer1971}. Define $\tau_e^j$ as in equation \eqref{eq.time-changed_intertimes}, and let $\EDFfun$ be the empirical distribution function associated with the sample $(\tau_e^j)_j$, namely 
\begin{equation*}
 \EDFfun(t) = \frac{1}{n_e}\sum_{j=1}^{n_e} \one \left\lbrace \tau_e^j \leq t \right\rbrace.
\end{equation*}
If $(\tau_e^j)_j$ are independent samples from a unit-rate exponential distribution and if $n_e$ is sufficiently large, then $\EDFfun$ is expected to be close to $F(t)=1-e^{-t}$. 
Let $d$ be a distance between cumulative distribution functions. We call $c_e:=d(\EDFfun, F)$ the critical value associated with the sample. 

On a probability space $\probabilitySpace$, let $\sigma_i$ be $n_e$ i.i.d. random variables $\sim \text{Exp}(1)$, $i=1,\dots, n_e$, and define the random EDF 
\begin{equation*}
 F^{\sigma}(t) = \frac{1}{n_e}\sum_{i=1}^{n_e} \one \left\lbrace \sigma_i \leq t\right\rbrace.
\end{equation*}

We say that the null hypothesis that $\arrivalTimes_{j}$, $j=1, \dots, n_e$, are generated by the $e$-th component $\countingProc$ of $\multiCountingProc$ cannot be rejected with confidence greater than $100\text{x}(1-\alpha)\text{\%}$ if 
\begin{equation*}
 \Prob\left( d(F^\sigma, F) \geq c_e \right) \leq \alpha.
\end{equation*}
Since the function $c \mapsto \Prob\left( d(F^\sigma, F) \geq c \right)$ is non-decreasing, the smaller $d(\EDFfun, F)$ is, the less confidently the null hypothesis can be rejected. Therefore, given the test statistics $c_e:=d(\EDFfun, F)$, its associated p-value $\alpha(d(\EDFfun,F)) := \Prob(d(F^\sigma, F) \geq d(\EDFfun,F))$ is such that $100\text{x}(1-\text{p-value})\text{\%}$ is the highest level of confidence with which the null hypothesis can be rejected.

Furthermore, the independence among the components $e=1,\dots, \numEventTypes$ can be assessed by drawing correlograms, and by testing that the time-changed inter-arrival times of the ground process are i.i.d. with $\sim \text{Exp}(\numEventTypes)$.

\section{State-dependent Hawkes processes}
A $\numEventTypes$-dimensional counting process $\multiCountingProc$ is called Hawkes process if it admits a continuous compensator $\compensator$ with intensities 
\begin{equation}\label{eq.intensity_ordHawkes}
 \intensity(t) = \baseRate_e + \sum_{\eone=1}^{\numEventTypes}\intzerot \hawkesKernel\subscriptee(t-s)d\countingProc[\eone](s), \qquad e=1,\dots,\numEventTypes,
\end{equation}
for some non-negative base rates $\baseRate_e \geq 0$, and some non-negative locally integrable functions $\hawkesKernel\subscriptee \geq 0$ that are supported on the non-negative half line.

The matrix-valued function $t\mapsto [\hawkesKernel\subscriptee(t)]_{e,\eone = 1, \dots, \numEventTypes}$ is referred to as the kernel of the Hawkes process $\multiCountingProc$.  If all the kernel functions are integrable, the spectral radius $\rho$ of the $\numEventTypes\times\numEventTypes$-matrix of $\Lone$ norms $\Vert \hawkesKernel\subscriptee \Vert_{1}$ is called radius of the Hawkes kernel; if some of the kernel functions are not integrable, the spectral radius is set to $+\infty$.  

A $\numEventTypes$-dimensional Hawkes process is asymptotically stationary if the radius of its kernel is smaller than $1$; in this case the the intensity process $\intensity[ ]$ is asymptotically stationary.  

A model based on Hawkes processes can capture a phenomenon's event-driven nature, and the possibly complex ways in which past events affect the occurrence of future events. Because of the mechanism through which orderbooks operate, Hawkes processes are therefore well suited to model order dynamics in such trading venues and the resulting time evolutions of prices. Yet, a modeller might need to include more in her work. More accurate models will indeed be obtained by recording not only the events' arrival times, but also the state of the system every time such events occurred. The rationale for introducing state-dependent Hawkes processes is based on this additional need. Their study was initiated by \citealp{MP20hyb} and \citealp{MP18sta}. Following these works, we introduce state-dependent Hawkes  processes in the next section (Section \ref{sec.sdHawkes_def_and_props}), we present a general simulation algorithm in Section \ref{sec.simulation},  we discuss parametric families of state-dependent Hawkes processes in Section \ref{sec.parametric_hawkes}, and among these families we select the case of power-law kernels to focus on in Section \ref{sec.powerlaw_kernels}.

\subsection{Definition and main properties}\label{sec.sdHawkes_def_and_props}
Let $\stateSpace$ be a state space. We will always assume that $\stateSpace$ is finite, so that we can label its elements as $x=1, \dots, \numStates$, where $\numStates$ is the number of possible states of the system. A state-dependent counting process will be a pair $(\multiCountingProc, \stateVariable)$, where for all $t$, $\multiCountingProc(t)$ records the number of events occurred by time $t$ as per formula \eqref{eq.NTE_counting_proc}, and $\stateVariable(t)$ records the state of the system at time $t$. More specifically, we have
\begin{defi}[{\citealp[Definition 2.1]{MP18sta}}]\label{def.sdHawkes}
 Let $\multiCountingProc$ be a $\numEventTypes$-dimensional counting process. Let $\stateVariable$ be a continuous-time piecewise-constant process in the finite state space $\stateSpace$ of cardinality $\numStates$. Let $\internalHistory$ be the minimal complete right-continuous filtration generated by the pair $(\multiCountingProc, \stateVariable)$, which is referred to as internal history. Then, we say that $(\multiCountingProc, \stateVariable)$ is a state-dependent Hawkes process if 
 \begin{enumerate}
  \item $\multiCountingProc$ admits a $\internalHistory$-continuous compensator with intensities
  \begin{equation}\label{eq.intensity_of_sdHawkes}
   \intensity(t) = \baseRate_e + \sum_{\eone=1}^{\numEventTypes}\int_{[0,t)} \hawkesKernel\subscriptee(t-s, \stateVariable(s))d\countingProc[\eone](s), \qquad e=1,\dots,\numEventTypes,
  \end{equation}
  for some $\numEventTypes$ non-negative base rates $\baseRate_e\geq 0$, $\, e=1, \dots, \numEventTypes$, and some measurable functions $\hawkesKernel: \R_+ \times \stateSpace \rightarrow \R_+$ such that $\hawkesKernel(\cdot, x)$ is locally integrable for all $x$ in $\stateSpace$;
  \item $\stateVariable$ jumps only at arrival times $\arrivalTimes[ ]_{n}$ of $\multiCountingProc$, and there exist $\numEventTypes$ transition matrices $\transProb_e(\cdot,\cdot)$, $e=1,\dots, \numEventTypes$, defined on $\stateSpace$ such that for all $n$
  \begin{equation}\label{eq.markov_update_stateVariable}
   \Prob\left( \stateVariable(\arrivalTimes[ ]_{n})=x \,  \vert \,  E_n, \, \internalHistory_{\arrivalTimes[ ]_{n}-} \right)
   =
   \transProb_{E_n}\left(\stateVariable(\arrivalTimes[ ]_{n}-),x\right),
   \qquad
   x=1, \dots, \numStates,
  \end{equation}
  where $\stateVariable(\arrivalTimes[ ]_{n}- ) = \lim_{t\uparrow \arrivalTimes[ ]_{n}} \stateVariable(t)$ is the state of the system immediately before the $n$-th event $E_n$, and $\internalHistory_{\arrivalTimes[ ]_{n}-} = \bigvee_{\epsilon>0} \internalHistory_{\arrivalTimes[ ]_{n} - \epsilon}$ represents the information available immediately before this event.
 \end{enumerate}
\end{defi}

Given a $\numEventTypes$-dimensional vector $\baseRate$ with non-negative components, given a $\numEventTypes\times\numEventTypes$-dimensional matrix valued function $\hawkesKernel$ defined on $\R_{+} \times \stateSpace$, and given $\numEventTypes$ transition matrices $\transProb_e$, $e=1,\dots,\numEventTypes$, defined on $\stateSpace$, conditions for the existence of a pair $(\multiCountingProc,\stateVariable)$ that satisfies the requirements of Definition \ref{def.sdHawkes} with $\baseRate$ as vector of base rates, $\hawkesKernel$ as matrix of Hawkes kernels and $\transProb$ as transition probabilities are obtained by relying on existence results for the more general class of hybrid marked point processes introduced in \citealp{MP20hyb} -- see \citealp[Theorem 2.3]{MP18sta}. We will use the following sufficient condition for existence, uniqueness and non-explosiveness.
\begin{prop}[{\citealp[Theorem 2.3]{MP18sta}}]\label{prop.existence_sdHawkes}
 Let $\baseRate$ be a $\numEventTypes$-dimensional vector with non-negative components; let $\hawkesKernel\subscriptee$ be non-negative functions defined on $\R_{+} \times \stateSpace$, where the indexes $e$ and $\eone$ range from $1$ to $\numEventTypes$; let $\transProb_e$ be transition matrices defined on the state space $\stateSpace$, for $e=1,\dots,\numEventTypes$. If the functions $\hawkesKernel\subscriptee$ are bounded for all $e,\eone=1,\dots,\numEventTypes$, then a state-dependent Hawkes process $(\multiCountingProc, \stateVariable)$ with base rates $\baseRate_e$, kernels $\hawkesKernel\subscriptee$ and transition probabilities $\transProb_e$ exists, and such state-dependent Hawkes process is unique and non-explosive. 
\end{prop}

The mentioned link with the general class of hybrid marked point processes is obtained by the introduction of the following $\numEventTypes\numStates$-dimensional counting process $\hybridHawkes$. It gives a complementary point of view on state-dependent Hawkes processes, to which we will refer in the following. We define it here and we set the notation for later use.

Given a state-dependent Hawkes process $(\multiCountingProc, \stateVariable)$ as in Definition \ref{def.sdHawkes}, let $\arrivalTimes[]_n$ and $\event[n]$ be the sequences of arrival times and events that equivalently describe the counting process component $\multiCountingProc$ of the pair $(\multiCountingProc, \stateVariable)$, as per equation \eqref{eq.NTE_counting_proc}. Let $\stateVariable_n$ be the sequence of states $X(\arrivalTimes[]_n)$, for $n=1,2,\dots$. Then, the 
$\numEventTypes\numStates$-dimensional counting process
\begin{equation}\label{eq.hybridHawkes}
 \hybridHawkes_{e,x}(t) := 
 \sum_{n} \one \left\lbrace \arrivalTimes[]_n \leq t, \, \event[n] = e, \, 
 \stateVariable_n = x \right\rbrace
\end{equation}
is called the hybrid-MPP counterpart of  $(\multiCountingProc, \stateVariable)$.
We have that the $j$-th jump time $\arrivalTimes[e]_j$ of the $e$-th component of $\multiCountingProc$ is the $j$-th order statistic of $\lbrace \arrivalTimes[e,x]_k: \, k=1,2,\dots; \, \, x=1,\dots,\numStates \rbrace$, where $(\arrivalTimes[e,x]_k)_k$ are the jump times of the $(e,x)$-th component of $\hybridHawkes$. Similarly, $\arrivalTimes[]_n$ is the $n$-th order statistics of $\lbrace \arrivalTimes[e,x]_k: \, k=1,2,\dots; \, \, e=1,\dots, \numEventTypes; \, \,  x=1,\dots,\numStates \rbrace$. The $(e,x)$-th component $ \hybridHawkes_{e,x}$ of the hybrid-MPP counterpart of $(\multiCountingProc, \stateVariable)$ admits a continuous compensator with density given by 
\begin{equation}\label{eq.intensity_of_hybridMPP}
 \hybridIntensity_{e,x}(t) 
 =
 \transProb_{e}\left(\stateVariable(t),x\right)
 \left(
 \baseRate_e
 +\sum_{\eone,\, \xone} \int_{[0,t)} \hawkesKernel\subscriptee(t-s,\xone)d\hybridHawkes_{\eone,\xone}(s)
 \right),
\end{equation}
where $\transProb_{e}$ is the transition matrix associated with event type $e$, and $\hawkesKernel\subscriptee$, for $e,\, \, \eone = 1,\dots,\numEventTypes$, are the Hawkes kernels of $\multiCountingProc$.


\subsection{Simulation}\label{sec.simulation}
In \citealp{MP18sta}, the authors describe a simulation algorithm for state-dependent Hawkes process based on Ogata's thinning algorithm (\citealp{Oga81lew}). We recall the iterative step of this algorithm here, which will later serve as the foundation for our simulation of order executions in a limit order book.

In the following algorithm, the triple $(\arrivalTimes[]_n,\event[n],\stateVariable_n)$ refers to the $n$-th elements of the sequences of arrival times, events labels and states introduced above, which equivalently represent a state-dependent Hawkes process $(\multiCountingProc, \stateVariable)$. Moreover, for every $t$ we let $\bar{\intensity[]}(t):= \intensity[1](t) + \dots + \intensity[\numEventTypes](t)$ be the sum of the intensities of the $\numEventTypes$ components of $\multiCountingProc$. Notice that the computation of $\bar{\intensity[]}(t)$ requires $\lbrace(\arrivalTimes[]_n,\event[n],\stateVariable_n): \, \arrivalTimes[]_n <t\rbrace$. Moreover, we assume that $\bar{\intensity[]}$ is decreasing in time. 
\begin{algorithm}
 \caption{{\citealp[Algorithm 2.4]{MP18sta}}}
 \label{algo.MP18_ogata}
 \begin{algorithmic}[5]
  \REQUIRE $(\arrivalTimes[]_i, \event[i], \stateVariable_i)_{i=1,\dots,n-1}$
  \STATE set $t:=\arrivalTimes[]_{n-1}$
  \STATE set $\xi:=0$
  \WHILE{$\xi=0$}
  \STATE draw $U \sim \text{Exp}(\bar{\intensity[]}(t))$
  \STATE set $\xi :=1$ with probability $\bar{\intensity[]}(t+U)/\bar{\intensity[]}(t)$
  \STATE update $t\leftarrow t+U$
  \ENDWHILE
  \STATE set $\arrivalTimes[]_n:=t$
  \STATE draw $\event[n]$ in $\lbrace 1,\dots,\numEventTypes\rbrace$ with probabilities proportional to $\lbrace \intensity[1](\arrivalTimes[]_n), \dots, \intensity[\numEventTypes](\arrivalTimes[]_n)\rbrace$
  \STATE draw $\stateVariable_n$ in $\lbrace 1,\dots,\numStates\rbrace$ with probabilities $\lbrace \transProb_{\event}(\stateVariable_{n-1},1), \dots,\transProb_{\event}(\stateVariable_{n-1},\numStates)\rbrace$
  \RETURN $(\arrivalTimes[]_n,\event[n],\stateVariable_n)$
 \end{algorithmic}
\end{algorithm}

We will comment on the computational complexity of this algorithm in Corollary \ref{corol.complexities}. 


Future expansion: give a clustering interpretation of a state-dependent Hawkes process (generalising that valid for ordinary Hawkes processes proved in \citealp{HO74clu}) and, based on this interpretation, describe a simulation algorithm in the spirit of \citealp{MR05per}.




\subsection{Parametric families of state-dependent Hawkes processes}\label{sec.parametric_hawkes}
Proposition \ref{prop.existence_sdHawkes} says that specifying a state-dependent Hawkes process is tantamount to specifying: (i) a vector of base rates and a (bounded) Hawkes kernel for the intensities of arrival times; (ii) transition functions for the states. The vector of base rates lives in the subset $\R_{+}^{\numEventTypes}$ of the finite dimensional space $\R^{\numEventTypes}$. Moreover, since the state space is assumed finite, the transition functions for the states also live in finite dimensional spaces: there is one $\numStates\times\numStates$-dimensional transition matrix for every event type $e=1,\dots, \numEventTypes$. Therefore, to fully have parametric models based on state-dependent Hawkes processes, one needs only to select a parametric class of Hawkes kernels $\lbrace\hawkesKernel^\theta : \, \theta \in \Theta \rbrace$. In this section we set the stage for the parametric models that we will employ in the rest of our work. 

Fix a parametric class of bounded Hawkes kernels $\lbrace\hawkesKernel^\theta : \, \theta \in \Theta \rbrace$. We can talk about a log-likelihood function associated with state-dependent Hawkes processes with kernels in this class, considered as a function of the base rates $\baseRate_e$, $e=1,\dots,\numEventTypes$, of the parameter $\theta \in \Theta$, and implicitly of the $\numStates\numEventTypes\numStates$ values $\transProb_{e}(x,\xone)$, for $e=1,\dots,\numEventTypes$ and $x,\xone=1,\dots,\numStates$. Explicit expressions for log-likelihood functions of state-dependent Hawkes processes are derived from the abstract characterisation of likelihoods in terms of Janossy densities of point processes -- see \citealp[Chapters 7 and 14]{DVJ08int}.
\begin{prop}[{\citealp[Theorem 3.1]{MP18sta}}]
 Let $\lbrace\hawkesKernel^\theta : \, \theta \in \Theta \rbrace$ be a parametric family of bounded Hawkes kernels and let $\intensity^{\baseRate,\theta}=\intensity^{\baseRate,\theta}(t)$, for $e=1,\dots, \numEventTypes$, be the intensities obtained from equation \eqref{eq.intensity_of_sdHawkes} in dependence of the base rates $\baseRate_e$ and the parameter $\theta$ of the Hawkes kernel.  Given a realisation of the state-dependent Hawkes process $(\multiCountingProc, \stateVariable)$ consisting of arrival times $t_n$, events $e_n$ and states $x_n$, the log-likelihood function associated with this realisation is 
 \begin{equation}\label{eq.loglikelihood_general}
  \log \likelihood (\nu,\theta) 
  =
  \sum_{n=1}^{\groundProc(\timeHorizon)} \transProb_{e_n}(x_{n-1},x_{n})
  +\sum_{n=1}^{\groundProc(\timeHorizon)} \intensity[e_{n}]^{\baseRate,\theta}(t_n)
  -\sum_{e=1}^{\numEventTypes} \int_{0}^{\timeHorizon} \intensity^{\baseRate,\theta}(t)dt, 
 \end{equation}
where $\transProb$ are the transition probabilities of $(\multiCountingProc, \stateVariable)$, $\timeHorizon$ is the time horizon of the realisation, and $\groundProc(\timeHorizon)$ is the total number of events in the realisation, i.e. the value that the ground process of  $\multiCountingProc$ takes at time $\timeHorizon$. 
\end{prop}
The assumption of bounded Hawkes kernels in Proposition \ref{prop.parametric_hawkes_kernels} is in place to guarantee existence, uniqueness and non-explosiveness of a state-dependent Hawkes process for any choice of kernels in such family, base rates and transition probabilities.  

Let $\impCoef_{\eone,\xone, e}$ and $\decCoef_{\eone,\xone, e}$ be real numbers, with the indexes $\eone$ and $e$ ranging from $1$ to $\numEventTypes$, and the index $\xone$ ranging from $1$ to $\numStates$. Think of $\impCoef: \xone \mapsto [\impCoef_{\eone,\xone,e}]_{\eone,e}$ and of $\decCoef: \xone \mapsto [\decCoef_{\eone,\xone,e}]_{\eone,e}$ as matrix-valued functions defined on the state space $\stateSpace$. Since $\stateSpace$ is finite, the pair $(\impCoef,\decCoef)$ can in fact represent the parameter $\theta = (\impCoef,\decCoef)$ of a parametric family of Hawkes kernels; indeed we can set
\begin{equation}\label{eq.parametric_hawkes_kernel_general}
 \hawkesKernel^{\theta}\subscriptee(t,\xone)
 = \parametricHawkesKernel(t,\, \impCoef_{\eone,\xone,e},\, \decCoef_{\eone,\xone,e}),
\end{equation}
for some fixed non-negative function $\parametricHawkesKernel$. The parameters $\impCoef\subscriptexe$ will be used below to modulate the instantaneous impacts on $\intensity[e]$ of events of type $\eone$ after which the state of the system is $\xone$, namely the following update rule will hold $\intensity[e](\arrivalTimes[]_n) - \intensity[e](\arrivalTimes[]_n - ) = \sum_{\eone, \xone} \impCoef\subscriptexe \one \lbrace \delta\groundProc(\arrivalTimes[]_n) = \delta\hybridHawkes_{\eone,\xone}(\arrivalTimes[]_n)\rbrace$. Moreover, the parameters $\decCoef\subscriptexe$ will be used to modulate the decaying effect on $\intensity[e]$ of events of type $\eone$ after which the state is $\xone$.
The  following proposition specifies the formulae for the intensities and the log-likelihood in the parametric case of equation \eqref{eq.parametric_hawkes_kernel_general}, and Corollary \ref{corol.complexities} addresses their computational cost. 
\begin{prop}\label{prop.parametric_hawkes_kernels}
 Let $\lbrace\hawkesKernel^{\impCoef, \decCoef} : \, \, (\impCoef,\decCoef)=:\theta \in \Theta \rbrace$ be a parametric family of bounded Hawkes kernels as in equation \eqref{eq.parametric_hawkes_kernel_general}, for some fixed non-negative function $\parametricHawkesKernel=\parametricHawkesKernel(t,a,b)$. Then, the intensities $\intensity[e]$ in equation \eqref{eq.intensity_of_sdHawkes} are evaluated according to the formula
 \begin{equation}\label{eq.evaluation_of_intensity}
 \intensity(t) = \baseRate_e 
 +\sum_{\eone, \, \xone} \sum_{\arrivalTimes[\eone, \xone]_{j} < t} 
 \parametricHawkesKernel(t-\arrivalTimes[\eone, \xone]_{j}, \, \impCoef_{\eone,\xone, e}, \,  \decCoef_{\eone,\xone, e}),
\end{equation}
where in the outermost sum the index $\eone$ refers to event types and ranges from $1$ to $\numEventTypes$, the index $\xone$ refers to states and ranges from $1$ to $\numStates$, and where, in the innermost sum, $(\arrivalTimes[\eone, \xone]_{j})_j$ are the jump times of the $(\eone, \xone)$-th component of the hybrid-MPP counterpart $\hybridHawkes$ of $(\multiCountingProc,\stateVariable)$ defined in equation \eqref{eq.hybridHawkes}. Moreover, assuming that $0$ is the time origin of a realisation of $(\multiCountingProc, \stateVariable)$ and $\timeHorizon$ is its time horizon,  the log-likelihood is evaluated according to the formula
\begin{equation}\label{eq.evaluation_of_loglikelihood}
 \log \likelihood(\baseRate,\impCoef,\decCoef) = \Phi(0,\timeHorizon)
 + \lplus(0,\timeHorizon) - \lminus(0,\timeHorizon), 
\end{equation}
where for $t_0 < t_1$ we have defined
\begin{equation*}
\begin{split}
 \Phi(t_0,t_1) =& \sum_{t_0<\arrivalTimes[]_n \leq t_1} \transProb_{\event}\big(\stateVariable(\arrivalTimes[]_n - ), \stateVariable(\arrivalTimes[]_n)\big), 
 \\
 \lplus(t_0,t_1) =& \sum_{e=1}^{\numEventTypes} \sum_{t_0<\arrivalTimes[e]_j\leq t_1} \log \left(\intensity[e](\arrivalTimes[e]_j)\right), 
 \\
 \lminus(t_0,t_1) =& (t_1-t_0)\sum_{e=1}^{\numEventTypes} \baseRate_e
 \\
 & +\sum_{e=1}^{\numEventTypes} 
 \sum_{\eone,\xone} \sum_{\arrivalTimes[\eone,\xone]_j < t_1}
 K\big(\max(0,t_0-\arrivalTimes[\eone,\xone]_j), t_1 - \arrivalTimes[\eone,\xone]_j,  \impCoef\subscriptexe, \decCoef\subscriptexe\big),
\end{split}
 \end{equation*}
where $\eone$, $\xone$ and $\arrivalTimes[\eone,\xone]_j$ have the same meaning as in equation \eqref{eq.evaluation_of_intensity}, $(\arrivalTimes[e]_j)_j$ are the jump times of the $e$-th component $\countingProc$ of $\multiCountingProc$, $(\arrivalTimes[]_n)_n$ are the jump times of the ground process $\groundProc$ of  $\multiCountingProc$, and where 
\begin{equation*}
 K\big(s,t, a, b) = \int_{s}^{t} \parametricHawkesKernel(u,a,b)du.
\end{equation*}
\end{prop}
\begin{proof}
 The Stiejlties integral in equation \eqref{eq.intensity_of_sdHawkes} is equivalently expressed as $\sum \hawkesKernel\subscriptee (t-\arrivalTimes[\eone]_j, \, X(\arrivalTimes[\eone]_j) )$, where the sum is over the jump times $\arrivalTimes[\eone]_j$ of the $\eone$-th component $\countingProc[\eone]$ of $\multiCountingProc$. This sum is finite because the counting process is non-explosive. Since $\arrivalTimes[\eone]_j$ is the $j$-order statistic of $\lbrace \arrivalTimes[\eone,\xone]_k: \, \xone=1,\dots,\numStates, \,\,\, k=1,2,\dots\rbrace$, the summands can be rearranged to give 
 \begin{equation*}
  \sum_{\arrivalTimes[\eone]_j < t} \hawkesKernel\subscriptee (t-\arrivalTimes[\eone]_j, \, X(\arrivalTimes[\eone]_j))
  = \sum_{\xone=1}^{\numStates} \sum_{\arrivalTimes[\eone, \xone]_j < t} \hawkesKernel\subscriptee (t-\arrivalTimes[\eone, \xone]_j, \, X(\arrivalTimes[\eone,\xone]_j))
  = \sum_{\xone=1}^{\numStates} \sum_{\arrivalTimes[\eone, \xone]_j < t} \hawkesKernel\subscriptee (t-\arrivalTimes[\eone, \xone]_j, \, \xone),
 \end{equation*}
where in the second identity we have used that by definition $X(\arrivalTimes[\eone,\xone]_j) = \xone$ for any $\eone$ and any $j$. Hence by plugging in the parametric form of the Hawkes kernel, we obtain equation \eqref{eq.evaluation_of_intensity}. 

As for formula \eqref{eq.evaluation_of_loglikelihood}, starting from equation \eqref{eq.loglikelihood_general}, we only need to show that $\lminus(0,\timeHorizon) = \sum_{e=1}^{\numEventTypes} \int_{0}^{\timeHorizon}\intensity^{\nu,\impCoef, \decCoef}(t)dt$. To this purpose, we use equation \eqref{eq.evaluation_of_intensity} to write for arbitrary times $t_0<t_1$
\begin{equation*}
\begin{split}
\sum_{e=1}^{\numEventTypes}\int_{t_0}^{t_1} \left(\intensity[e](t) - \baseRate_e \right) dt 
=&
\sum_{e=1}^{\numEventTypes}\sum_{\eone,\xone}\int_{t_0}^{t_1} 
\sum_{\arrivalTimes[\eone,\xone]_j <t} \parametricHawkesKernel(t-\arrivalTimes[\eone,\xone]_j, \impCoef\subscriptexe, \decCoef\subscriptexe)dt 
\\
=&
\sum_{e=1}^{\numEventTypes}\sum_{\eone,\xone}
\Bigg\lbrace
\sum_{\arrivalTimes[\eone,\xone]_j \leq t_0}
\int_{t_0}^{t_1} 
\parametricHawkesKernel(t-\arrivalTimes[\eone,\xone]_j, \impCoef\subscriptexe, \decCoef\subscriptexe)dt 
\\
&+
\sum_{t_0<\arrivalTimes[\eone,\xone]_j \leq t_1}
\int_{\arrivalTimes[\eone,\xone]_j}^{t_1} \parametricHawkesKernel(t-\arrivalTimes[\eone,\xone]_j, \impCoef\subscriptexe, \decCoef\subscriptexe)dt 
\Bigg\rbrace.
\end{split}
\end{equation*}
To get to the second identity, we split the innermost sum over $\arrivalTimes[\eone,\xone]_j<t$ into the sum over $\arrivalTimes[\eone,\xone]_j \leq t_0$ and the sum over $t_0 < \arrivalTimes[\eone,\xone]_j<t$;  we swapped summation over  $\arrivalTimes[\eone,\xone]_j \leq t_0$ with integration in $dt$, whereas we applied Fubini theorem to go from $\int_{t_0}^{t_1}dt\sum_{t_0 < \arrivalTimes[\eone,\xone]_j<t}$ to $\sum_{t_0 < \arrivalTimes[\eone,\xone]_j<t_1}\int_{\arrivalTimes[\eone,\xone]_j}^{t_1}dt$. Then, by using the time primitive $K$ of $\parametricHawkesKernel$, we obtain the claimed expression for $\lminus(t_0,t_1)$.
\end{proof}

\begin{corol}[Corollary to Proposition \ref{prop.parametric_hawkes_kernels}]
\label{corol.complexities}
The computational complexity of evaluating the intensity $\intensity(t)$ in equation \eqref{eq.evaluation_of_intensity} is $O(\groundProc(t))$, and -- assuming that the primitive $K$ of $\parametricHawkesKernel$ is known in closed form (i.e. no numerical integration required) -- the computational complexity of evaluating the log-likelihood of a realisation of $(\multiCountingProc, \stateVariable)$ with time horizon $\timeHorizon$ is $O(\groundProc\squared(\timeHorizon))$. 

Furthermore, if the intensities are evaluated according to equation \eqref{eq.evaluation_of_intensity}, the computational complexity of simulating a state-dependent Hawkes process up to time $\timeHorizon$ using the iterative step in Algorithm \ref{algo.MP18_ogata} is $O(\numEventTypes\groundProc\squared(\timeHorizon))$.
\end{corol}
\begin{proof}
The number of summands in equation \eqref{eq.evaluation_of_intensity} is equal to the number of arrival times before $t$, namely $\groundProc(t-)$, hence the computational complexity of $\intensity(t)$ is $O(\groundProc(t))$. Similarly, the complexities of $\Phi(0,T)$ and of $\lminus(0,T)$ are $O(\groundProc(T))$. Instead, since every summand in the expression for $\lplus$ requires the evaluation of the intensity $\intensity$, the complexity of $\lplus(0,T)$ is $O(\groundProc\squared(T))$. To see this, write 
\begin{equation*}
 \lplus(0,T) = \sum_{n=1}^{\groundProc(T)}
 \log \left(
 \baseRate_{\event} + \sum_{m < n}\parametricHawkesKernel\left(\arrivalTimes[]_n - \arrivalTimes[]_m, \impCoef_{\event[m],\stateVariable_m,\event[n]}, \decCoef_{\event[m],\stateVariable_m,\event[n]}\right)
 \right),
\end{equation*}
and hence deduce that evaluating $\lplus(0,T)$ requires $\groundProc(T)(1+\groundProc(T))/2$ additions. 
Therefore, the overall computation complexity of the log-likelihood is $O(\groundProc\squared(T))$.

Finally, the iterative step described in Algorithm \ref{algo.MP18_ogata} requires $1+n_r$ evaluations of the overall intensity $\bar{\intensity[]}(t) = \intensity[1](t)+\dots+\intensity[\numEventTypes](t)$, where $n_r$ is the number of rejections. Therefore, updating from $\lbrace(\arrivalTimes[]_i,\event[i],\stateVariable_i): \, i=1,\dots,n-1\rbrace$ to $\lbrace(\arrivalTimes[]_i,\event[i],\stateVariable_i): \, i=1,\dots,n\rbrace$ costs $(1+n_r)\numEventTypes\groundProc(\arrivalTimes[]_{n-1})$ operations. Compounding all these costs, the overall complexity of the simulation up to time $\timeHorizon$ is $O(\numEventTypes\groundProc\squared(\timeHorizon))$. 
\end{proof}



A remarkable choice for a parametric family $\lbrace \hawkesKernel^{\impCoef,\decCoef}: \, \impCoef, \, \decCoef\rbrace$ of the form synthesised in equation \eqref{eq.parametric_hawkes_kernel_general} is 
\begin{equation}\label{eq.exponentialHawkesKernel}
 \hawkesKernel^{\alpha,\beta}\subscriptee(t,\xone) = \impCoef_{\eone,\xone,e} \, \exp\left( -\decCoef_{\eone,\xone,e} \, t \right),
\end{equation}
where the fixed function $\parametricHawkesKernel$ is $\parametricHawkesKernel(t,a,b)=a\exp(-bt)$.
This choice is remarkable because (i) for any non-negative $\decCoef\subscriptexe$ the kernels are bounded, hence satisfying the assumption in Proposition \ref{prop.existence_sdHawkes}; (ii) it offers computational advantages for the evaluation of the log-likelihood in equation \eqref{eq.evaluation_of_loglikelihood}. 
Point (i) guarantees the existence, uniqueness and non-explosiveness of a state-dependent Hawkes process for any choice of non-negative parameters $\baseRate_e$, 
$\impCoef_{\eone,\xone, e}$ and $\decCoef_{\eone,\xone, e}$, and any transition matrices $\transProb_e$. Point (ii) is one of the reasons that motivated  exponential kernels in the two models implemented in \citealp{MP18sta}. We will now expand on this computation advantages. 

Both in the context of simulating a state-dependent Hawkes process and in the context of maximum likelihood estimation, several evaluations of the intensities $\intensity$ are required. 
In the presence of several events -- each happening at the arrival times $\arrivalTimes[\eone,\xone]_j$ --, each of these evaluations involves the long iterations spelled out in Proposition \ref{prop.parametric_hawkes_kernels}, and this may be compounded with a large number of states and/or of event types (curse of dimensionality). However, in the case of exponential kernels as in equation \eqref{eq.exponentialHawkesKernel}, knowledge of the intensities at some earlier time $s$ can be reused for a quicker evaluation of $\intensity(t)$ at a later time $t$. Indeed, for $e,\eone=1,\dots,\numEventTypes$ and $\xone=1,\dots,\numStates$, let $\tilde{\intensity[]}_{\eone,\xone,e}(s):= \sum_{\arrivalTimes[\eone,\xone]_j<s}\impCoef_{\eone,\xone,e}\exp(-\decCoef_{\eone,\xone,e}(s-\arrivalTimes[\eone,\xone]_j))$, so that $\intensity[e](s)=\baseRate_e+\sum_{\eone,\xone}\tilde{\intensity[]}_{\eone,\xone,e}(s)$. Then, for $s<t$ it holds
\begin{equation}\label{eq.update_of_exponential_intensity}
 \intensity[e](t) = \baseRate_e + \sum_{\eone,\xone}\tilde{\intensity[]}_{\eone,\xone,e}(s)\exp\left(-\decCoef_{\eone,\xone,e}(t-s)\right)
 + \sum_{\eone,\xone}\sum_{s\leq\arrivalTimes[\eone,\xone]_j<t} \impCoef\subscriptexe \exp\left(-\impCoef\subscriptexe(t-\arrivalTimes[\eone,\xone]_j)\right).
\end{equation}
In the latter formula, the first summation is only over the indexes $\eone$ and $\xone$, and represents the reusage of the past evaluations at time $s$, updated by the factors $\exp(-\decCoef (t-s))$. The second summation instead represents the new contribution to the intensity $\intensity[e](t)$ at time $t$ stemming from events that happened in between $s$ and $t$. 

By using the updating rule in equation \eqref{eq.update_of_exponential_intensity}, the computational complexity of evaluating $\lplus(0,T)$ of equation \eqref{eq.evaluation_of_loglikelihood} in case of exponential kernels is brought down to $O(\groundProc(T))$, hence making the log-likelihood's complexity grow only linearly in the dimension of the sample. 
Similarly, since in case of exponential kernels the $(n+1)$-th iteration of Algorithm \ref{algo.MP18_ogata} can compute the required intensity $\bar{\intensity[]}$ relying on the evaluation of the same function at the $n$-th iteration, the overall complexity of simulating a state-dependent Hawkes process with exponential kernels up to time $\timeHorizon$ is reduced to $O(\numEventTypes\groundProc(\timeHorizon))$.

\subsection{Power-law kernels}\label{sec.powerlaw_kernels}
As outlined in the introduction, the motivation for our work stems from  \citealp{BM14haw}. There, the authors implemented a non-parametric estimation of ordinary Hawkes processes to calibrate their Hawkes-modelled market replayer. After non-parametric procedure, the resulting shapes of Hawkes kernels were shown to be fit by power-law kernels, exhibiting the decay $\hawkesKernel\subscriptee(t) \asymp t^{-\decCoef\subscriptee}$ for some exponents $\decCoef\subscriptee$ typically found in a right neighborhood of $1$.

Building on Bacry and Muzy's findings, we adopt the following parametric form for our state-dependent Hawkes kernels:
\begin{equation}\label{eq.powerlaw_kernels}
 \hawkesKernel\subscriptee(t,\xone) 
 = \impCoef\subscriptexe \big( t+ 1 \big)^{-\decCoef\subscriptexe},
\end{equation}
for some non-negative coefficients $\impCoef\subscriptexe \geq 0$ and $\decCoef\subscriptexe > 1$. 
This means that with respect to the general expression in \eqref{eq.parametric_hawkes_kernel_general}, we fix the non-negative function $\parametricHawkesKernel$ to be $\parametricHawkesKernel(t,a,b) = a(t+1)^{-b}$.  
Beside the alignment with the mentioned asymptotic decay, this choice guarantees, on the one hand, the boundedness of the kernels (which implies existence, uniqueness and non-explosiveness of the state-dependent Hawkes process); on the other hand, it guarantees the interpretation of $\impCoef\subscriptexe$ as the instantaneous impact  that an event of type $\eone$ followed by a state $\xone$ has on the intensity $\intensity[e]$. Both these two properties mirror what we had for the exponential case in equation \eqref{eq.exponentialHawkesKernel}, adopted in the models implemented by \citealp{MP18sta}.

From Proposition \ref{prop.parametric_hawkes_kernels} and from the subsequent discussion on exponential kernels it follows that adopting the power-law kernels in equation \eqref{eq.powerlaw_kernels} poses a computational challenge for the implementation of our model, in particular with regards to the maximum likelihood estimation. In the remaining of this section we present the formulae involved in our algorithms and later in Section \ref{sec.the_python_module} we will take a close look at their implementations, showing how the computational challenge was tackled. 

\begin{prop}\label{prop.intensity_and_compensator_powerlaw}
 Let $(\multiCountingProc,\stateVariable)$ be a $\numEventTypes$-dimensional state-dependent Hawkes process on a finite state space of cardinality $\numStates$. Let $\baseRate_1, \dots, \baseRate_{\numEventTypes}$ be the non-negative base rates, and let $\transProb_1, \dots,\transProb_{\numEventTypes}$ be the transition matrices on the state space associated with the event types $e=1, \dots, \numEventTypes$. Assume that the Hawkes kernels $\hawkesKernel\subscriptee$ are given by equation \eqref{eq.powerlaw_kernels} for some non-negative coefficients $\impCoef\subscriptexe \geq 0$ and $\decCoef\subscriptexe > 1$.  For $\eone = 1,\dots,\numEventTypes$, and $\xone=1,\dots,\numStates$, let $\arrivalTimes[\eone,\xone]_j$ be the jump times of $(\eone,\xone)$-th component $\hybridHawkes_{\eone,\xone}$ of the hybrid-MPP counterpart of $(\multiCountingProc,\stateVariable)$, and for $e = 1,\dots,\numEventTypes$ and $s<t\leq u$ define
 \begin{equation}\label{eq.definitions_of_ESSE}
  \begin{split}
   \ESSE\subscriptexe(s,t):=& \sum_{s\leq \arrivalTimes[\eone,\xone]_j < t}\big(t-\arrivalTimes[\eone,\xone]_j +1\big)^{-\decCoef\subscriptexe},
   \\
   \ESSEone\subscriptexe(s,t):=& \sum_{s\leq \arrivalTimes[\eone,\xone]_j < t}\big(t-\arrivalTimes[\eone,\xone]_j +1\big)^{-\decCoef\subscriptexe}\log(t-\arrivalTimes[\eone,\xone]_j +1),
   \\
   \ESSEtwo\subscriptexe(s,t):=& \sum_{s\leq \arrivalTimes[\eone,\xone]_j < t}\big(t-\arrivalTimes[\eone,\xone]_j +1\big)^{1-\decCoef\subscriptexe}\log(t-\arrivalTimes[\eone,\xone]_j +1),
   \\
   \ESSEthree\subscriptexe(s,t,u):=& \sum_{s\leq \arrivalTimes[\eone,\xone]_j < t}\big(u-\arrivalTimes[\eone,\xone]_j +1\big)^{1-\decCoef\subscriptexe}.
  \end{split}
 \end{equation}
 Then, the $e$-th component of the intensity vector $(\intensity[1],\dots,\intensity[\numEventTypes])$ has the expression
 \begin{equation}\label{eq.intensity_powerlaw}
  \intensity[e](t) = \baseRate_e +\sum_{\eone,\xone}\impCoef\subscriptexe\ESSE\subscriptexe(0,t),
 \end{equation}
and the increment from time $t_0$ to time $t_1$ of the compensator of the $e$-th component $\countingProc$ the counting process $\multiCountingProc$ has the expression
\begin{equation}\label{eq.eval_compensator_powerlaw}
\begin{split}
 \compensator_e(t_0,t_1)=&\int_{t_0}^{t_1} \intensity[e](s)ds
 \\
 =&
 (t_1 - t_0)\baseRate_e
 + \sum_{\eone,\xone} \frac{\impCoef\subscriptexe}{\decCoef\subscriptexe - 1}
 \Bigg(
 \ESSEthree\subscriptexe(0,t_0,t_0) 
 \\
 & - \ESSEthree\subscriptexe(0,t_0,t_1)
 +\# \left\lbrace t_0 \leq \arrivalTimes[\eone,\xone]_j <t_1\right\rbrace
 - \ESSEthree\subscriptexe(t_0,t_1,t_1)
 \Bigg),
\end{split} 
\end{equation}
where $\# \lbrace t_0 \leq \arrivalTimes[\eone,\xone]_j <t_1\rbrace = \hybridHawkes_{\eone,\xone}(t_1 -)-\hybridHawkes_{\eone,\xone}(t_0-)$ is the number of arrival times $\arrivalTimes[\eone,\xone]_j$ in the half-open interval $[t_0,t_1)$. 
\end{prop}
\begin{proof}
 Straightforward computations. 
\end{proof}


\begin{prop}\label{prop.derivatives_of_loglikelihood}
 Assume the setting of Proposition \ref{prop.intensity_and_compensator_powerlaw}. Recall equation \eqref{eq.loglikelihood_general} for the log-likelihood and let $\lplus$ and $\lminus$ be as in Proposition \ref{prop.parametric_hawkes_kernels}, with the parametric form of the Hawkes kernels as in equation \eqref{eq.powerlaw_kernels}. Let $\timeHorizon$ be the time horizon of the realisation of $(\multiCountingProc, \stateVariable)$. 
 Then, the gradient of the log-likelihood is computed based on the following expressions for the derivatives of $\lplus$ and $\lminus$ with respect to the model's parameters.
 \begin{itemize}
  \item Derivatives of $\lplus$:
  \begin{equation}\label{eq.derivatives_of_lplus}
  \begin{split}
   \frac{\partial\lplus(0,\timeHorizon)}{\partial\baseRate_e}
   =&
   1 \cdot \intensity[e]\inverse(\arrivalTimes[e])
   = \sum_{\arrivalTimes[e]_j < \timeHorizon}\frac{1}{\intensity[e](\arrivalTimes[e]_j)},
   \\
   \frac{\partial\lplus(0,\timeHorizon)}{\partial\impCoef\subscriptexe}
   =&
   \ESSE\subscriptexe(0,\arrivalTimes[e])\cdot \intensity[e]\inverse(\arrivalTimes[e]),
   \\
   \frac{\partial\lplus(0,\timeHorizon)}{\partial\decCoef\subscriptexe}
   =&
   -\impCoef\subscriptexe
   \ESSEone\subscriptexe(0,\arrivalTimes[e])\cdot \intensity[e]\inverse(\arrivalTimes[e]),
  \end{split}
 \end{equation}
 where the symbol $\cdot$ denotes scalar product between two one-dimensional arrays; where $\arrivalTimes[e] = (\arrivalTimes[e]_1, \arrivalTimes[e]_2, \dots )$ is the ordered array of jump times of the $e$-th component $\countingProc$ of $\multiCountingProc$;  where $\intensity[e](\arrivalTimes[e])$ and $\intensity[e]\inverse(\arrivalTimes[e])$ are the arrays whose $j$-th components are respectively $\intensity[e](\arrivalTimes[e]_j)$ and $1/\intensity[e](\arrivalTimes[e]_j)$; and where $\ESSE\subscriptexe(0,\cdot)$ and $\ESSEone\subscriptexe(0,\cdot)$ from equation \eqref{eq.definitions_of_ESSE} are applied component-wise to the arrays $\arrivalTimes[e]$.
 \item Derivatives of $\lminus$:
 \begin{equation}\label{eq.derivatives_of_lminus}
  \begin{split}
   \frac{\partial\lminus(0,\timeHorizon)}{\partial\baseRate_e}
   =&
   \timeHorizon,
   \\
   \frac{\partial\lminus(0,\timeHorizon)}{\partial\impCoef\subscriptexe}
   =&
   \frac{1}{\decCoef\subscriptexe - 1}
   \Bigg(
   \ESSEthree\subscriptexe(0,0,0)  - \ESSEthree\subscriptexe(0,0,\timeHorizon)
    +\hybridHawkes_{\eone,\xone}(\timeHorizon-)
   - \ESSEthree\subscriptexe(0,\timeHorizon,\timeHorizon)
   \Bigg),
   \\
   \frac{\partial\lminus(0,\timeHorizon)}{\partial\decCoef\subscriptexe}
   =&
   \frac{\impCoef\subscriptexe}{1-\decCoef\subscriptexe}\
   \left(
   \frac{\partial\lminus(0,\timeHorizon)}{\partial\impCoef\subscriptexe}
   +\ESSEtwo\subscriptexe(0,\timeHorizon)
   \right),
  \end{split}
 \end{equation}
 where $\ESSEtwo\subscriptexe$ and $\ESSEthree\subscriptexe$ were defined in equation \eqref{eq.definitions_of_ESSE}.
 \end{itemize}
\end{prop}
\begin{proof}
 Straightforward computations. 
\end{proof}

\section{A state-dependent Hawkes model for order-driven markets}

\subsection{Order-driven markets}\label{sec.order-driven_markets}
Order-driven markets are trading venues organised around \emph{limit orders}. A limit order is the fundamental action that market participants in order-driven markets can perform. It is represented by a 4-tuple $(t,q,p,d)$, where $t$ denotes time, $q$ denotes size, $p$ denotes price, and $d$ denotes direction. We say that a market participant posts a limit order $(t,q,p,d)$ if at time $t$ she submits to the exchange her commitment to buy ($d=1$) or sell ($d=-1$) the amount $q$ at the price $p$. The price $p$ is interpreted as the highest price at which she is committed to buy if $d=1$, or the lowest price at which she is committed to sell if $d=-1$. A matching algorithm run by the trading platform looks for possible counterparts to her trade: if other participants' commitments to trade exist to (partially) fulfill her order (at a price not worse than $p$ from her point of view), then her order is (partially) cleared. The fraction of her order that is cleared disappears from the market, and it is said to be \emph{executed};  the fraction that could not be fulfilled is recorded in the \emph{limit order book}, waiting for counterparts to trade with.

A limit order book is a grid of equally spaced prices at which non-executed limit orders sit. The space between consecutive grid nodes is called the tick size of the LOB, which we denote by $\tickSizeOfLOB$. Prices are increasing from left to right. At every node of the grid, outstanding limit orders to buy or sell at the corresponding price are collected; these limit orders are also said to be queuing there. Spontaneously -- i.e. by market forces  -- such orders organise in a way that buy offers will be displaced on the left (the so-called \emph{bid side}) and sell offers will be on the right (the so-called \emph{ask side}). Indeed, if there were a buy (respectively, sell) limit order to the right (left) of some sell (buy) limit orders, or at the same node, then the matching algorithm would have matched them, clearing them out from the orderbook.  Therefore, the whole configuration of a limit order book at time $t$ is given when the following variables are specified:
\begin{enumerate}
\item the best ask price $\bestAskPrice_{t}$, i.e. the lowest price at which one can find sell offers;
\item the best bid price $\bestBidPrice_t$, i.e. the highest price at which one can find buy offers;
\item the volume $\nthBestAskVolume[i]_t$ of ask offers at price $\nthBestAskPrice[i]_t= \bestAskPrice_t + (i-1)\tickSizeOfLOB$, for $i=1,2,\dots$, i.e. the quantity 
\begin{equation*}
 \nthBestAskVolume[i]_t = \sum_{\lbrace(s,q,\nthBestAskPrice[i]_t,-1): \quad s\leq t\rbrace} q,
\end{equation*}
where the sum is over all the outstanding (i.e. non-executed) sell limit orders submitted by time $t$ for  a price of $\bestAskPrice_t + (i-1)\tickSizeOfLOB$;
\item the volume $\nthBestBidVolume[i]_t$ of bid offers at price $\nthBestBidPrice[i]_t=\bestBidPrice_t - (i-1)\tickSizeOfLOB$, for $i=1,2,\dots$, i.e. the quantity 
\begin{equation*}
 \nthBestBidVolume[i]_t = \sum_{\lbrace (s,q,\nthBestBidPrice[i]_t,1): \quad s\leq t\rbrace} q,
\end{equation*}
where the sum is over all the outstanding (i.e. non-executed) buy limit orders submitted by time $t$ for  a price of $\bestBidPrice_t - (i-1)\tickSizeOfLOB$.
\end{enumerate} 

% The configuration of the LOB at time $t$ is thus depicted in four steps: 1. locate the best bid price $\bestBidPrice_t$ on the price grid;  2. locate the best ask price $\bestAskPrice_t$ on the price grid; 3. on every bid price level $\nthBestBidPrice[i]_t$, draw a column as high as the volume $\nthBestBidVolume[i]_t$  of bid offers at that price; 4. on every ask price level $\nthBestAskPrice[i]_t$, draw a column as high as the volume $\nthBestAskVolume[i]_t$  of ask offers at that price.

The whole configuration of a limit order book at time $t$ is described by the variables $(\bestAskPrice_t, \bestBidPrice_t, \lbrace (\nthBestAskVolume[i]_t, \nthBestBidVolume[i]_t) : \, \, i=1,2,\dots\rbrace)$. Apart from these variables, other derived quantities are useful to assess the properties of an orderbook. These are the spread, the mid-price and the volume imbalance (or queue imbalance). 

The spread at time $t$ is the distance $\lvert \bestAskPrice_t - \bestBidPrice_t \rvert$ between best ask price  and best bid price, which we denote by $\LOBspread_t$.  The mid-price at time $t$ is the mid point in between  $\bestBidPrice_t$ and $\bestAskPrice_{t}$, which we denote by $\price^{m}_t$ and is given by $\price^m_t = (\bestAskPrice_t + \bestBidPrice_{t})/{2}$.
Finally, the $n$-levels volume imbalance (or queue imbalance) at time $t$, denoted $\volumeImbalance^{n}_t$, is the normalised excess of limit orders on the first $n$ levels of the bid side compared to the limit orders on the first $n$ levels of the ask side, namely
\begin{equation}\label{eq.definition_volume_imbalance}
\volumeImbalance^{n}_t =
\frac{\sum_{i\leq n}\nthBestBidVolume[i]_t - \sum_{i\leq n}\nthBestAskVolume[i]_t}{\sum_{i\leq n}\nthBestBidVolume[i]_t + \sum_{i\leq n}\nthBestAskVolume[i]_t},
\end{equation}
where $\sum_{i\leq n}\nthBestBidVolume[i]_t$ (respectively, $\sum_{i\leq n}\nthBestAskVolume[i]_t$) is the cumulative volumes on the first $n$ bid (ask) levels. 
At least for the case $n=1$, the queue imbalance $\volumeImbalance^n$ is widely accepted as a reliable signal for the next mid-price move (\citealp{CDJ18enh}): when it is close to -1 the mid-price will likely decrease, and when it is close to +1 it will likely increase. 

A variant of the volume imbalance that takes into account the dynamic excess of bid offers compared to ask offers arrived to the orderbook in a time window of length $\Delta T$ is the order flow imbalance. It was introduced in \citealp{CKS14pri}, and its definition for the $i$-th level is
\begin{equation*}
\orderFlowImbalance_{\Delta T, \, i}(t)
=
\sum_{t-\Delta T \leq s < t} 
\Big(
\nthBestBidVolume[i]_t \one (\delta\bestBidPrice_t \geq 0)
- \nthBestBidVolume[i]_{t-} \one (\delta\bestBidPrice_t \leq 0)
- \nthBestAskVolume[i]_t \one (\delta\bestAskPrice_t \leq 0)
+ \nthBestAskVolume[i]_{t-} \one (\delta\bestAskPrice_t \geq 0)
\Big),
\end{equation*}
where the sum is take over all event times in the half open interval $[t-\Delta T, t)$, and where $\delta\bestBidPrice_t : = \lim_{\epsilon\downarrow 0} (\bestBidPrice_t - \bestBidPrice_{t-\epsilon} )$, and analogously for $\delta\bestAskPrice_t$. 

We will incorporate proxies for the queue imbalance and the order flow imbalance in our state-dependent Hawkes model, see Section \ref{sec.sdHawkes_model}.

When a limit order comes to the market, it modifies the previous configuration of the orderbook. Proposition \ref{prop.lob_update} gives the updating rule for the variables $(\bestAskPrice_t, \bestBidPrice_t, \lbrace (\nthBestAskVolume[i]_t, \nthBestBidVolume[i]_t) : \, \, i=1,2,\dots\rbrace)$.
\begin{prop}\label{prop.lob_update}
 Let $(t,q,p,-1)$ be a sell limit order that arrives to the market at time $t$. Let $(\bestAskPrice_{t-}, \bestBidPrice_{t-},\lbrace  (\nthBestAskVolume[i]_{t-}, \nthBestBidVolume[i]_{t-}) : \, \, i=1,2,\dots\rbrace )$ represent the limit order book immediately before $t$. Then, prices and volumes are updated according to the following formulae:
 \begin{equation}\label{eq.lob_update_sell_order}
  \begin{split}
   \bestBidPrice_t =& \bestBidPrice_{t-} - (N-1)\tickSizeOfLOB;
   \\
   \nthBestBidVolume[k]_{t} 
   =& \nthBestBidVolume[k+N-1]_{t} 
   - \left( q^{k+N-2}-q^{k+N-1}\right), \qquad k=1,2,\dots ; 
   \\
   \bestAskPrice_t = & \bestAskPrice_{t-} 
   - \left(\bestAskPrice_{t-} - p\right) \one\left\lbrace q^{\infty} > 0\right\rbrace ; 
   \\
   \nthBestAskVolume[k]_t = &  \nthBestAskVolume[k-\delta\bestAskPrice_t]_{t-}
   +q^{\infty}\one\left\lbrace p=\bestAskPrice_t + (k-1)\tickSizeOfLOB \right\rbrace, \qquad k=1,2,\dots ;
  \end{split}
 \end{equation}
 where $\delta\bestAskPrice_t = \bestAskPrice_t - \bestAskPrice_{t-}$ denotes the (non-positive) jump of the best ask price;  $N := \inf\lbrace n\geq 0 : \, q<\sum_{i=1}^{n} \nthBestBidVolume[i]_{t-} \quad \text{ or } \quad  \nthBestBidPrice[n]_{t-}\leq p \rbrace$ denotes the number of levels on the bid side involved in servicing the incoming order; and 
 \begin{equation*}
  \begin{split}
   q^{i}:=& \max\left( 0, 
   q-\sum_{k=1}^{i\wedge N} \nthBestBidVolume[k]_{t-}
   \right)
   \\
   =& \max\left( 0, 
   q-\sum_{k\leq i} \nthBestBidVolume[k]_{t-}\one\left\lbrace \nthBestBidPrice[k]_{t-} \geq p\right\rbrace
   \right)
  \end{split}
 \end{equation*}
 is the outstanding quantity to be executed after the first $i$ levels have been consumed. 
 For a buy limit order  $(t,q,p,+1)$, the updating rules are similar, with opposite side and direction. 
\end{prop}

The arrival of a limit order to the market triggers two events: one is the consumption of the liquidity capable to (partially) service the limit order, the other is the addition of the non-executed part of the limit order to the appropriate queue in the orderbook. Proposition \ref{prop.decomposition_of_limit_order} states the terms of this decomposition.

\begin{prop}\label{prop.decomposition_of_limit_order}
 Given the configuration $(\bestAskPrice_{t-}, \bestBidPrice_{t-},\lbrace  (\nthBestAskVolume[i]_{t-}, \nthBestBidVolume[i]_{t-}) : \, \, i=1,2,\dots\rbrace )$ of the limit order book immediately before time $t$, processing the sell limit order $(t,q,p,-1)$ is equivalent to processing the pair of orders $[(t,q_M,0,-1),(t,q-q_M,p,-1) ]$, where 
 \begin{equation*}
  q_M:= \min\left(q, \sum_{i\geq 1} \nthBestBidVolume[i]_{t-} \one \left\lbrace \nthBestBidPrice[i]_{t-}\geq p\right\rbrace\right).
 \end{equation*}
Similarly, processing the buy limit order $(t,q,p,+1)$ is equivalent to processing the pair of orders $[(t,q_M,\infty,+1),(t,q-q_M,p,+1) ]$, where 
 \begin{equation*}
  q_M:= \min\left (q, \sum_{i\geq 1} \nthBestAskVolume[i]_{t-} \one \left\lbrace \nthBestAskPrice[i]_{t-}\leq  p\right\rbrace\right).
 \end{equation*}
\end{prop}

The first component $(t,q_M,0,-1)$ in the decomposition $[(t,q_M,0,-1),(t,q-q_M,p,-1) ]$ of the sell limit order $(t,q,p,-1)$ is called \emph{sell market order}. Notice that in the 4-tuple $(t,q_M,0,-1)$, the price $p$ is set to zero, and this guarantees immediate execution: the sale of the amount $q_M$ is instantaneously matched with outstanding buy limit orders on the bid side, and no fraction of $q_M$ is put into the queue. On the contrary, the second component $(t,q-q_M,p,-1)$ specifies exactly that part of $(t,q,p,-1)$ which will be queued. 
Similarly, $(t,q_M,\infty,+1)$ represents the market order component of the buy limit order $(t,q,p,+1)$, and it is referred to as \emph{buy market order}. The price $p=\infty$ is a way to express the fact that a counterpart for the purchase of the amount $q_M$ will instantaneously be found on the ask side of the orderbook. In the following, when talking about market orders (either buy or sell), we will be referring to the first components  in the decomposition of all limit orders with non-zero market order size $q_M >0$.

Given a time window $\timeWindow$, the evolution in time of the limit order book $\lbrace (\bestAskPrice_t, \bestBidPrice_t,\lbrace (\nthBestAskVolume[i]_t, \nthBestBidVolume[i]_t) : \, \, i=1,2,\dots\rbrace):  \quad 0\leq t \leq \timeHorizon \rbrace$ results from the history of all limit order submissions,  cancellations and executions that happened in $\timeWindow$. If every limit order in   $\lbrace (t,q,p,d): \, 0\leq t\leq \timeHorizon\rbrace$ is decomposed as per in Proposition \ref{prop.decomposition_of_limit_order}, then the seller-initiated trades that happened in $\timeWindow$ are $\lbrace (t,q_M,0,-1):\, \,  q_M>0, \, 0\leq t\leq \timeHorizon \rbrace$, and the buyer-initiated trades that happened in $\timeWindow$ are $\lbrace (t,q_M,\infty,+1):\, \,  q_M>0, \, 0\leq t\leq \timeHorizon \rbrace$.  Hence, in the following we will identify trades with market orders of non-zero size $q_M>0$. Notice that the actual trigger of a trade might be a limit order $(t,q,p,d)$ with non-trivial price specification $0<p<\infty$ rather than a market order, but -- because of the asserted equivalence in Proposition \ref{prop.decomposition_of_limit_order} -- referring to these triggers as ``market orders'' does not alter the evolution of the limit order book.  





\subsection{The model of Bacry and Muzy (2014)}\label{sec.BM14_model}
Two aspects of the time evolution of the order-driven market are modelled: trades and changes in the mid-price. As explained in the previous section (Section \ref{sec.order-driven_markets}), trades can in fact be identified with submissions of market orders, and we will do this here.

Four streams of arrival times are considered: the stream $(\arrivalTimes[1]_j)_j$ of times when limit orders are executed on the bid side (equivalently identified with the arrival times of sell market orders); the stream  $(\arrivalTimes[2]_j)_j$ of times when limit orders are executed on the ask side (equivalently identified with arrival times of buy market orders); the stream  $(\arrivalTimes[3]_j)_j$ of times when the mid-price decreases; the stream  $(\arrivalTimes[4]_j)_j$ of times when the mid-price increases. Correspondingly, these sequences of random times give rise to a four-dimensional counting process $\multiCountingProc=(\countingProc[1],\dots, \countingProc[4])$ with the following interpretation of its components:
\begin{itemize}
 \item $\countingProc[1](t)$ denotes the number of seller-initiated trades that happened before or at time $t$ (identified with the number of market orders arrived on the bid side of the orderbook by time $t$);
 \item $\countingProc[2](t)$ denotes the number of buyer-initiated trades that happened before or at time $t$ (identified with the number of market orders arrived on the ask side of the orderbook by time $t$);
 \item $\countingProc[3](t)$ denotes the number of downward jumps of the mid-price  that happened before or at time $t$;
 \item $\countingProc[4](t)$ denotes the number of upward jumps of the mid-price  that happened before or at time $t$.
\end{itemize}

Notice that $\countingProc[4]-\countingProc[3]$ is a proxy for the mid-price. Indeed, given the tick size $\tickSizeOfLOB$ of the limit order book and the initial position $\midprice_0$ of the mid-price,  the paths $t\mapsto \midprice_0 +(\countingProc[4](t)-\countingProc[3](t))\tickSizeOfLOB$ and $t\mapsto \midprice_t$ will have discrepancies only when $\vert\delta\midprice(t)\vert \geq 2\tickSizeOfLOB$, and this rarely happens (in particular for limit order books with large tick size).

The counting process $\multiCountingProc$ is assumed to be an ordinary Hawkes process, hence admitting a continuous compensator with an intensity $\intensity[]=(\intensity[1],\dots,\intensity[4])$ of the type in equation \eqref{eq.intensity_ordHawkes}. Hence, we let $\baseRate$ be the $4$-dimensional vector of base rates, and we let $\hawkesKernel$ be the $4\times4$-dimensional matrix of Hawkes kernels. In the model of \citealp{BM14haw}, the following assumptions are in place on the so-introduced four-dimensional Hawkes process:
\begin{assumption}\label{assumption.BM14}
 \begin{enumerate}[label={\roman{*}.} , ref={\ref{assumption.BM14}.\roman{*}}]
  \item\label{assumption.BM14.symmetric_baseRates} (symmetric base rates)
  $\baseRate_{2k-1}=\baseRate_{2k}$,  for $k=1,2$;
  \item\label{assumption.BM14.null_price-jumps_baseRates} (fully endogenous price jumps) $\baseRate_3 = \baseRate_4=0$;
  \item \label{assumption.BM14.symmetric_cross-excitation}(symmetric cross-excitation)
  $\hawkesKernel_{2k-1,2j}=\hawkesKernel_{2j,2k-1}$, for $j,k=1,2$;
  \item \label{assumption.BM14.symmetric_self-excitation}(symmetric self-excitation)
  $\hawkesKernel_{2k-1,2j-1}=\hawkesKernel_{2k,2j}$, for $j,k=1,2$.
 \end{enumerate}
\end{assumption}
Assumption \ref{assumption.BM14.symmetric_baseRates} says that the base rate of arrival of sell market orders is equal to that of buy market orders, and that the base rate at which the mid-price increases is the same as the one at which it decreases. In fact, Assumption \ref{assumption.BM14.null_price-jumps_baseRates} adds that the base rates of mid-price jumps are null. This is interpreted by saying that sources of exogenous information about the fundamental price of the asset exchanged are brought to the orderbook by market orders, which then may or may not trigger changes in the mid-price. These mid-price changes can be triggered either by consuming the available liquidity at some price levels (instantaneous price adjustments) or by causing market makers to react and move limit orders from one level to another (delayed price adjustments). The effects of these triggers are modelled in the Hawkes kernels $\hawkesKernel_{k,j+2}$ for $j,k=1,2$. In Section \ref{sec.BM14_walking_the_book}, we will see that \citealp{BM14haw} reports that calibration on real world data showed that in fact delayed price adjustments have no explanatory power on mid-price jumps, and $\hawkesKernel_{1,3}$ and $\hawkesKernel_{2,4}$ will be turned into Dirac deltas to describe this.   

Assumptions \ref{assumption.BM14.symmetric_cross-excitation} and \ref{assumption.BM14.symmetric_self-excitation} account for the symmetries between bid/ask sides of the trades and up/down jumps of the mid-price. They result in a bloc-decomposition of the $4\times4$-matrix $\hawkesKernel$ into four $2\times2$-symmetric matrices of the form
\begin{equation*}
 \left(
 \begin{tabular}{cc}
  1 & 0 \\ 0 & 1
 \end{tabular}
 \right)s_h(t)
 +
 \left(
 \begin{tabular}{cc}
  0 & 1 \\ 1 & 0
 \end{tabular}
\right) c_h(t)
\end{equation*}
for some kernel functions $s_h(t)$ and $c_h(t)$, with $h=1,\dots,4$, which respectively represent self-excitation and cross-excitation. These blocs are labelled clock-wise starting from the top-left bloc. In the next section (Section \ref{sec.BM14_walking_the_book}), we focus on the bloc with label $h=2$ (top-right bloc), i.e. on the sub-matrix $\lbrace \hawkesKernel_{\eone,e}: \, \eone =1,2; \, \, e=3,4 \rbrace$.
\subsubsection{Market orders walking the book}\label{sec.BM14_walking_the_book}
There is a discrepancy between the probabilistic features of the stochastic process introduced above and one of the orderbook's mechanics that it is supposed to model. Indeed, assume that a trader places a buy market order (respectively,  sell market order) of a size larger than the available liquidity on the first ask level (on the first bid level).\footnote{In the notation of Proposition \ref{prop.lob_update}, such orders are identified by the condition $N>1$.} When such an order is executed, the limit orders sitting on the first ask (bid) level are depleted, and as a result the mid-price increases (decreases). When this happens, it is said that the market order \emph{walks the book}. We will also refer to such market orders as \emph{aggressive} market orders. 

A buy (sell) market order coming into the exchange and walking the book at time $t$ would be represented in the Bacry\&Muzy model by the equation $\delta\countingProc[2](t) = \delta\countingProc[4](t) = 1$ (respectively, $\delta\countingProc[1](t) = \delta\countingProc[3](t) = 1$). However, the components of a multidimensional Hawkes process jump simultaneously with probability zero. In other words, if $\lbrace (\arrivalTimes[e]_j)_j: \, e=1,.\dots,\numEventTypes\rbrace$ are the arrival times associated with a $\numEventTypes$-dimensional Hawkes process, it holds $\Prob(\arrivalTimes[e]_j = \arrivalTimes[\eone]_{j\derivative}, \, \text{for some } j,j\derivative\geq 1, \text{ and } e\ne\eone) = 0$.

A way around this discrepancy had therefore to be designed; Bacry and Muzy call it ``impulsive impact kernel''. 
Since the direction of the causality is unambiguous (a market order originates first and as a result of its execution the mid--price jumps), they propose to allow the diagonal entries in the top-right bloc of the Hawkes kernel to posses an atomic component. This means that,  with the notation introduced in the previous paragraph, they assume 
\begin{equation*}
 s_2(t) = \diracDelta_t + \tilde{s}_2(t),
\end{equation*}
where $\diracDelta_t$ is the Dirac delta concentrated at $t=0$, and $\tilde{s}_2 (t)$ is the non-atomic component of $s_2(t)$, i.e. a regular non-negative Lebesgue-integrable function defined on the non-negative half line. $\diracDelta_t$ accounts for instantaneous price adjustments when trades happen; $\tilde{s}_2 (t)$ accounts for delayed price adjustments after trades happened. Notice that the atomic component in the Hawkes kernel raises interesting mathematical questions about existence of an associated Hawkes process, which the authors defer to future works. From a numerical point of view, the atomic component $\diracDelta_t$ is encoded as a non-negative $\Lone$ function supported in some small right-neighbourhood of the origin and  with a certain positive $\Lone$-norm. 

Although a priori $\tilde{s}_2$ is not expected to be null, the calibration on real data performed in \citealp{BM14haw} finds that this is in fact the  case. Less surprisingly, $c_2 = c_2 (t)$ is also found to be identically null (buy market orders have no influence on price decrease, and sell market orders have no influence on price increase).  Consequently, the top-right bloc in the decomposition of $\hawkesKernel$ is a multiple of a $2\times2$-identity matrix, where the multiplying factor $s_2$ is equal to a Dirac delta concentrated in $t=0$. This form of the top-right bloc is the defining property of the ``impulsive impact kernel''. 

An impulsive impact kernel is therefore a way to model the fact that the influence of market orders on the changes in the mid-price is binary: either they instantaneously trigger a jump, or they have no influence on the mid-price.  In conjunction with Assumption \ref{assumption.BM14.null_price-jumps_baseRates}, this means that in Bacry\&Muzy model, exogenous information -- if any -- is carried to the orderbook by market orders, whose effect does not accrue over time. Instead, it is the flow of limit orders that is responsible for those price jumps that are not a result of market orders walking the book.

Non-market-orders-induced changes in the mid-price result either from new limit orders posted inside the spread, or from the cancellation of existing orders sitting on the first levels of either sides. The first alternative reduces the spread, the second widens it. Collectively, we refer to these modifications in the orderbook configuration as \emph{limit orders' re-adjustments}.   Bacry\&Muzy model does not keep track of limit orders, but captures their effect on the mid-price via the bottom-right bloc of the Hawkes kernels: $s_3=s_3(t)$ (respectively, $c_3=c_3(t)$)  models the effect that a price jump in one direction has on the likelihood of subsequent jumps in the same (opposite) direction resulting from limit orders' re-adjustments.

In Section \ref{sec.sdHawkes_model} we will present an alternative description of the orderbook dynamics which, on the one hand, will not necessitate atomic components in the Hawkes kernel; on the other hand, it will increase the granularity by keeping track of limit orders. Crucially, this will hinge on the introduction of a state-variable $\stateVariable$ paired with the  counting process $\multiCountingProc$. 

\subsubsection{Price impact profiles in Bacry\&Muzy model}\label{sec.BM14_price_impact_profiles}
In \citealp[Section 4]{BM14haw}, the authors study how a labelled agent's market orders can affect the evolution of the orderbook dynamics. This addresses the impact of a trader's own activity on the asset's price that she is trading, and is the main focus of our work. In this paragraph we describe Bacry\&Muzy approach in the case of a liquidation, i.e. the sale of some (usually large) amount of shares in a trading venue that operates via a limit order book.\footnote{In fact, \citealp{BM14haw} give a more general description in which the labelled agent can intervene in the market either by sell market orders, buy market orders or both.}

Let the instant at which the liquidator becomes active in the market be the origin of time. Let $A(t)$ be the number of sell market orders sent by the liquidator before time  $t$. The map $t\mapsto A(t)$ is called execution schedule, and describes how the liquidator is distributing the overall sale, or  \emph{metaorder}, over time.  In practice, the metaorder comprises several smaller \emph{child orders}, and $A$ is thought of as a description of the rate of their execution during the time window $[0,\timeHorizon)$. Notice, however, that the volumes of the child orders are left out from the model. 

From the point of view of the trading mechanism, $A(t)$ is similar to the first component $\countingProc[1](t)$ of the four-dimensional Hawkes process $\multiCountingProc(t)$ around which the orderbook dynamics are modelled: while $\countingProc[1](t)$ represents the aggregate anonymous quantity of sell market orders that all the market participants other than the liquidator have sent to the orderbook by time $t$, $A(t)$ represents the identifiable orders sent by the liquidator.  However, the mathematical representation of $A(t)$ is different from that of  $\countingProc[1](t)$. Indeed, $A(t)$ is assumed to be a non-negative non-decreasing right-continuous deterministic function, such that the associated measure $dA(t)$ is supported in the half-open interval $[0,\timeHorizon)$, for some $\timeHorizon>0$ representing the time horizon of the liquidation.

The presence of the liquidator affects the dynamics of the orderbook by contributing to the intensities of $\multiCountingProc$. For $e=1,\dots,4$ and $t>0$ it is set 
\begin{equation}\label{eq.BM14_intensity_with_liquidator}
 \intensity[e](t) = 
 \baseRate_e
 +\sum_{\eone=1}^{4}\int_{[0,t)} \hawkesKernel\subscriptee (t-s)d\countingProc[\eone](s) 
 + \int_{[0,t)} \theta_e(s)dA(s),
\end{equation}
where $\theta_1 = \theta_1(t)$ (respectively, $\theta_2=\theta_2(t)$) represents the impact of the liquidator's market orders on the arrival of other participants' sell (buy) market orders, and $\theta_3 = \theta_3(t)$ (respectively, $\theta_4 = \theta_4(t)$) represents the impact of the liquidator's market orders on downward (upward) jumps of the mid-price.

Notice that, from the point of view of the study of price impact, $\theta_3(t)$ and $\theta_4(t)$ are crucial aspects of the model. Since there is no a priori reason to assume that the liquidator's market orders have an effect on price different from other participants', it is set $\theta_3(t) = \hawkesKernel_{1,3}(t) = s_2(t)$ and $\theta_4(t)=\hawkesKernel_{1,4}(t)=c_2(t)=0$. Practically, this translates into the fact that the measure of the liquidator's price impact is mapped back to the estimated overall impact of other participants' market orders, and the liquidator's impact profile will result from the combination of the execution schedule $A(t)$ with these estimates.

More precisely, the price impact is assessed as follows. Because of the assumed symmetries (Assumption \ref{assumption.BM14}), in the absence of the liquidator (i.e. when $A\equiv 0$), the expected price movement $\Expectation [\countingProc[4](t) - \countingProc[3](t)]$ is null for all $t$. Instead, when $A(t)$ is non-constant,  the liquidator's intervention tends to depress this expected price. It is therefore the expected curve $t\mapsto \Expectation[\countingProc[4](t)-\countingProc[3](t)]$ that is defined as the price impact profile. A closed-form formula for this expectation is derived in \citealp[Proposition 5.1 and Corollary 5.2]{BM14haw}, where indeed the main ingredients are the execution schedule $A(t)$ and the functions $s_h(t)$, $c_h(t)$, for $h=1,\dots,4$. The closed-form formula has also the merit to reconcile this rather theoretical measurement with more empirical ones such as \citealp{BN13mar} and  \citealp{MVMGFVLM09mar}.  

Our proposed assessment of price impact profiles in Section \ref{sec.price_impact_profiles} is motivated by the will to offer alternatives to the following three choices of Bacry\&Muzy model. Firstly, we will attempt a measurement of the impact profile that is scenario-dependent, i.e. we will refrain from measuring the \emph{average} impact, yet factoring out the ``noise'' associated with the random evolution of the orderbook dynamics. Secondly, we will account for stochastic execution schedule $A(t)$, bringing the mathematical description of them closer to the one of $\countingProc[1](t)$. Thirdly, we will not assume that the liquidator's market orders impact the price jumps in the same way as other market orders do.

At a first glance, assuming $\theta_3(t) = s_2(t)$ seems very reasonable, but in connection with the impulsive impact kernel it limits some of the questions that the model could otherwise answer. Indeed, recall from the previous section (Section \ref{sec.BM14_walking_the_book}) that market orders have binary effects on price movements: either they deplete the available liquidity sitting at the best bid/ask prices (they walk the book), or they have no influence on the mid-price (when the liquidity that they demand is fully serviced by limit orders sitting on level 1 of either sides). Assuming that the liquidator's market orders impact the price jumps in the same way as other market orders do would therefore imply that the liquidator chooses to walk the book at an average rate equal to the overall proportion of aggressive market orders. Instead, one would like to have a measurement of price impact that could distinguish liquidator's strategies with different child orders' sizes. Possible questions could be: does a liquidator that never sends a child order for an amount larger than the best bid volume have no price impact? Or, is a liquidator with reliable forecasts for a future price downturn better off to send aggressive market orders before this downturn materialises? To which extent is her liquidation exacerbating the downturn or even triggering a fire sale? We will tackle these questions with a state-dependent Hawkes process, where the state variable will account for the volumes in the orderbook and it will interact with the liquidator's child orders' sizes. 
 
\subsection{State-dependent Hawkes model}\label{sec.sdHawkes_model}
Our state-dependent Hawkes model of an order-driven market is a more granular version of Bacry\&Muzy model that embeds in the event-state relation the mechanism underlying the ``impulsive impact kernel'' of \citealp{BM14haw}.
We will leave the interpretation of the first two components of the counting process as it was in their original model, but we will modify the third and the fourth components to account, rather than for proxy of the mid-price, for limit order events that put pressure on price decreases and price increases respectively. A proxy for the mid-price will be retrieved from the time-evolution of the state variable. 

More precisely, we consider four streams of arrival times: the stream $(\arrivalTimes[1]_j)_j$ of times when limit orders are executed on the bid side (equivalently identified with the arrival times of sell market orders); the stream  $(\arrivalTimes[2]_j)_j$ of times when limit orders are executed on the ask side (equivalently identified with arrival times of buy market orders); the stream  $(\arrivalTimes[3]_j)_j$ of times when a new limit order is queued on the ask side, or when an existing limit order on the bid side is cancelled from its queue; the stream  $(\arrivalTimes[4]_j)_j$ of times when a new limit order is queued on the bid side, or when an existing limit order on the ask side is cancelled from its queue. Limit orders posted on or cancelled from ask/bid sides refer to those events that happened on the first $n$ levels, where $n$ is a hyperparameter of the model that we fix at the beginning.

The four sequences of random times give rise to a four-dimensional counting process $\multiCountingProc=(\countingProc[1],\dots, \countingProc[4])$ with the following interpretation of its components:
\begin{itemize}
 \item $\countingProc[1](t)$ denotes the number of seller-initiated trades that happened before or at time $t$ (identified with the number of market orders arrived on the bid side of the orderbook by time $t$);
 \item $\countingProc[2](t)$ denotes the number of buyer-initiated trades that happened before or at time $t$ (identified with the number of market orders arrived on the ask side of the orderbook by time $t$);
 \item $\countingProc[3](t)$ denotes the number of limit orders posted on the ask side (i.e. on one of its first $n$ levels) by time $t$, plus the number of limit orders cancelled from the bid side (i.e. from one of its first $n$ levels) by time $t$;
 \item $\countingProc[4](t)$ denotes the number of limit orders posted on the bid side (i.e. on one of its first $n$ levels) by time $t$, plus the number of limit orders cancelled from the ask side (i.e. from one of its first $n$ levels) by time $t$.
\end{itemize}

While in Bacry\&Muzy model the difference $\countingProc[4] - \countingProc[3]$ was linked to a proxy for the mid-price, here in our model the same difference is linked to the order flow imbalance $\orderFlowImbalance$, because 
\begin{equation*}
 \sum_{i=1}^{n}\orderFlowImbalance_{\Delta T, \, i}(t) \approx  \frac{\countingProc[4](t-\Delta T,t)-\countingProc[3](t-\Delta T,t)}{\countingProc[4](t-\Delta T,t) + \countingProc[3](t-\Delta T,t)},
\end{equation*}
where $\countingProc(t-\Delta T, t) = \countingProc(t) - \countingProc(t-\Delta T)$ is the number of  events of type $e$ that occured in the time window $[t-\Delta T, t]$. This is a proxy rather than the actual order flow imbalance for two reasons. The first is that it only tracks the number of limit orders posted or cancelled and not their sizes; the second is that  in the actual order flow imbalance ask (respectively, bid) volumes are diminished by executions of buy (sell) orders, which are not taken into account in the difference $\countingProc[4] - \countingProc[3]$. A way to take into account executions of orders could be to replace $\countingProc[4](t-\Delta T, t)$ with $(\countingProc[4]-\countingProc[1])(t-\Delta T, t)$, i.e.  the net flow of limit orders on the bid side, and $\countingProc[3](t-\Delta T, t)$ with $(\countingProc[3]-\countingProc[2])(t-\Delta T, t)$, i.e.  the net flow of limit orders on the ask side. However, because of the far larger number of limit orders compared to market orders, the given expression already approximates the order flow imbalance in a satisfactory way. 

The counting process $\multiCountingProc$ is paired with the state variable $\stateVariable$. At time $t$, the state variable $\stateVariable(t)$ summarises the configuration $(\bestBidPrice_t,\bestAskPrice_t,\lbrace( \nthBestBidVolume[i]_t, \nthBestAskVolume[i]_t): \, \, i=1,2,\dots \rbrace)$ of the limit order book at time $t$, by recording a proxy for the $n$-levels volume imbalance, and the variation of the mid-price compared to time $t-$. More precisely, 
\begin{equation}\label{eq.state-variable}
 \stateVariable(t)
 =
 \begin{pmatrix}
 \stateVariable_1 (t)
 \\
 \stateVariable_2(t)
 \end{pmatrix}
 :=
 \begin{pmatrix}
 \one\lbrace\delta\midprice(t)>0\rbrace - \one\lbrace\delta\midprice(t)<0\rbrace
 \\
 \half\sum_{k=0}^{K-1} (2k-K+1 )\one\left\lbrace \frac{k-K}{K} \leq \volumeImbalance^n_t < \frac{2(k+1)-K}{K} \right\rbrace
 \end{pmatrix},
\end{equation}
where $\delta\midprice(t) = \lim_{\epsilon\downarrow 0} (\midprice(t) - \midprice(t-\epsilon) )$, and $\volumeImbalance^{n}_t$ was defined in equation \eqref{eq.definition_volume_imbalance}. The first component $\stateVariable_1$ of the state variable $\stateVariable$ can take the values $-1$, $0$, $+1$, respectively denoting downward jump in the mid-price, unchanged mid-price, and upward jump in the mid-price. The second component $\stateVariable_2$ of the state variable $\stateVariable$ is a discretisation of the $n$-levels volume imbalance $\volumeImbalance^{n}_t$, and -- assuming that $K$ is odd -- it takes integer values from $-(K-1)/2$ to $(K-1)/2$, spanning the full range of possible values of $\volumeImbalance^n$ from $-1$ to $+1$. 
It follows from the definition of $\stateVariable_2$ that if at time $t$ we have that $\stateVariable_2 (t) = x_2$, then the $n$-levels queue imbalance $\volumeImbalance^n_t$ at time $t$ must be in the half-open interval $[(2x_2 -K -1)/2K, \, \, (2x_2 +1)/K \, [$.
Notice that $\stateVariable_2$ depends on the two additional parameters $n$ and $K$: the former is the number $n$ of levels of the limit order books taken into account in the computation of the volume imbalance $\volumeImbalance^n$ (this is the same number $n$ of levels considered for the definition of the components $\countingProc[3]$ and $\countingProc[4]$); the latter is the number $K$ of points in the partition of the interval $[-1,1]$  used for the discretisation of $\volumeImbalance^n$. 

The pair $(\multiCountingProc, \stateVariable)$ is modelled as a state-dependent Hawkes process, hence we assume that there are base rates $\baseRate_e$, Hawkes kernels $\hawkesKernel\subscriptee=\hawkesKernel\subscriptee(t, \xone)$ and transition matrices $\transProb_e$ such that  Definition \ref{def.sdHawkes} is satisfied. The number of event types is $\numEventTypes = 4$ and the number of states is $\numStates = 3K$, since the state variable $\stateVariable = (\stateVariable_1, \stateVariable_2)$ is such that $\stateVariable_1$ can take three possible values, and $\stateVariable_2$ can take $K$ possible values. Furthermore, we assume the parametric form of equation \eqref{eq.powerlaw_kernels} for the state-dependent Hawkes kernels; this guarantees that their decay in the limit as $t\uparrow\infty$ is aligned to the findings in \citealp{BM14haw}. 

When a new event occurs, i.e. when one of the components $\countingProc$ of $\multiCountingProc$ jumps, the state variable $\stateVariable$ is updated as per in equation \eqref{eq.markov_update_stateVariable}. In the case of $e=1,2$, the update of the first component $\stateVariable_1$ models the mechanism that led to the ``impulsive impact kernel'', providing an alternative description for it that does not require the introduction of singular components in the Hawkes kernel. Indeed, assume that a sell (buy) market order arrives at time $\arrivalTimes[1]_j$ ($\arrivalTimes[2]_j$, respectively), and that $\stateVariable(\arrivalTimes[1]_j -) = (x_1,x_2)$ (respectively, $\stateVariable(\arrivalTimes[2]_j -) = (x_1,x_2)$) for some $x_1$ in $\lbrace -1, 0, +1\rbrace$ and some $x_2$ in $\lbrace (1-K)/2, (3-K)/2, \dots, (K-1)/2 \rbrace$. Then, the mid-price jumps downward (upward) with probability $p_{-}:=\sum_{y_2 = (1-K)/2}^{(K-1)/2} \transProb_{1}((x_1, x_2), (-1,y_2))$ (respectively, $p_{+}:=\sum_{y_2 = (1-K)/2}^{(K-1)/2} \transProb_{2}((x_1, x_2), (+1,y_2))$), and it remains unchanged with probability $p_{0}:=1-p_{-}=\sum_{y_2 = (1-K)/2}^{(K-1)/2} \transProb_{1}((x_1, x_2), (0,y_2))$  (respectively, $p_{0}:=1-p_{+}=\sum_{y_2 = (1-K)/2}^{(K-1)/2} \transProb_{2}((x_1, x_2), (0,y_2))$).\footnote{ There is no chance that a sell (buy) market order can cause an increase (decrease, respectively) in the mid-price.} This jump of the state variable happens exactly at the arrival time $\arrivalTimes[1]_j$ (respectively, $\arrivalTimes[2]_j$)  of the sell (buy) market order, and it naturally captures the mechanism responsible for the market-order-induced price change: $p_{-}$ (respectively, $p_{+}$) represents the probability that a sell (buy) market order  walks the book given its submission, and $p_{0}$  represents the probability that it does not. Notice that $p_{-}$ (respectively, $p_{+}$) and $p_{0}$ depend on the state of the limit order book immediately before the arrival of the sell (buy) market order, and in particular they depend on $x_2$. This enriches the description of the orderbook mechanism, accounting for the fact that it is less likely that a sell (buy) market order walks the book when the volumes on the bid (ask) side are high, namely $p_{-}(x_1,x_2) \leq p_{-}(x_1,\tilde{x}_2)$ if $x_2\geq \tilde{x}_2$ (respectively, $p_{+}(x_1,x_2) \leq p_{+}(x_1,\tilde{x}_2)$ if $x_2\leq \tilde{x}_2$).

Moreover, the first component $\stateVariable_1$ of the sate variable $\stateVariable$ enables to write the following proxy for the mid-price:
\begin{equation}\label{eq.midprice_proxy}
 \midprice_0 + \frac{\tickSizeOfLOB}{2} \intzerot \stateVariable_1(s) d\groundProc(s),
\end{equation}
where $\tickSizeOfLOB$ is the tick size of the limit order book and $\groundProc = \countingProc[1] + \dots + \countingProc[4]$ is the ground process of $\multiCountingProc$. This proxy for the mid-price  is the same as the one used in \citealp{BM14haw}, albeit obtained via different variables. 

The second component $\stateVariable_2$ of the state variable $\stateVariable$ reproduces the state variable of the queue-imbalance model in \citealp{MP18sta}. It is conceived as the main indicator of the regime in which limit and market orders will arrive to the exchange: in high-frequency markets trading algorithms send their orders in response to observable quantities of the limit order book configuration, and a prominent one is indeed the queue imbalance. It is therefore expected that when $\stateVariable_2$ is positive (negative), the intensities of events of types $e=2,4$ (respectively, $e=1,3$) will be higher, because market participants following the queue imbalance signal will expect the price to increase (decrease). After the price change,  the volumes of deeper queues on the ask (bid) side enter the computation of the queue imbalance, and this will likely reset the signal.  As noted in \citealp{MP18sta}, this interaction can be deemed responsible for the mean-reverting behaviour of price dynamics in high-frequency markets. 


\subsubsection{Volume modelling}
Our measurement of price impact will also require to reproduce the update to the limit order book configuration that happens when an order arrives. The precise mechanics of this update were described in Proposition \ref{prop.lob_update}. In order to reproduce them, a proxy for the volumes $\lbrace (\nthBestAskVolume[i]_t, \nthBestBidVolume[i]_t): \, i=1,2,\dots\rbrace$ in the limit order book at time $t$ needs to be derived from the state variable $\stateVariable(t)$. In this paragraph, we describe how this is done. 

Recall that the hyperparameter $n$ of the model is fixed and it denotes the number of levels considered for the definition of the third and fourth components of $\multiCountingProc$.  The same hyperparameter $n$ also enters in the definition of the second component $\stateVariable_2$ of the state variable, which keeps track of a $K$-level discretisation of the queue imbalance $\volumeImbalance^n$. We consider normalised volumes up to level $n$, namely we assume that $\sum_{i=1}^{n} (\nthBestBidVolume[i]_t + \nthBestAskVolume[i]_t) \equiv 1$. All volumes in the orderbook and all sizes of market/limit orders are expressed in this unit.  Correspondingly, we consider the $2n$-tuple $(\nthBestAskVolume[1]_t,\nthBestBidVolume[1]_t, \dots, \nthBestAskVolume[n]_t,\nthBestBidVolume[n]_t)$ and we assume that it is distributed as a Dirichlet random variable with $2n$-dimensional parameter $\dirichletParam=\dirichletParam(\stateVariable(t)) \in \R_{+}^{2n}$ that depends on the state variable at time $t$. Given the time evolution of the limit order book in the time window $\timeWindow$, an estimator for $\dirparam(x)$, with $x$ ranging from $1$ to $3K$, can easily be obtained by maximum likelihood estimation. Once $\dirparam$ is known, the mechanics in Proposition \ref{prop.lob_update} can be reproduced by drawing from the conditional distribution $\text{Dir}_{\dirichletParam(\stateVariable(t))}(\cdot \vert \stateVariable_2 (t) )$; this is the Dirichlet distribution of  the $2n$-tuple $(\nthBestAskVolume[1]_t,\nthBestBidVolume[1]_t, \dots, \nthBestAskVolume[n]_t,\nthBestBidVolume[n]_t)$ with parameter $\dirparam(\stateVariable(t))$ conditioned on 
\begin{equation*}
 \frac{2\stateVariable_2(t) - K - 1}{2K}
 \leq 
 \underbrace{\sum_{i=1}^{n} \left( \nthBestBidVolume[i]_t - \nthBestAskVolume[i]_t \right)}_{= \volumeImbalance^n_t}
 <
 \frac{2\stateVariable_2(t) +1}{K}.
\end{equation*}
We rely on rejection sampling to perform such drawing. The details of our implementation of this sampling technique will be given in Section \ref{sec.pipest.rejection_sampling}. 
Here, in Algorithm \ref{algo.lob_update}, we describe how to reproduce the orderbook update in the case of the arrival of a sell market order $(t,q_M, 0, -1)$. This is the case that we will actually use in our measurement of price impact, because we will adopt the perspective of liquidation. The case of buy market orders is analogous. 

\begin{algorithm}[h]
 \caption{State update via orderbook mechanics (sell market order)}
 \label{algo.lob_update}
 \begin{algorithmic}[5]
  \REQUIRE $\stateVariable(t-)$, $(t,q_M, 0, -1)$
  \STATE set $\ell := (2\stateVariable_2(t-) - K - 1)/(2K)$
  \STATE set $u:= (2\stateVariable_2(t-) +1)/K$
  \STATE sample $V_{t-} = (\nthBestAskVolume[1]_{t-},\nthBestBidVolume[1]_{t-},\dots,\nthBestAskVolume[n]_{t-},\nthBestBidVolume[n]_{t-}) \sim \text{Dir}_{\dirichletParam(\stateVariable(t-))}(\cdot \vert \stateVariable_2 (t-) )$ 
  \STATE set $\stateVariable_1(t) := -1$ if $q_M \geq \nthBestBidPrice[1]_{t-}$; $0$ otherwise
  \STATE initialise $q:=0$
  \FOR {$i$ in $1,\dots,n$}
  \STATE set $v:= \min_{+} (\nthBestBidVolume[i]_{t-}, q_M - q )$
  \STATE set $\nthBestBidVolume[i]_{t} := \nthBestBidVolume[i]_{t-} - v$
  \STATE update $q\leftarrow q+v$
  \ENDFOR
  \STATE set $B:= \sum_{i=1}^{n} \nthBestBidVolume[i]_t$
  \STATE set $A:= \sum_{i=1}^{n} \nthBestAskVolume[i]_{t-}$
  \STATE set $I:= (B-A)/(B+A)$ 
  \STATE set $\stateVariable_2(t):= (2k-K+1)/2$ if $(k-K)/K \leq I < (2k+2 -K)/K$, where $k=0,\dots,K-1$
  \RETURN $(\stateVariable_1(t),\stateVariable_2(t))$
 \end{algorithmic}
\end{algorithm}

Line 4 and Lines 6:10 in Algorithm \ref{algo.lob_update} account for equations \eqref{eq.lob_update_sell_order} in Proposition \ref{prop.lob_update}. Line 4 says that the bid price (and consequently the mid-price) decreases if the size of the sell market order is larger than the available liquidity sitting on the first bid level. Lines 6:10 cancel from the bid queues the orders whose execution has been triggered by the arrival of $(t,q_M,0,-1)$.  On line 7 we used the notation $\min_{+}(a,b) = \max(0,\min(a,b))$ for $a$ and $b$ real numbers. 

\begin{remark}\label{remark.order_size_after_volume_sampling}
 Algorithm \ref{algo.lob_update} implicitly requires that the size $q_M$ of the market order is known before the volume proportions $(\nthBestAskVolume[1]_{t-},\nthBestBidVolume[1]_{t-},\dots,\nthBestAskVolume[n]_{t-},\nthBestBidVolume[n]_{t-})$. In practice, it might happen that the size of a market participant's order is a function of the volumes observed in the orderbook. We can reproduce this by adding after the sampling of $V_{t-}$ on Line 3 the computation of $q_M$, and then proceed along the same lines.
\end{remark}


\section{Price impact profiles}\label{sec.price_impact_profiles}
% Describe how the labelled agent (i.e. the liquidator) is introduced in the market model; define the impact profile \`a la Bacry-Muzy and the normalised one-sided impact profile.  

Measuring price impact requires two things. The first is to modify the  model $(\multiCountingProc,\stateVariable)$ of Section \ref{sec.sdHawkes_model} in a way to account for a labelled agent, whose impact we wish to measure. The second is to extrapolate to which extent the labelled agent is responsible for the evolution of the price dynamics that emerge from the state process $(\stateVariable(t))_{t}$.  Section \ref{sec.labelled_agent} describes the former; Sections \ref{sec.impact_a_la_BM} and \ref{sec.one-side_impact} will present two different ways to do the latter.

\subsection{Labelled agent}\label{sec.labelled_agent}
We account for a labelled agent in the market, and we aim to measure her impact on the dynamics of the orderbook. We take the perspective of a liquidation, namely we consider our agent (also referred to as liquidator) to be selling the amount $\metaorderSize$ of asset. The case of acquisition is mutatis mutandis the same. 

We let $\timeWindow$ represent the time window of the liquidation. The quantity $\metaorderSize$ will be referred to as the size of the liquidator's metaorder, or her initial inventory, and it will be normalised with respect to the overall volume $\sum_{i=1}^{n}(\nthBestAskVolume[i]_{0} + \nthBestBidVolume[i]_{0})$ of offers sitting on the first $n$ levels of the orderbook  at the start of the liquidation window. Hence, we say that the quantity $\metaorderSize$ is liquidity-normalised (or liquidity-denominated). Recall that $n$ is the number of levels taken into account for the evaluation of the queue imbalance $\volumeImbalance^n$ that enters the definition of the second component $\stateVariable_2$ of the state variable in equation \eqref{eq.state-variable}.     

We assume that the liquidator intervenes in the market only by sending sell market orders; she will never place a limit order to queue on the ask side, but she will initiate trades with existing offers on the bid side. 
Hence, the liquidation is described by the sequence $\lbrace (\arrivalTimes[0]_j, q_{M,j},0,-1): \, \, j=1,2,\dots \rbrace$ of sell market orders sent by the liquidator. For every $j$, $\arrivalTimes[0]_j$ is the time stamp of the liquidator's $j$-th child market order, and $q_{M,j}$ is its size. 
We suppose that the stream of (random) times $\arrivalTimes[0]_1 < \arrivalTimes[0]_2 < \dots$ is confined in $\timeWindow$. We assume non-explosiveness, so that the number of liquidator's market orders is finite if the time horizon $\timeHorizon$ of the execution window is not $+\infty$. Moreover, we let $\initialTime = \arrivalTimes[0]_1$ represent the time at which the liquidator begins her intervention in the market, and we let $\terminationTime:=\sup\lbrace\arrivalTimes[0]_j\leq\timeHorizon: \, j=1,2,\dots\rbrace$ be the time at which the liquidation stops. 
For every $j$, the size $q_{M,j}$ is liquidity-normalised, i.e. it is expressed as a fraction of the overall volume $\sum_{i=1}^{n}(\nthBestAskVolume[i]_{\arrivalTimes[0]_j-} + \nthBestBidVolume[i]_{\arrivalTimes[0]_j-})$ of offers sitting on the first $n$ levels of the orderbook immediately before the arrival of the liquidator's $j$-th market order.

Because of our normalisation of volumes, the map  $t\mapsto \sum_{i=1}^{n}(\nthBestAskVolume[i]_t +\nthBestBidVolume[i]_t)$ is constantly equal to $1$. In real-world limit order books, the overall volume of offers in the order book may change over time, so that we are in fact normalising by a non-constant quantity. However, we assume that this variability is irrelevant for our purposes, in the sense captured by the following 
\begin{assumption}\label{assumption.termination-time_of_liquidation}
Let $\metaorderSize$ be the size of the liquidator's metaorder, and for $j$ in $\N$,  let $z_j = q_{M,1}+\dots+q_{M,j}$ be the sum of all liquidity-normalised sizes of the first $j$ child market orders sent by the liquidator. Then, the termination time $\terminationTime$ of the liquidation is assumed to coincide with the smallest time stamp $\arrivalTimes[0]_j$ among the liquidator's market orders such that $z_j \geq \metaorderSize$, namely
 \begin{equation*}
\terminationTime = \inf\left\lbrace \arrivalTimes[0]_j \geq \initialTime : \, \, 
 \sum_{k=1}^{j} q_{M,k} \geq \metaorderSize \right\rbrace.
 \end{equation*}
\end{assumption}

We introduce the liquidator's presence in the model described in Section \ref{sec.sdHawkes_model} by expanding the dimension of the counting process $\multiCountingProc$: we let the zero-th component $\countingProc[0](t)$ count the liquidator's market orders sent to the exchange by time $t$. In other words, from the overall sequence $(\arrivalTimes[1]_j)_j$ of arrival times of market orders described in Section \ref{sec.sdHawkes_model}, we extract those sent by the liquidator and we label them as $(\arrivalTimes[0]_j)_j$; we then let 
\begin{equation*}
 \countingProc[0](t):=\sum_{j\geq 1}\one \left\lbrace \arrivalTimes[0]_j \leq t \right\rbrace
\end{equation*}
count the number of trades initiated by the liquidator that happened before or at time $t$. Notice that the map $t\mapsto \countingProc[0](t)$ represents how the liquidator is splitting in time the execution of her metaorder and thus, in the terminology of Section \ref{sec.BM14_price_impact_profiles}, it is the liquidator's execution schedule, which was there denoted with the symbol $A(t)$.

The pair $(\multiCountingProc, \stateVariable)$ will be a state-dependent Hawkes process where the counting process component $\multiCountingProc$ is five-dimensional and has power-law kernels, and the state process $\stateVariable$ is as in equation \eqref{eq.state-variable}. The event types will be labelled $e=0,1,\dots,4$ and the states will be labelled $x=1,\dots,3K$ or $x=(x_1,x_2)$ with $x_1 = -1,0,+1$ and $x_2 = -(K-1)/2,\dots,+(K+1)/2$.  The following assumption is in place on the intensities:
\begin{assumption}\label{assumption.imp_and_dec_coef_from_liquidator}
 For all $e=1,\dots,4$, the Hawkes kernel $\hawkesKernel_{0,e}$ coincides with  $\hawkesKernel_{1,e}$, namely
 \begin{equation*}
  \impCoef_{0,\xone,e}=\impCoef_{1,\xone,e} , \quad \decCoef_{0,\xone,e}= \decCoef_{1,\xone,e}, 
 \end{equation*}
for all $e=1,\dots,4$ and all $\xone = 1,\dots, 3K$. 
\end{assumption}

Assumption \ref{assumption.imp_and_dec_coef_from_liquidator} will always be in place in what follows. It says that the rates of arrival of market and limit orders to the exchange are modified by the liquidator's interventions in the same way as they are by other participants' sell market orders. More precisely, for $e=1,\dots,4$ it holds 
\begin{equation}\label{eq.intensities_when_liquidator_is_present}
\begin{split}
\intensity[e](t) =& \baseRate_e 
 + \sum_{\eone = 0}^{4}\int_{[0,t)} \hawkesKernel_{\eone,e}(t-s,\stateVariable(s))d\countingProc[\eone](s)
 \\
 =& \baseRate_e 
 + \sum_{\eone = 1}^{4}\int_{[0,t)} \hawkesKernel_{\eone,e}(t-s,\stateVariable(s))d\countingProc[\eone](s)
 + \int_{[0,t)} \hawkesKernel_{1,e}(t-s,\stateVariable(s))d\countingProc[0](s).
 \end{split}
\end{equation}
This is our state-dependent analogue to equation \eqref{eq.BM14_intensity_with_liquidator}, where however the execution schedule $t \mapsto A(t)=\countingProc[0](t)$ is now stochastic. The stochastic behaviour of the liquidator's execution schedule is parametrised by $\baseRate_0$, $\impCoef_{\eone,\xone,0}$ and $\decCoef_{\eone,\xone,0}$ for $\eone = 0,\dots,4$ and $\xone=1,\dots,3K$. These parameters represent decision variables for the liquidator. As a consequence, the liquidator's execution schedule admits an absolutely continuous compensator $\compensator_{0}$ with density 
\begin{equation}\label{eq.intensity_of_liquidator}
 \intensity[0](t)
 =
 \baseRate_0 \one_{[0,\terminationTime)}(t) + 
 \sum_{\eone,\xone}\one_{[0,\terminationTime)}(t)
 \int_{[0,t)} \hawkesKernel_{\eone,0} (t-s,\xone) d\hybridHawkes_{\eone,\xone}(s),
\end{equation}
where $\hawkesKernel_{\eone,0}(t,\xone) = \impCoef_{\eone,\xone,0}(t+1)^{-\decCoef_{\eone,\xone,0}}$, and $\terminationTime$ is the termination time of the liquidation. 

A simple configuration for the execution schedule is given in the following definition. 
\begin{defi}\label{def.poissonina_liquidation}
The execution schedule $t \mapsto \countingProc[0](t)$ is said to be of constant intensity if $\impCoef_{\eone,\xone,0} = 0$ for all $\eone = 0,\dots,4$ and all $\xone=1,\dots,3K$. In this case we talk about a Poissonian liquidation with rate $\baseRate_0$. 
\end{defi}

Assumption \ref{assumption.imp_and_dec_coef_from_liquidator} guarantees consistency in the effect that trades have on the orderbook dynamics. Nonetheless, since none of the components of $\multiCountingProc$ represents price jumps, this consistency assumption does not imply that the number of liquidator's market orders that  walk the book  is proportional to the number of those sent by other participants, as it did in Bacry\&Muzy model. Instead, this proportion is reserved to the liquidator's decision variables, and it will stem from a nuanced interaction between her orders and the volumes of offers on the bid side. 

More precisely, for $j=1,2,\dots$ let $(\arrivalTimes[0]_j, q_{M,j},0,-1)$ be the liquidator's child market orders, as denoted above. The liquidator controls the (stochastic) scheduling of her orders by setting the values of the Hawkes parameters $\baseRate_0$, $\impCoef_{\eone,\xone,0}$ and $\impCoef_{\eone,\xone,0}$, which modulate the sequence of arrival times $\arrivalTimes[0]_j$. Additionally, for every $j$ she sets the size $q_{M,j}$ of her $j$-th child market order, satisfying the measurability constraint $q_{M,j} \, \hat{\in} \, $ $ \filtrationF_{\arrivalTimes[0]_j -} = \bigvee_{\epsilon>0}\filtrationF_{\arrivalTimes[0]_j -\epsilon}$, where $\filtrationF$ is a history of $(\multiCountingProc,\stateVariable)$.  Then, the evolution of the limit order book is simulated by combining Algorithm \ref{algo.MP18_ogata} and Algorithm \ref{algo.lob_update}, as detailed in Algorithm \ref{algo.lob_with_liquidator}.

\begin{algorithm}[h]
\caption{Simulation of orderbook in the presence of liquidator}
\label{algo.lob_with_liquidator}
 \begin{algorithmic}[5]
  \REQUIRE $(\arrivalTimes[]_i,\event[i],\stateVariable_i)_{i=1,\dots,n-1}$, $q_M$
  \STATE \textbf{do} Lines 1:8 of Algorithm \ref{algo.MP18_ogata}
  \STATE draw $\event[n]$ with probabilities proportional to $\lbrace \intensity[0](\arrivalTimes[]_n),\intensity[1](\arrivalTimes[]_n),\dots,\intensity[4](\arrivalTimes[]_n)\rbrace$
  \IF {$\event[n] = 0$}
  \STATE set $\stateVariable_n \leftarrow$ Algorithm \ref{algo.lob_update}($\stateVariable_{n-1}$,$(\arrivalTimes[]_n,q_M,0,-1)$)
  \ELSE
  \STATE draw $\stateVariable_n$ in $\lbrace 1,\dots,3K\rbrace$ with probabilities $\lbrace \transProb_{\event}(\stateVariable_{n-1},1), \dots,\transProb_{\event}(\stateVariable_{n-1},3K)\rbrace$
  \ENDIF
  \RETURN $(\arrivalTimes[]_n,\event[n],\stateVariable_n)$
 \end{algorithmic}
\end{algorithm}

\begin{remark}
 Notice that Algorithm \ref{algo.lob_with_liquidator} requires the size $q_M$ of the next liquidator's order. Assuming that $j-1$ liquidator's orders have already been executed, then this size is $q_{M,j}$. If this size can be computed beforehand (i.e. $q_{M,j}$ is measurable with respect to $\mathcal{H}_{\arrivalTimes[]_{n-1}}$ where $\mathcal{H}$ is the internal history of $(\multiCountingProc,\stateVariable)$), then the algorithm takes  $q_M \leftarrow q_{M,j}$ as a non-negative real number. If instead this is not possible, $q_M$ is passed as a method to compute the $j$-th liquidator's order size. See also Remark \ref{remark.order_size_after_volume_sampling}. 
\end{remark}

Once the liquidator's execution has been simulated together with the whole orderbook dynamics following the steps of Algorithm \ref{algo.lob_with_liquidator}, we can estimate the transition matrix $\transProb_{0}$ corresponding to the  state update that happens when a liquidator's order arrives to the market. More precisely, for $x$ and $y$ in $\stateSpace$, we set 
\begin{equation}\label{eq.transProb_liquidator}
  \transProb_{0}(x,y) = \Prob\left(\stateVariable(t) = y \big\vert \Delta\countingProc[0](t)=1, \, \stateVariable(t-)=x\right) = \frac{\sum_j \one\lbrace\stateVariable(\arrivalTimes[0]_j - )=x, \, \, \stateVariable(\arrivalTimes[0]_j )=y\rbrace} {\sum_j \one\lbrace\stateVariable(\arrivalTimes[0]_j - )=x, \rbrace}.
\end{equation}
This matrix $\transProb_0$ will enter our formulae for the price impact profile. We underline that the estimation of this matrix is done \emph{a posteriori}: it is neither a decision variable for the liquidator, nor it is assumed to match the transition matrix corresponding to event type 1, because the sizes of liquidator's orders can differ from the average size of other participants' orders. 

A simple case for the sizing of the liquidator's market orders is given in the following definition. 
\begin{defi}\label{def.liquidation_with_constant_size}
 Let $\lbrace (\arrivalTimes[0]_j, q_{M,j}, 0, -1): \, j=1,2,\dots\rbrace$ be the sequence of liquidator's market orders. We say that the liquidation is of constant order size if there exists a fixed real number $q_{M}\geq 0$ such that $q_{M,j} = q_{M}$ for all $j$.
\end{defi}

\begin{prop}\label{prop.termination_for_constant-size_liquidation}
 Let $\lbrace (\arrivalTimes[0]_j, q_{M}, 0, -1): \, j=1,2,\dots\rbrace$ represent a liquidation of constant order size. Let $\metaorderSize$ be the liquidator's initial inventory, and let $\terminationTime = \sup\lbrace \arrivalTimes[0]_j : j=1,2,\dots\rbrace$. Then, under Assumption \ref{assumption.termination-time_of_liquidation}, it holds
 \begin{equation}\label{eq.termination_for_constant-size_liquidation}
 \terminationTime = \inf \left\lbrace t\geq 0: \, \, \countingProc[0](t)\geq 
 \left\lceil \frac{\metaorderSize}{q_{M}}\right\rceil \right\rbrace
 \end{equation}
 where $\lceil x\rceil$ is the smallest integer greater than or equal to $x$. 
\end{prop}

Another way to express equation \eqref{eq.termination_for_constant-size_liquidation} is to say that $\countingProc[0](t) = \lceil\metaorderSize/q_{M}\rceil$ for all $t\geq \terminationTime$.
Hence, combining constant order size and constant liquidation intensity gives a particularly simple case where the distribution of the termination time is fully known. We will refer to this case when studying the one-sided impact profile in Section \ref{sec.one-side_impact}.

\begin{prop}\label{prop.terminationTime_poisson_and_constant-size}
 Under Assumption \ref{assumption.termination-time_of_liquidation}, a Poissonian liquidation with constant order size $q_{M}$ is such that its termination time $\terminationTime=\inf\lbrace \countingProc[0](t)\geq \lceil \metaorderSize / q_{M}\rceil \rbrace$ satisfies 
 \begin{equation}\label{eq.hazard_terminationTime}
  \Prob(\terminationTime >t ) 
  =
  \sum_{n=0}^{\lceil \metaorderSize / q_{M} \rceil -1 }
  \frac{(\baseRate_0 t)^{n}}{n!}\exp \Big(-\baseRate_{0} t \Big)
 \end{equation}
and 
\begin{equation}\label{eq.expected_terminationTime}
 \Expectation [\terminationTime \wedge t ]
 =
 \frac{\lceil \metaorderSize / q_{M} \rceil}{\baseRate_0}
 - \frac{1}{\baseRate_{0}}
  \sum_{n=0}^{\lceil \metaorderSize / q_{M} \rceil -1 }
  \left( \left\lceil \frac{\metaorderSize}{q_{M}}\right\rceil - n \right) 
  \frac{(\baseRate_0 t)^{n}}{n!}\exp \Big(-\baseRate_{0} t \Big),
\end{equation}
for all $t\geq 0$.
\end{prop}
\begin{proof}
 Equation \eqref{eq.hazard_terminationTime} follows from the fact that $\terminationTime$ is the arrival time of the $\lceil \metaorderSize/q_{M} \rceil$-th point of a Poisson process with  intensity $\baseRate_{0}$. Equation \eqref{eq.expected_terminationTime} follows from writing $
  \Expectation[\terminationTime \wedge t ] = 
  \intzerot \Prob(\terminationTime > s ) ds
 $
 and using Lemma \ref{lemma.expected_terminationTime} below.
\end{proof}
\begin{lemma}\label{lemma.expected_terminationTime}
 For any non-negative integer $N$ and real numbers $y_1<y_2$ it holds
 \begin{equation*}
  \int_{y_1}^{y_2}
  e^{-x}\sum_{n=0}^{N} \frac{x^{n}}{n!}dx
  =
  \sum_{n=0}^{N}
  (N-n+1)
  \left(
  \frac{y_{1}^{n}}{n!} e^{-y_1} -  \frac{y_{2}^{n}}{n!} e^{-y_2} 
  \right).
 \end{equation*}
\end{lemma}
\begin{proof}
 We claim by induction over $n$ that 
 \begin{equation}\label{eq.induction_in_lemma.expected_terminationTime}
   \int_{y_1}^{y_2}
   \frac{x^{n}}{n!}e^{-x}dx
   \quad
  =
   \quad 
  e^{-y_1} \sum_{k=0}^{n} \frac{y_{1}^{k}}{k!}  -  e^{-y_2}\sum_{k=0}^{n}\frac{y_{2}^{k}}{k!}.
 \end{equation}
When equation \eqref{eq.induction_in_lemma.expected_terminationTime} is established, the lemma follows from 
\begin{equation*}
  e^{-y_1} \sum_{n=0}^{N}\sum_{k=0}^{n} \frac{y_{1}^{k}}{k!}  -  e^{-y_2}\sum_{n=0}^{N}\sum_{k=0}^{n}\frac{y_{2}^{k}}{k!}
 \quad  = \quad 
 \sum_{k=0}^{N}\sum_{n=k}^{N} \left(\frac{y_{1}^{k}}{k!}e^{-y_1}  -  \frac{y_{2}^{k}}{k!}e^{-y_2}\right),
\end{equation*}
which is just a rearrangement of the order of summations.

The case $n=0$ in equation \eqref{eq.induction_in_lemma.expected_terminationTime} is apparent. Hence, for $n\geq0$ observe that
\begin{equation*}
 \frac{d}{dx} \left( \frac{x^{n+1}}{(n+1)!}e^{-x} \right)
 = 
 \frac{x^{n}}{n!}e^{-x}  -  \frac{x^{n+1}}{(n+1)!}e^{-x},
\end{equation*}
whence
\begin{equation*}
 \frac{y_{2}^{n+1}}{(n+1)!}e^{-y_{2}}  -  \frac{y_{1}^{n+1}}{(n+1)!}e^{-y_{1}}
 =
 \int_{y_{1}}^{y_{2}}
 \left(\frac{x^{n}}{n!}  -  \frac{x^{n+1}}{(n+1)!}\right)e^{-x},
\end{equation*}
and this proves the induction step.
\end{proof}






\subsection{Price impact profiles \`a la Bacry-Muzy}\label{sec.impact_a_la_BM}
We partition the state space $\stateSpace$ according to the values of the first component $\stateVariable_1$ of the state variable $\stateVariable= (\stateVariable_1, \stateVariable_2)$. More precisely, we define
\begin{equation}\label{eq.partition_of_state_space}
\begin{split}
 \stateSpace^{x_1} := & \left\lbrace y=(y_1,y_2) \in \stateSpace: \, \, y_1 = x_1\right\rbrace,
 \\
 \weakInflSpace := & \stateSpace^{o} \cup \stateSpace^{+},
 \qquad
 \weakDeflSpace := \stateSpace^{o} \cup \stateSpace^{-}.
\end{split}
\end{equation}
We refer to states $x$ in $\stateSpace^{+}$ (respectively, in $\stateSpace^{-}$) as inflationary (deflationary) states, and we refer to states $x$ in $\weakInflSpace$ (respectively, in $\weakDeflSpace$) as weakly inflationary (weakly deflationary) states.

The jump times for the mid-price consequently give rise to the counting processes 
\begin{equation}\label{eq.jump_times}
 \countingProc[]^{x_{1}}(t) := \sum_{n}\one\left\lbrace \arrivalTimes[]_n\leq t , \stateVariable_1(\arrivalTimes[]_n) = x_1 \right\rbrace
 =
 \sum_{e=0}^{4} \sum_{x\in\stateSpace^{x_1}} \hybridHawkes_{e,x}(t),
 \qquad x_1 \in \lbrace-1,0,+1\rbrace,
\end{equation}
where $\arrivalTimes[]_n$ is the $n$-th jump time of the ground process, and $\hybridHawkes_{e,x}(t)$ is the value at time $t$ of the $(e,x)$-th component of the hybrid MPP counterpart of $(\multiCountingProc,\stateVariable)$.
With this notation we can rewrite the integral quantity in equation \eqref{eq.midprice_proxy} as 
\begin{equation*}
 \intzerot \stateVariable_1 (s)d\groundProc(s)
 =
 \countingProc[]^{+}(t) - \countingProc[]^{-}(t)
%  \sum_{e=0}^{4}\left(
%  \sum_{x\in\stateSpace^{+}} \hybridHawkes_{e,x}(t)
%  - \sum_{x\in\stateSpace^{-}} \hybridHawkes_{e,x}(t)
%  \right),
\end{equation*}
where $\groundProc = \countingProc[0] + \dots + \countingProc[4]$ is the ground process of the five-dimensional component $\multiCountingProc$ of $(\multiCountingProc,\stateVariable)$.

Recall that in Bacry\&Muzy model, the price impact profile was defined as the expected difference between upward price jumps and downward price jumps. This definition hinged on the assumed symmetry between the two jumps (Assumptions \ref{assumption.BM14.symmetric_baseRates}, \ref{assumption.BM14.symmetric_cross-excitation} and \ref{assumption.BM14.symmetric_self-excitation}). To give a scenario-dependent version of their price impact profiles, we first translate this assumption into our state-dependent model, and we then define impact profiles ``\`a la Bacry-Muzy'' utilising compensators. 

\begin{defi}\label{def.price-simmetry}
 The state-dependent Hawkes model $\sdHawkesPair$ is said price-symmetric if 
 \begin{enumerate}[label={\roman{*}.} , ref={\ref{def.price-simmetry}.\roman{*}}]
 \item\label{def.price-simmetry.base_rates}
 for all $y$ in $\stateSpace$ it holds
 \begin{equation}\label{eq.price-simmetry.base_rates}
  \sum_{x\in\inflationarySpace}\sum_{e=1}^{4}\baseRate_{e}\transProb_{e}(y,x)
  =
  \sum_{x\in\deflationarySpace}\sum_{e=1}^{4}\baseRate_{e}\transProb_{e}(y,x);
 \end{equation}
\item\label{def.price-simmetry.kernels}
for all $\xone,y$ in $\stateSpace$, and  all $\eone = 1,\dots,4$, there exists a permutation $\sigma_{E}$ of $\lbrace 1,\dots,4\rbrace$ and a bijective map $\sigma_{\stateSpace} : \inflationarySpace\rightarrow\deflationarySpace$ such that for all $e=1,\dots,4$ and all $x$ in $\inflationarySpace$
\begin{enumerate}%[label={\roman{*}.}]
\item $\impCoef_{\eone,\xone,e}\transProb_{e}(y,x) = \impCoef_{\eone,\xone,\sigma_{E}(e)} \transProb_{\sigma_{E}(e)}(y,\sigma_{\stateSpace}(x))$;
\item $\decCoef_{\eone,\xone,e} = \decCoef_{\eone,\xone,\sigma_{E}(e)}$.
\end{enumerate}
\end{enumerate}
\end{defi}
The condition in Definition \ref{def.price-simmetry.base_rates}  captures the idea that, given the current state $y$, 
transitions to inflationary states induced by immigrants and transitions to deflationary states induced by immigrants are equally likely.
The conditions in Definition \ref{def.price-simmetry.kernels} capture the idea that, given the current state $y$, every event-state pair $(\eone,\xone)$ excites an  event-state pair $(e,x)$ with inflationary state $x\in\inflationarySpace$  the same way as it excites an event-state pair $(\sigma_{E}(e),\sigma_{\stateSpace}(x))$ with deflationary state $\sigma_{\stateSpace}(x) \in\deflationarySpace$; in other words, offsprings from every event-state pair $(\eone,\xone)$ are equally likely to be associated with inflationary states or with deflationary states. 

Notice that the condition in Definition \ref{def.price-simmetry.base_rates} could be made stronger by requiring the existence of a permutation $\sigma_E$ of $\lbrace 1,\dots,4\rbrace$ and a bijective map $\sigma_{\stateSpace} : \inflationarySpace\rightarrow\deflationarySpace$ such that for all $e=1,\dots,4$ and all $x$ in $\inflationarySpace$ it holds $\baseRate_{e}\transProb_{e}(y,x) = \baseRate_{\sigma_{e}(e)}\transProb_{\sigma_{e}(e)}(y,\sigma_{\stateSpace}(x))$. The latter, which  implies equation \eqref{eq.price-simmetry.base_rates}, would align the requirement on the base rates more closely to the requirement on the Hawkes kernels. However, the weaker formulation was preferred because it suffices for our purposes. 

\begin{lemma}\label{lemma.price-simmetry}
 If $\sdHawkesPair$ is price-symmetric, then for all $t\geq 0$ it holds
 \begin{equation*}
  \sum_{x\in\inflationarySpace}\sum_{e=1}^{4}
  \transProb_{e}(\stateVariable(t),x)\ell_{e}(t)
  =
  \sum_{x\in\deflationarySpace}\sum_{e=1}^{4}
  \transProb_{e}(\stateVariable(t),x)\ell_{e}(t),
 \end{equation*}
where
\begin{equation*}
 \ell_{e}(t) = \baseRate_{e}
 +
 \sum_{\eone=1}^{4}\int_{[0,t)} \hawkesKernel\subscriptee(t-s,\stateVariable(s)d\countingProc[\eone](s).
\end{equation*}
\end{lemma}
\begin{proof}
 Let $(\arrivalTimes[]_n,\event[n], \stateVariable_n)$ be the sequence of arrival times, events and states. Then, we can write
 \begin{equation*}
  \begin{split}
   \sum_{x\in\inflationarySpace}\sum_{e=1}^{4}
   \transProb_{e}(\stateVariable(t),& x)
   \sum_{\eone=1}^{4}\int_{[0,t)}\hawkesKernel\subscriptee(t-s,\stateVariable(s))d\countingProc[\eone](s)
   \\
   =&
   \sum_{n:\arrivalTimes[]_n <t}
   \sum_{x\in\inflationarySpace}\sum_{e=1}^{4}
   \transProb_{e}(\stateVariable(t),x)
   \hawkesKernel_{\event[n],e}(t-\arrivalTimes[]_n,\stateVariable_n)
   \\
   =&
   \sum_{n:\arrivalTimes[]_n <t}
   \sum_{x\in\inflationarySpace}\sum_{e=1}^{4}
   \transProb_{\sigma_{E}(e)}(\stateVariable(t),\sigma_{\stateSpace}(x))
   \hawkesKernel_{\event[n],\sigma_{E}(e)}(t-\arrivalTimes[]_n,\stateVariable_n)
   \\
   =&
   \sum_{n:\arrivalTimes[]_n <t}
   \sum_{x\in\deflationarySpace}\sum_{e=1}^{4}
   \transProb_{e}(\stateVariable(t),x)
   \hawkesKernel_{\event[n],e}(t-\arrivalTimes[]_n,\stateVariable_n)
   \\   
   =&
   \sum_{x\in\deflationarySpace}\sum_{e=1}^{4}
   \transProb_{e}(\stateVariable(t), x)
   \sum_{\eone=1}^{4}\int_{[0,t)}\hawkesKernel\subscriptee(t-s,\stateVariable(s))d\countingProc[\eone](s).
   \end{split}
 \end{equation*}
\end{proof}

 In the absence of the liquidator, i.e. when $\multiCountingProc$ has only the four components $\countingProc[1], \dots, \countingProc[4]$ introduced in Section \ref{sec.sdHawkes_model}, the statement of Lemma \ref{lemma.price-simmetry} is rephrased as 
 \begin{equation*}
  \sum_{x\in\inflationarySpace}\sum_{e=1}^{4}\hybridIntensity_{e,x}(t)
  =
  \sum_{x\in\deflationarySpace}\sum_{e=1}^{4}\hybridIntensity_{e,x}(t),
 \end{equation*}
where $\hybridIntensity_{e,x}(t)$ is the intensity of the $(e,x)$-th component $\hybridHawkes_{e,x}$ of the hybrid MPP-counterpart of $\sdHawkesPair$. Instead, when the liquidator is present, the intensities $\hybridIntensity_{e,x}(t)$ also incorporate the effect  induced by the jumps of $\countingProc[0]$ and thus 
\begin{equation*}
 \hybridIntensity_{e,x}(t) = \transProb_e(\stateVariable(t),x)
 \left(\ell_{e}(t) + \int_{[0,t)}\underbrace{\hawkesKernel_{0,e}}_{=\hawkesKernel_{1,e}}(t-s,\stateVariable(s))d\countingProc[0](s)
 \right),
\end{equation*}
for all $e=1,\dots,4$ and all $x$ in $\stateSpace$. 
In other words, the symmetry is assumed to exist when the liquidator is not present in the market: it is a measure of the disruption of this symmetry that will account for her impact.

\begin{defi}\label{def.BM_impact}
 Let $\sdHawkesPair$ be a price-symmetric state-dependent Hawkes model with the zero-th component $\countingProc[0]$ of the five-dimensional counting process $\multiCountingProc$ representing a liquidator's execution schedule. Let $\internalHistory$ be the internal history of $\sdHawkesPair$. Then, the BM-price impact profile of $\countingProc[0]$ is the $\internalHistory$-compensator of $\Nminus - \Nplus$. 
\end{defi}

\begin{remark}
\begin{enumerate}[label=\roman{*}.]
 \item 
 The choice of sign in Definition \ref{def.BM_impact} is such that the liquidator's impact is positive when the price plunges as a consequence of her interventions. In \citealp{BM14haw}, instead, the price impact profile was defined as the expected difference between upward price jumps and downward price jumps; whence liquidator's interventions that cause the price to plunge would have negative impact profile with their choice. 
 \item 
 Compensators are usually defined for non-decreasing processes. However, since $\countingProc[]^{-} - \countingProc[]^{+}$ is of bounded variation, its compensator is unambiguously defined as the difference $\compensator^{-}-\compensator^{+}$, where $\compensator^{-}$ is the compensator of $\Nminus$ and $\compensator^{+}$ is the compensator of $\Nplus$.
\end{enumerate}
\end{remark}

\begin{prop}\label{prop.bm_impact}
 Let $\sdHawkesPair$ be price-symmetric. Let Assumption \ref{assumption.imp_and_dec_coef_from_liquidator} be in place, and additionally assume that $\hawkesKernel_{0,0} \equiv 0$. Then, the BM-price impact profile in $\sdHawkesPair$ has density
 \begin{equation}\label{eq.intensity_of_BM-impact}
 \Big(\intensityDeflationary - \intensityInflationary\Big)(t)
  =\directImpact(t) + \indirectImpact(t),
  \end{equation}
  where 
  \begin{equation*}
   \directImpact(t)=  
  \sum_{x\in\deflationarySpace}\transProb_{0} (\stateVariable(t),x)
  \left(\baseRate_{0}  
  + \sum_{\eone=1}^{4}\sum_{\xone=1}^{3K} \int_{[0,t)} \hawkesKernel_{\eone,0}(t-s,\xone)d\hybridHawkes_{0,\xone}
  \right)
  \one_{[0,\terminationTime)} (t),
  \end{equation*}
 where $\terminationTime$ is the termination time of the liquidation, and
 \begin{equation*}
  \indirectImpact(t)=
  \sum_{e=1}^{4}\sum_{\xone=1}^{3K} \int_{[0,t)} \hawkesKernel_{1,e}(t-s,\xone)d\hybridHawkes_{0,\xone}
  \left(
  \sum_{x\in\deflationarySpace}
  -
  \sum_{x\in\inflationarySpace}
  \right) \transProb_{e}(\stateVariable(t),x)
 \end{equation*}
\end{prop}
\begin{proof}
 Equation \eqref{eq.intensity_of_BM-impact} follows from the computation of $\intensityDeflationary$ (respectively, of $\intensityInflationary$) as the sum of the intensities of $\hybridHawkes_{e,x}$ for $e = 0,\dots,4$ and $x$ in $\deflationarySpace$ (in $\inflationarySpace$, respectively). Indeed, from equations \eqref{eq.intensity_of_hybridMPP} and \eqref{eq.intensities_when_liquidator_is_present} it follows that 
 \begin{equation*}
 \begin{split}
  \intensityDeflationary(t) 
  =&
  \sum_{x\in\deflationarySpace}\Bigg\lbrace
  \transProb_{0}(\stateVariable(t),x) \intensity[0](t)
  \\
  &  +
  \sum_{e=1}^{4} \transProb_{e}(\stateVariable(t),x) 
  \left(
  \underbrace{
  \baseRate_e 
 + \sum_{\eone = 1}^{4}\sum_{\xone=1}^{3K}\int_{[0,t)} \hawkesKernel_{\eone,e}(t-s,\xone)d\hybridHawkes_{\eone,\xone}(s)
 }_{=\ell_{e}(t)}
 + \int_{[0,t)} \hawkesKernel_{1,e}(t-s,\xone)d\hybridHawkes_{0,\xone}(s)
 \right)
  \Bigg\rbrace,
  \\
  \end{split}
 \end{equation*}
 where $\intensity[0](t)$ is as in \eqref{eq.intensity_of_liquidator}, and 
 \begin{equation*}
 \begin{split}
  \intensityInflationary(t) 
  =&
  \sum_{x\in\inflationarySpace}
  \sum_{e=1}^{4} \transProb_{e}(\stateVariable(t),x) 
  \left(
  \underbrace{
  \baseRate_e 
 + \sum_{\eone = 1}^{4}\sum_{\xone=1}^{3K}\int_{[0,t)} \hawkesKernel_{\eone,e}(t-s,\xone)d\hybridHawkes_{\eone,\xone}(s)
 }_{=\ell_{e}(t)}
 + \int_{[0,t)} \hawkesKernel_{1,e}(t-s,\xone)d\hybridHawkes_{0,\xone}(s)
 \right).
   \end{split}
 \end{equation*}
 By Lemma \ref{lemma.price-simmetry}, the terms $\sum_{x\in\deflationarySpace}\sum_{e=1}^{4}\transProb_{e}(\stateVariable(t),x) \ell_{e}(t)$ and $\sum_{x\in\inflationarySpace}\sum_{e=1}^{4}\transProb_{e}(\stateVariable(t),x) \ell_{e}(t)$ will  cancel out from the difference $\intensityDeflationary(t) - \intensityInflationary(t)$. 
 \end{proof}
 
 Equation \eqref{eq.intensity_of_BM-impact} decomposes the intensity of the BM-price impact profile in two components, namely $\directImpact(t)$ and $\indirectImpact(t)$. Both are null if $\countingProc[0] (t) \equiv 0$. The former is referred to as ``direct'' impact and stems from those summands of the execution schedule's intensity $\intensity[0](t) = \sum_{x=1}^{3K}\hybridIntensity_{0,x}(t)$ that are associated with deflationary states, namely $\directImpact(t) = \sum_{x\in\deflationarySpace} \hybridIntensity_{0,x}(t)$. Notice that $\directImpact(t)\geq 0$ for all $t>0$ and  $\directImpact(t) = 0$ for all $t>\terminationTime$. On the contrary, the second term $\indirectImpact(t)$ stems from events originated by participants other than the liquidator but in response to the liquidator's interventions, whence the name of ``indirect'' impact. It can have either sign and it is in general non-zero even beyond the termination time; for this reason it is linked to the transient impact. More precisely, for $t>\terminationTime$ it holds $\indirectImpact(t) = \sum_{e=1}^{4}\sum_{\xone} \sum_{j} \hawkesKernel_{1,e}(t-\arrivalTimes[0]_j,\stateVariable(\arrivalTimes[0]_j)) (\sum_{x\in \deflationarySpace} - \sum_{x\in \inflationarySpace})\transProb_{e}(\stateVariable(t),x)$, and  the transient price impact profile is the map $t\mapsto \intzerot \indirectImpact(s)ds$, restricted to the interval $t\geq \terminationTime$, which decays in time as $t^{1-\decCoef}$, where $\decCoef := \min\lbrace \decCoef_{1,\xone,e}: \, \, e=1,\dots,4 \text{ and } \stateVariable(\arrivalTimes[0]_j) = \xone \text{ for some }j\rbrace > 1$. Finally, the BM-price impact of the liquidator is permanent in the market if $\compensator^{-}(t) - \compensator^{+}(t) \nrightarrow 0$ as $t\rightarrow\infty$. 
 
 The direct impact component $\directImpact(t)$ of the intensity $(\intensityDeflationary - \intensityInflationary)(t)$ encompasses the transition matrix $\transProb_0$ associated with the state update that occurs when liquidator's orders are executed. Recall that for $x$ and $y$ in $\stateSpace$, $\transProb_{0}(x,y)$ is estimated according to equation \eqref{eq.transProb_liquidator}; hence it summarises the state transitions that stem from Algorithm \ref{algo.lob_update} during the simulation of the execution. This disentangles the effects of liquidator's orders (whose sizes $q_{M,j}$ are set by the liquidator) from the effects of other market orders, i.e. $\transProb_0 \neq \transProb_1$ in general, allowing to investigate the impact of different execution strategies.
 
 In particular, the liquidator might choose to send market orders with sizes that never exceed the available liquidity on the first bid level; this would cause $\transProb_0(y,x)=0$ for all $y$ in $\stateSpace$ and all deflationary $x$ in $\deflationarySpace$, and thus $\directImpact(t)\equiv 0$. Nonetheless, the overall impact would not be null, because of the indirect term $\indirectImpact(t)$. Indeed, even without ever walking the book, the liquidator's orders would modify the volumes in the orderbook, pushing the state trajectory $t\mapsto \stateVariable(t)$ to dwell on states $\lbrace y=(y_1,y_2)\in\stateSpace: \, \, y_2<0\rbrace$ for longer. The probability of transitioning from these states to deflationary states is higher than the probability of transitioning to inflationary states, hence making the term $(\sum_{x\in\deflationarySpace} - \sum_{x\in\inflationarySpace})\transProb_{e}(\stateVariable(t), x)$ positive, and contributing to the impact via the indirect term $\indirectImpact(t)$.
 Notice that this form of impact would not be captured by the less granular model presented in Section \ref{sec.BM14_model}. This is because, in the ``impulsive impact kernel'', in order to account for orders that never walk the book one would have to set $\theta_3(t)\equiv0$ in equation \eqref{eq.BM14_intensity_with_liquidator}, and this would annihilate the price impact altogether.  It is instead with a more granular description of the interaction between market orders and volumes that our state-dependent model can capture the impact of market orders that do not walk the book.   

\subsection{One-sided impact profiles}\label{sec.one-side_impact}
In this section we look at a formulation of the price impact profile that does not presuppose any symmetry in the price. This is done by focusing on the information content of the filtration generated by the execution schedule $\countingProc[0]$ and its relevance in explaining price movements. In this sense, the current approach is inspired by filtering. 

The price impact formulated in Section \ref{sec.impact_a_la_BM} (in Section \ref{sec.BM14_price_impact_profiles}) rested on the assumption that, once volatility is factored out by looking at compensators (by taking expectation, respectively), in the absence of the liquidator the increase of $\Nminus$ (of $\countingProc[3]$) is cancelled by the increase of $\Nplus$ (of $\countingProc[4]$, respectively). This makes sense when one wishes to extrapolate general properties of price impact, because on average across several days and several markets (i.e. overall) this cancellation is indeed expected to hold.

Measuring the price impact in a particular market scenario, instead, has to take into account the specific market trend during the liquidation. In most cases, calibrating a Hawkes model to a given observation of a limit order book reveals that price increases and price decreases were not symmetric, showing either a bullish or a bearish market. We therefore formulate a notion of price impact that does not rest on any assumption of symmetry. 

Liquidator's interventions will impact on event-state pairs $(e,x)$ of the hybrid MPP-counterpart $\hybridHawkes$ of $\sdHawkesPair$ with weakly deflationary state $x$ in $\weakDeflSpace$, making them more likely. Instead, they cannot possibly increase $\Nplus$. Hence, we focus on the counting process 
\begin{equation}\label{eq.weakly_deflationary_coutning_process}
 \weakDeflPP (t) : = \sum_{n}\one\left\lbrace
 \arrivalTimes[]_n \leq t, \, \, \stateVariable_n \in \weakDeflSpace
 \right\rbrace,
\end{equation}
where $\arrivalTimes[]_n$ is the $n$-th jump of the ground process $\groundProc$ of $\multiCountingProc$ and $\stateVariable_n = \stateVariable(\arrivalTimes[]_n)$, and we seek to measure how much of its tendency to increase is due to the liquidator's interventions. We will refer to $\weakDeflPP$ as the weakly deflationary counting process in the state-dependent model $\sdHawkesPair$.

Recall that in a probability space $(\Omega,\sigmaAlgebra,\mu)$ two sigma algebrae $\setAlgebraA, \setAlgebraB < \sigmaAlgebra$ are orthogonal if $\mu(A\cap B)=\mu(A)\mu(B)$ for all $A$ in $\setAlgebraA$ and all $B$ in $\setAlgebraB$. Any $\setAlgebraA < \sigmaAlgebra$ is always orthogonal to the trivial sigma-algebra $\setAlgebraB = \lbrace \emptyset, \Omega\rbrace$. Hence, given $\sigmaAlgebraG < \sigmaAlgebra$, we can define
$\sigmaAlgebraG\orthogonal$ as the largest sub-sigma algebra of $\sigmaAlgebra$ that is orthogonal to $\sigmaAlgebraG$. Then, given the probability space $\probabilitySpace$ and the sub-sigma algebra $\sigmaAlgebraG < \sigmaAlgebra$, every centred square-integrable random variable $X$ admits the $\Ltwo$-orthogonal decomposition $X = \Expectation[X\vert \sigmaAlgebraG] + \Expectation[X\vert\sigmaAlgebraG\orthogonal]$.
If $X$ is not centred, then we can write $X = \Expectation[X\vert \sigmaAlgebraG] + \Expectation[X\vert\sigmaAlgebraG\orthogonal] - \Expectation[X]$.
From an information-theoretic point of view, $\Expectation[X\vert\sigmaAlgebraG]$ is the best $\Ltwo$-predictor of $X$ given the information in $\sigmaAlgebraG$, and $\Expectation[X\vert\sigmaAlgebraG\orthogonal]$ is what of $X$ cannot be explained by using information in $\sigmaAlgebraG$. The following definition is motivated by this point of view. 

\begin{defi}\label{def.one-sided_impact}
 In the state-dependent Hawkes model $\sdHawkesPair$, let the zero-th component $\countingProc[0]$ of $\multiCountingProc$ represent the liquidator's execution schedule. Let $\internalHistory$ be the internal history of $\sdHawkesPair$, and let $\weakDeflCompensator$ be the $\internalHistory$-compensator of $\weakDeflPP$, with intensity $\weakDeflIntensity$. Let $\liquidatorSigmaAlgebra$ be the filtration generated by $\countingProc[0]$, namely $\liquidatorSigmaAlgebra(t) = \sigma(\countingProc[0](s): \, s\leq t)$. Then, we define the one-sided impact profile of  $\countingProc[0]$ as the map $t\mapsto \impactProfile(t)$, where
 \begin{equation}\label{eq.one-sided_impact}
  \impactProfile(t):=
  \left(\weakDeflCompensator(t) 
    - \intzerot\Expectation\left[\weakDeflIntensity(s)\vert \liquidatorSigmaAlgebra\orthogonal(s)\right]ds \right)\squared
    \Bigg/
      \left(\weakDeflCompensator(t) 
    - \intzerot\Expectation\left[\weakDeflIntensity(s)\vert \liquidatorSigmaAlgebra(s)\right]ds \right)\squared, 
 \end{equation}
 where, for every $t$,  $\liquidatorSigmaAlgebra\orthogonal(t)$ is the orthogonal sigma-algebra of $\liquidatorSigmaAlgebra(t)$ in $\internalHistory(t)$.
\end{defi}

In the absence of liquidator, $\countingProc[0] \equiv 0$ and thus $\sigmaAlgebraG_0(t)$ is trivial for all $t$. Therefore in this case, $\sigmaAlgebraG_{0}\orthogonal(t) = \internalHistory(t)$ for all $t$ and $\impactProfile \equiv 0$. 
At the opposite extreme, if we were to imagine a market where all participants except the liquidator make only actions that increase the mid-price (bid limit orders posted inside the spread, cancellation of first-level ask limit orders, buy market orders walking the book), then $\weakDeflIntensity$ would be adapted to $\sigmaAlgebraG_0$ and $\impactProfile(t)=\infty$ for all $t$. Hence, the impact profile defined in Definition \ref{def.one-sided_impact} is a non-negative unitless measure of the liquidator's price impact that ranges from $0$ (when the liquidator is not active in the market) to $\infty$ (when she is the only market participant willing to sell). 

We will perform our measurement of price impact based on equation \eqref{eq.one-sided_impact} in the special case of a Poissonian liquidation with constant order size (Definitions \ref{def.poissonina_liquidation} and \ref{def.liquidation_with_constant_size}). This is because the numerator of $\impactProfile(t)$ becomes explicit, as we outline now. The actual implementation of this formula will be given in Lemma \ref{lemma.implementation_of_one-sided_impact}.

\begin{prop}\label{prop.impact_poisson_and_constant-size}
 Under Assumption \ref{assumption.termination-time_of_liquidation}, in a Poissonian liquidation with constant order size 
 it holds
 \begin{equation}\label{eq.explicit_numerator_of_impProfile}
 \begin{split}
  \weakDeflCompensator(t) 
    -& \intzerot\Expectation\left[\weakDeflIntensity(s)\vert \liquidatorSigmaAlgebra\orthogonal(s)\right]ds
\\
=&
\baseRate_{0}\intzerot \left(\one_{[0,\terminationTime)} (s) - \Prob(\terminationTime>s)\right)ds
+
\sum_{x\in\weakDeflSpace} \sum_{e=1}^{4}
\intzerot ds \, \, \transProb_{e}(\stateVariable(s),x)\Kappa_{1,e}(s),
\end{split}
 \end{equation}
 where 
 \begin{equation*}
  \Kappa_{1,e}(s) :=
\sum_{\xone=1}^{3K}\int_{[0,s)}\hawkesKernel_{1,e}(s-u, \xone)
\Big\lbrace
d\hybridHawkes_{0,\xone}(u)
- \baseRate_{0}\transProb_{0}(\stateVariable(u),\xone)\Prob(\terminationTime>u) du
\Big\rbrace.
 \end{equation*}
\end{prop}
\begin{proof}
 Since $\sum_{x\in\weakDeflSpace}\transProb_{0}(y,x) = 1$, we have that 
 \begin{equation*}
  \weakDeflIntensity(t)
  =
  \baseRate_{0} \one_{[0,\terminationTime)} (t)
  + \sum_{x\in\weakDeflSpace} \sum_{e=1}^{4}
  \transProb_e(\stateVariable(t),x)\sum_{\xone} \int_{[0,t)} \hawkesKernel_{1,e}(t-s, \xone) d\hybridHawkes_{0,\xone}(s)
  + R_{t}^{\sigmaAlgebraG_{0}\orthogonal},
 \end{equation*}
where the remainer $R_{t}^{\sigmaAlgebraG_{0}\orthogonal}$ is $\sigmaAlgebraG_{0}\orthogonal(t)$-measurable and given by
\begin{equation*}
 R_{t}^{\sigmaAlgebraG_{0}\orthogonal}
 =
 \sum_{x\in\weakDeflSpace}\sum_{e=1}^{4}
 \transProb_{e}(\stateVariable(t),x) \ell_{e}(t),
\end{equation*}
where $\ell_e(t)$ is as in Lemma \ref{lemma.price-simmetry}. Therefore, 
\begin{equation*}
 \begin{split}
  \Expectation[\weakDeflIntensity(t)\,\vert\, \sigmaAlgebraG_{0}\orthogonal(t) ]
  =& 
  \baseRate_{0} \Prob(\terminationTime > t)
  \\
  &+
  \sum_{x\in\weakDeflSpace} \sum_{e=1}^{4}
  \transProb_e(\stateVariable(t),x)\sum_{\xone} \int_{[0,t)} \hawkesKernel_{1,e}(t-s, \xone) \transProb_{0}(\stateVariable(s),\xone) \baseRate_{0}\Prob(\terminationTime>s) ds
  \\
  &+ R^{\sigmaAlgebraG_{0}\orthogonal}_{t}.
 \end{split}
\end{equation*}
Integrating in time gives equation \eqref{eq.explicit_numerator_of_impProfile}.
\end{proof}




\section{The python module \texttt{pipest}}\label{sec.the_python_module}
\subsection{File system}
\begin{center}
\begin{table}
\begin{tabular}{|c|c|}
 \toprule
   \includegraphics[width=.6\textwidth]{hierG.png}
   &
   \includegraphics[width=.3\textwidth]{callG.png}
   \\
   \midrule
 \multicolumn{2}{|c|}{
 \begin{scriptsize}
 {\rowcolors{5}{gray!20!white!50}{gray!10!white!40}
 \begin{tabular}{rrllrl}
 \\
 \hline
 \\ \textbf{id} &  \textbf{depth} & \textbf{file} & \textbf{extension} & \textbf{parent} & \textbf{path}
 \\
 \hline
 \\
 0 &      1 &                pipest/ &      None &      -1 &                                            ./
 \\
 1 &      2 &              sdhawkes/ &      None &       0 &                                   ./sdhawkes/ 
 \\
 2 &      3 &             modelling/ &      None &       1 &                         ./sdhawkes/modelling/
 \\
 3 &      3 &             lob\_model &       pyx &       2 &            ./sdhawkes/modelling/lob\_model.pyx
 \\
 4 &      3 &                 model &       pyx &       2 &                ./sdhawkes/modelling/model.pyx
 \\
 5 &      3 &             resources/ &      None &       1 &                         ./sdhawkes/resources/
 \\
 6 &      3 &           computation &       pyx &       5 &          ./sdhawkes/resources/computation.pyx
 \\
 7 &      3 &        impact\_profile &       pyx &       5 &       ./sdhawkes/resources/impact\_profile.pyx
 \\
 8 &      3 &            plot\_tools &        py &       5 &            ./sdhawkes/resources/plot\_tools.py \\  9 &      3 &        mle\_estimation &       pyx &       5 &       ./sdhawkes/resources/mle\_estimation.pyx \\ 10 &      3 &     uncertainty\_quant &       pyx &       5 &    ./sdhawkes/resources/uncertainty\_quant.pyx \\ 11 &      3 &     minimisation\_algo &       pyx &       5 &    ./sdhawkes/resources/minimisation\_algo.pyx \\ 12 &      3 &            simulation &       pyx &       5 &           ./sdhawkes/resources/simulation.pyx \\ 13 &      3 &       goodness\_of\_fit &       pyx &       5 &      ./sdhawkes/resources/goodness\_of\_fit.pyx \\ 14 &      3 &   nonparam\_estimation &       pyx &       5 &  ./sdhawkes/resources/nonparam\_estimation.pyx \\ 15 &      2 &               lobster/ &      None &       0 &                                    ./lobster/ \\ 16 &      3 &            py\_scripts/ &      None &      15 &                         ./lobster/py\_scripts/ \\ 17 &      3 &     read\_from\_lobster &        py &      16 &     ./lobster/py\_scripts/read\_from\_lobster.py \\ 18 &      3 &  prepare\_from\_lobster &       pyx &      16 &  ./lobster/py\_scripts/prepare\_from\_lobster.pyx \\ 19 &      3 &                  data/ &      None &      15 &                               ./lobster/data/ \\ 20 &      2 &           calibration/ &      None &       0 &                                ./calibration/ \\ 21 &      2 &                  main &        py &      20 &                         ./calibration/main.py  \\
 \hline
 \end{tabular}
 }
 \end{scriptsize}
 }
 \\
 \bottomrule
\end{tabular}
\label{tab.filesystem}
\caption{Core file system represented as hierarchical network (top left panel), network of function calls (top right panel), and dataframe (bottom panel).}
\end{table}
\end{center}

\subsection{Initialisation and a first simulation}
   \begin{tcolorbox}[breakable, size=fbox, boxrule=1pt, pad at break*=1mm,colback=cellbackground, colframe=cellborder]
\prompt{In}{incolor}{2}{\boxspacing}
\begin{Verbatim}[commandchars=\\\{\}]
\PY{n}{n\PYZus{}states}\PY{o}{=}\PY{p}{[}\PY{l+m+mi}{3}\PY{p}{,}\PY{l+m+mi}{5}\PY{p}{]} 
\PY{n}{n\PYZus{}events} \PY{o}{=} \PY{l+m+mi}{4}  \PY{c+c1}{\PYZsh{} number of event types, \PYZdl{}d\PYZus{}E\PYZdl{}}
\PY{n}{n\PYZus{}levels} \PY{o}{=} \PY{l+m+mi}{2}
\PY{n}{upto\PYZus{}level} \PY{o}{=} \PY{l+m+mi}{2}
\PY{n}{time\PYZus{}start}\PY{o}{=}\PY{l+m+mf}{0.0}
\PY{n}{time\PYZus{}end}\PY{o}{=}\PY{n}{time\PYZus{}start}\PY{o}{+}\PY{l+m+mf}{1.0}\PY{o}{*}\PY{l+m+mi}{60} \PY{c+c1}{\PYZsh{} one minute}
\PY{n}{model} \PY{o}{=} \PY{n}{sd\PYZus{}hawkes\PYZus{}model}\PY{o}{.}\PY{n}{SDHawkes}\PY{p}{(}
    \PY{n}{number\PYZus{}of\PYZus{}event\PYZus{}types}\PY{o}{=}\PY{n}{n\PYZus{}events}\PY{p}{,} \PY{n}{list\PYZus{}of\PYZus{}n\PYZus{}states}\PY{o}{=}\PY{n}{n\PYZus{}states}\PY{p}{,}
    \PY{n}{number\PYZus{}of\PYZus{}lob\PYZus{}levels}\PY{o}{=}\PY{n}{n\PYZus{}levels}\PY{p}{,} \PY{n}{volume\PYZus{}imbalance\PYZus{}upto\PYZus{}level}\PY{o}{=}\PY{n}{upto\PYZus{}level}\PY{p}{)}
\end{Verbatim}
\end{tcolorbox}
\begin{tcolorbox}[breakable, size=fbox, boxrule=1pt, pad at break*=1mm,colback=cellbackground, colframe=cellborder]
\prompt{In}{incolor}{3}{\boxspacing}
\begin{Verbatim}[commandchars=\\\{\}]
\PY{n}{d\PYZus{}E}\PY{o}{=}\PY{n}{model}\PY{o}{.}\PY{n}{number\PYZus{}of\PYZus{}event\PYZus{}types}
\PY{n}{d\PYZus{}S}\PY{o}{=}\PY{n}{model}\PY{o}{.}\PY{n}{number\PYZus{}of\PYZus{}states}
\PY{n+nb}{print}\PY{p}{(}\PY{l+s+s2}{\PYZdq{}}\PY{l+s+s2}{d\PYZus{}E=}\PY{l+s+si}{\PYZob{}\PYZcb{}}\PY{l+s+s2}{; d\PYZus{}S=}\PY{l+s+si}{\PYZob{}\PYZcb{}}\PY{l+s+s2}{\PYZdq{}}\PY{o}{.}\PY{n}{format}\PY{p}{(}\PY{n}{d\PYZus{}E}\PY{p}{,}\PY{n}{d\PYZus{}S}\PY{p}{)}\PY{p}{)}
\end{Verbatim}
\end{tcolorbox}
\begin{tcolorbox}[breakable, size=fbox, boxrule=.5pt, pad at break*=1mm, opacityfill=0]
\prompt{Out}{outcolor}{3}{\boxspacing}
\begin{Verbatim}[commandchars=\\\{\}]
d\_E=4; d\_S=15
\end{Verbatim}
\end{tcolorbox}
\begin{tcolorbox}[breakable, size=fbox, boxrule=1pt, pad at break*=1mm,colback=cellbackground, colframe=cellborder]
\prompt{In}{incolor}{4}{\boxspacing}
\begin{Verbatim}[commandchars=\\\{\}]
\PY{n}{model}\PY{o}{.}\PY{n}{volume\PYZus{}enc}\PY{o}{.}\PY{n}{volimb\PYZus{}limits}
\end{Verbatim}
\end{tcolorbox}
\begin{tcolorbox}[breakable, size=fbox, boxrule=.5pt, pad at break*=1mm, opacityfill=0]
\prompt{Out}{outcolor}{4}{\boxspacing}
\begin{Verbatim}[commandchars=\\\{\}]
array([-1. , -0.6, -0.2,  0.2,  0.6,  1. ])
\end{Verbatim}
\end{tcolorbox}
\begin{tcolorbox}[breakable, size=fbox, boxrule=1pt, pad at break*=1mm,colback=cellbackground, colframe=cellborder]
\prompt{In}{incolor}{5}{\boxspacing}
\begin{Verbatim}[commandchars=\\\{\}]
\PY{n}{model}\PY{o}{.}\PY{n}{state\PYZus{}enc}\PY{o}{.}\PY{n}{df\PYZus{}state\PYZus{}enc}
\end{Verbatim}
\end{tcolorbox}
\begin{tcolorbox}[breakable, size=fbox, boxrule=.5pt, pad at break*=1mm, opacityfill=0]
\prompt{Out}{outcolor}{5}{\boxspacing}
\end{tcolorbox}
\begin{center}
\begin{scriptsize}
{\rowcolors{3}{gray!20!white!50}{gray!10!white!40}
\begin{tabular}{lrrrr}
 \\
 \toprule{} 
 &  \textbf{st\_1} &  \textbf{st\_2} &  \textbf{one\_dim\_label} & \textbf{multidim\_label} 
 \\
 \midrule
 \textbf{0}  &     0 &     0 &  0 &         (0, 0) 
 \\
 \textbf{1}  &     0 &     1 &    1 &     (0, 1)
 \\
 \textbf{2}  &     0 &     2 &    2 &         (0, 2) 
 \\
 \textbf{3}  &     0 &     3 &    3 &         (0, 3) 
 \\
 \textbf{4}  &     0 &     4 &    4 &         (0, 4) 
 \\
 \textbf{5}  &     1 &     0 &    5 &         (1, 0)
 \\
 \textbf{6}  &     1 &     1 &    6 &         (1, 1) 
 \\ 
 \textbf{7}  &     1 &     2 &    7 &         (1, 2) 
 \\
 \textbf{8}  &     1 &     3 &    8 &         (1, 3)
 \\
 \textbf{9}  &     1 &     4 &    9 &         (1, 4) 
 \\
 \textbf{10} &     2 &     0 &   10 &         (2, 0) 
 \\
 \textbf{11} &     2 &     1 &   11 &         (2, 1) 
 \\
 \textbf{12} &     2 &     2 &   12 &         (2, 2) 
 \\
 \textbf{13} &     2 &     3 &   13 &         (2, 3) 
 \\
 \textbf{14} &     2 &     4 &   14 &         (2, 4) 
 \\
 \bottomrule
 \end{tabular}
 }
 \end{scriptsize}
 \end{center}
% 
% \begin{Verbatim}[commandchars=\\\{\}]
%     st\_1  st\_2  one\_dim\_label multidim\_label
% 0      0     0              0         (0, 0)
% 1      0     1              1         (0, 1)
% 2      0     2              2         (0, 2)
% 3      0     3              3         (0, 3)
% 4      0     4              4         (0, 4)
% 5      1     0              5         (1, 0)
% 6      1     1              6         (1, 1)
% 7      1     2              7         (1, 2)
% 8      1     3              8         (1, 3)
% 9      1     4              9         (1, 4)
% 10     2     0             10         (2, 0)
% 11     2     1             11         (2, 1)
% 12     2     2             12         (2, 2)
% 13     2     3             13         (2, 3)
% 14     2     4             14         (2, 4)
% \end{Verbatim}
% \end{tcolorbox}
        
    Assign arbitrary values to the Hawkes parameters \(\nu_{e}\),
\(\alpha_{e',x',e}\) and \(\beta_{e',x',e}\)
\begin{tcolorbox}[breakable, size=fbox, boxrule=1pt, pad at break*=1mm,colback=cellbackground, colframe=cellborder]
\prompt{In}{incolor}{6}{\boxspacing}
\begin{Verbatim}[commandchars=\\\{\}]
\PY{n}{nus} \PY{o}{=} \PY{n}{np}\PY{o}{.}\PY{n}{array}\PY{p}{(}\PY{p}{[}\PY{l+m+mf}{0.0}\PY{p}{,}\PY{l+m+mf}{0.0}\PY{p}{,}\PY{l+m+mf}{1.0}\PY{p}{,}\PY{l+m+mf}{1.0}\PY{p}{]}\PY{p}{,}\PY{n}{dtype}\PY{o}{=}\PY{n}{np}\PY{o}{.}\PY{n}{float}\PY{p}{)}
\PY{n}{alphas} \PY{o}{=} \PY{n}{np}\PY{o}{.}\PY{n}{maximum}\PY{p}{(}
    \PY{n}{np}\PY{o}{.}\PY{n}{concatenate}\PY{p}{(}
        \PY{p}{[}\PY{l+m+mf}{0.2}\PY{o}{*}\PY{n}{np}\PY{o}{.}\PY{n}{ones}\PY{p}{(}\PY{p}{(}\PY{n}{d\PYZus{}E}\PY{p}{,}\PY{n}{d\PYZus{}S}\PY{p}{,}\PY{n}{d\PYZus{}E}\PY{o}{/}\PY{o}{/}\PY{l+m+mi}{2}\PY{p}{)}\PY{p}{,}\PY{n}{dtype}\PY{o}{=}\PY{n}{np}\PY{o}{.}\PY{n}{float}\PY{p}{)}\PY{p}{,}\PY{n}{np}\PY{o}{.}\PY{n}{zeros}\PY{p}{(}\PY{p}{(}\PY{n}{d\PYZus{}E}\PY{p}{,}\PY{n}{d\PYZus{}S}\PY{p}{,}\PY{n}{d\PYZus{}E}\PY{o}{\PYZhy{}}\PY{n}{d\PYZus{}E}\PY{o}{/}\PY{o}{/}\PY{l+m+mi}{2}\PY{p}{)}\PY{p}{,}\PY{n}{dtype}\PY{o}{=}\PY{n}{np}\PY{o}{.}\PY{n}{float}\PY{p}{)}\PY{p}{]}\PY{p}{,}
        \PY{n}{axis}\PY{o}{=}\PY{l+m+mi}{2}\PY{p}{)}\PY{p}{,}
    \PY{n}{np}\PY{o}{.}\PY{n}{random}\PY{o}{.}\PY{n}{uniform}\PY{p}{(}\PY{o}{\PYZhy{}}\PY{l+m+mf}{0.5}\PY{p}{,}\PY{l+m+mf}{0.5}\PY{p}{,}\PY{n}{size}\PY{o}{=}\PY{p}{(}\PY{n}{d\PYZus{}E}\PY{p}{,} \PY{n}{d\PYZus{}S}\PY{p}{,} \PY{n}{d\PYZus{}E}\PY{p}{)}\PY{p}{)}
\PY{p}{)}
\PY{n}{betas} \PY{o}{=} \PY{n}{np}\PY{o}{.}\PY{n}{exp}\PY{p}{(}\PY{n}{np}\PY{o}{.}\PY{n}{minimum}\PY{p}{(}\PY{l+m+mf}{2.0}\PY{p}{,}\PY{n}{np}\PY{o}{.}\PY{n}{random}\PY{o}{.}\PY{n}{uniform}\PY{p}{(}\PY{l+m+mf}{0.0001}\PY{p}{,}\PY{l+m+mf}{4.0}\PY{p}{,}\PY{n}{size}\PY{o}{=}\PY{p}{(}\PY{n}{d\PYZus{}E}\PY{p}{,} \PY{n}{d\PYZus{}S}\PY{p}{,} \PY{n}{d\PYZus{}E}\PY{p}{)}\PY{p}{)}\PY{p}{)}\PY{p}{)}
\PY{n}{model}\PY{o}{.}\PY{n}{set\PYZus{}hawkes\PYZus{}parameters}\PY{p}{(}\PY{n}{nus}\PY{p}{,}\PY{n}{alphas}\PY{p}{,}\PY{n}{betas}\PY{p}{)}
\end{Verbatim}
\end{tcolorbox}

    Assign arbitrary values to the transition matrices \(\phi_{e}\), for
\(e=0,\dots,d_E -1\)
\begin{tcolorbox}[breakable, size=fbox, boxrule=1pt, pad at break*=1mm,colback=cellbackground, colframe=cellborder]
\prompt{In}{incolor}{7}{\boxspacing}
\begin{Verbatim}[commandchars=\\\{\}]
\PY{n}{phis} \PY{o}{=} \PY{n}{model}\PY{o}{.}\PY{n}{state\PYZus{}enc}\PY{o}{.}\PY{n}{generate\PYZus{}random\PYZus{}transition\PYZus{}prob}\PY{p}{(}\PY{n}{n\PYZus{}events}\PY{o}{=}\PY{n}{d\PYZus{}E}\PY{p}{)}
\PY{n}{model}\PY{o}{.}\PY{n}{set\PYZus{}transition\PYZus{}probabilities}\PY{p}{(}\PY{n}{phis}\PY{p}{)}
\end{Verbatim}
\end{tcolorbox}

    Assign arbitrary values to the Dirichlet parameters \(\gamma_{x}\) for
\(x = 0,\dots,d_{S}-1\)
\begin{tcolorbox}[breakable, size=fbox, boxrule=1pt, pad at break*=1mm,colback=cellbackground, colframe=cellborder]
\prompt{In}{incolor}{8}{\boxspacing}
\begin{Verbatim}[commandchars=\\\{\}]
\PY{n}{gammas} \PY{o}{=} \PY{n}{np}\PY{o}{.}\PY{n}{random}\PY{o}{.}\PY{n}{uniform}\PY{p}{(}\PY{n}{low}\PY{o}{=}\PY{l+m+mf}{1.5}\PY{p}{,} \PY{n}{high} \PY{o}{=} \PY{l+m+mf}{5.6}\PY{p}{,}\PY{n}{size}\PY{o}{=}\PY{p}{(}\PY{n}{d\PYZus{}S}\PY{p}{,}\PY{l+m+mi}{2}\PY{o}{*}\PY{n}{n\PYZus{}levels}\PY{p}{)}\PY{p}{)}
\PY{n}{model}\PY{o}{.}\PY{n}{set\PYZus{}dirichlet\PYZus{}parameters}\PY{p}{(}\PY{n}{gammas}\PY{p}{)}
\end{Verbatim}
\end{tcolorbox}

Launch simulation.
\begin{tcolorbox}[breakable, size=fbox, boxrule=1pt, pad at break*=1mm,colback=cellbackground, colframe=cellborder]
\prompt{In}{incolor}{9}{\boxspacing}
\begin{Verbatim}[commandchars=\\\{\}]
\PY{n}{model}\PY{o}{.}\PY{n}{simulate}\PY{p}{(}\PY{n}{time\PYZus{}start}\PY{p}{,} \PY{n}{time\PYZus{}end}\PY{p}{,}
               \PY{n}{max\PYZus{}number\PYZus{}of\PYZus{}events}\PY{o}{=}\PY{l+m+mi}{10000}\PY{p}{,}
               \PY{n}{add\PYZus{}initial\PYZus{}cond}\PY{o}{=}\PY{k+kc}{True}\PY{p}{,}
               \PY{n}{store\PYZus{}results}\PY{o}{=}\PY{k+kc}{True}\PY{p}{,}\PY{n}{report\PYZus{}full\PYZus{}volumes}\PY{o}{=}\PY{k+kc}{False}\PY{p}{)}
\end{Verbatim}
\end{tcolorbox}
\begin{tcolorbox}[breakable, size=fbox, boxrule=.5pt, pad at break*=1mm, opacityfill=0]
\prompt{Out}{outcolor}{9}{\boxspacing}
\begin{scriptsize}
\begin{Verbatim}[commandchars=\\\{\}]
Simulation is being performed on the following machine:
posix.uname\_result(sysname='Linux', nodename='claudio-HP-EliteBook-850-G3',
release='5.3.0-51-generic', version='\#44\textasciitilde{}18.04.2-Ubuntu SMP Thu Apr 23 14:27:18
UTC 2020', machine='x86\_64')
SDHawkes.simulate: initial conditions have been acknowledged
time at start: 0.0
intensities at start: [ 0.00275302 0.00565692 1.0013807  1.0000223 ]
intensity\_overall at start: 2.0098129325887664
SDHawkes.simulate: simulation terminates
time at end: 60.28442720962448
number of simulated events: 280
SDHawkes.simulate: run\_time: 0.1 seconds
\end{Verbatim}
\end{scriptsize}
\end{tcolorbox}
\begin{tcolorbox}[breakable, size=fbox, boxrule=1pt, pad at break*=1mm,colback=cellbackground, colframe=cellborder]
\prompt{In}{incolor}{10}{\boxspacing}
\begin{Verbatim}[commandchars=\\\{\}]
\PY{n}{model}\PY{o}{.}\PY{n}{store\PYZus{}history\PYZus{}of\PYZus{}intensities}\PY{p}{(}\PY{n}{density\PYZus{}of\PYZus{}eval\PYZus{}points}\PY{o}{=}\PY{l+m+mi}{5000}\PY{p}{)}
\end{Verbatim}
\end{tcolorbox}
\begin{tcolorbox}[breakable, size=fbox, boxrule=1pt, pad at break*=1mm,colback=cellbackground, colframe=cellborder]
\prompt{In}{incolor}{11}{\boxspacing}
\begin{Verbatim}[commandchars=\\\{\}]
\PY{n}{model}\PY{o}{.}\PY{n}{plot\PYZus{}events\PYZus{}and\PYZus{}states}\PY{p}{(}\PY{n}{t\PYZus{}0}\PY{o}{=}\PY{l+m+mf}{28.5}\PY{p}{,}\PY{n}{t\PYZus{}1}\PY{o}{=}\PY{l+m+mf}{35.0}\PY{p}{)}
\end{Verbatim}
\end{tcolorbox}
\begin{tcolorbox}[breakable, size=fbox, boxrule=.5pt, pad at break*=1mm, opacityfill=0]
\prompt{Out}{outcolor}{11}{\boxspacing}
\begin{center}
    \adjustimage{max size={0.9\linewidth}{0.9\paperheight}}{events_and_states_small.png}
    \end{center}
    { \hspace*{\fill} \\}
\end{tcolorbox}    
\begin{tcolorbox}[breakable, size=fbox, boxrule=1pt, pad at break*=1mm,colback=cellbackground, colframe=cellborder]
\prompt{In}{incolor}{12}{\boxspacing}
\begin{Verbatim}[commandchars=\\\{\}]
\PY{n}{model}\PY{o}{.}\PY{n}{plot\PYZus{}events\PYZus{}and\PYZus{}states}\PY{p}{(}\PY{p}{)}
\end{Verbatim}
\end{tcolorbox}
\begin{tcolorbox}[breakable, size=fbox, boxrule=.5pt, pad at break*=1mm, opacityfill=0]
\prompt{Out}{outcolor}{12}{\boxspacing}
\begin{center}
    \adjustimage{max size={0.9\linewidth}{0.9\paperheight}}{events_and_states.png}
    \end{center}
    { \hspace*{\fill} \\}
\end{tcolorbox}
\begin{tcolorbox}[breakable, size=fbox, boxrule=1pt, pad at break*=1mm,colback=cellbackground, colframe=cellborder]
\prompt{In}{incolor}{13}{\boxspacing}
\begin{Verbatim}[commandchars=\\\{\}]
\PY{n}{model}\PY{o}{.}\PY{n}{plot\PYZus{}intensities}\PY{p}{(}\PY{p}{)}
\end{Verbatim}
\end{tcolorbox}
\begin{tcolorbox}[breakable, size=fbox, boxrule=.5pt, pad at break*=1mm, opacityfill=0]
\prompt{Out}{outcolor}{13}{\boxspacing}
\begin{center}
    \adjustimage{max size={0.7\linewidth}{0.7\paperheight}}{intensities.png}
    \end{center}
    { \hspace*{\fill} \\}
\end{tcolorbox}

\subsection{Maximum likelihood estimation and gradient descent algorithm}
The module implements maximum likelihood estimation for state-dependent Hawkes processes with power-law kernels. The files involved in this are \texttt{mle\_estimation.pyx}, \texttt{minimisation\_algo.pyx} and \texttt{computation.pyx} in the directory \texttt{resources/} -- see Table \ref{tab.filesystem}. We describe their implementation focusing on the parallelisation of the various algorithms, which is necessary to efficienty handle the large amount of high-frequency data that we wish to analyse. 

Given a realisation of the state-dependent Hawkes process, we estimate the parameters $\baseRate_e$, $\impCoef\subscriptexe$, $\decCoef\subscriptexe$ and $\transProb_e(\xone,x)$, for $e, \eone =0,\dots,\numEventTypes-1$ and $x, \xone = 0,\dots, \numStates -1$, by maximising the log-likelihood $\loglikelihood$ of equation \eqref{eq.evaluation_of_loglikelihood}. The parameters $\baseRate_e$, $\impCoef\subscriptexe$, $\decCoef\subscriptexe$ only appear in the summands $\lplus$ and $\lminus$ of $\loglikelihood$, whereas the transition probabilities $\transProb_e(\xone,x)$ only appear in the first summand $\Phi$ of $\loglikelihood$. Hence, the maximisation of $\loglikelihood$ decouples in the maximisation of $\lplus-\lminus$ with respect to $\baseRate_e$, $\impCoef\subscriptexe$, $\decCoef\subscriptexe$, and in the maximisation of $\Phi$ with respect to $\transProb_e(\xone,x)$. This decomposition was the upshot of Theorem 3.1 in \citealp{MP18sta}. It  implies the remarkable fact that, at least in the case of a finite state-space, maximum-likelihood estimation of state-dependent Hawkes processes is not harder than that of ordinary Hawkes processes.  

The maximiser of $\Phi$ is obtained by setting the parameters $\transProb_e(\xone,x)$ equal to the empirical transition probabilities, namely
\begin{equation*}
  \transProb_{e}(\xone,x) = 
  \frac{\sum_j \one\lbrace\stateVariable(\arrivalTimes[e]_j - )=\xone, \, \, \stateVariable(\arrivalTimes[e]_j )=x \rbrace} {\sum_j \one\lbrace\stateVariable(\arrivalTimes[e]_j - )=\xone \rbrace}.
\end{equation*}
This has very low computational cost and is implemented straightforwardly. We thus focus on $\lplus-\lminus$.



\begin{prop}\label{prop.partial_hawkes_components}
 Let $\ell(0,\timeHorizon):= \lplus(0,\timeHorizon) - \lminus(0,\timeHorizon)$, where $\lplus$ and $\lminus$ are as in Proposition \ref{prop.parametric_hawkes_kernels}. For $e=0,\dots,\numEventTypes-1$, let  $\impCoef\subscriptdotse := \lbrace \impCoef\subscriptexe: \, \, \eone = 0,\dots, \numEventTypes-1 , \, \, \xone=0,\dots,\numStates-1\rbrace$ and $\decCoef\subscriptdotse := \lbrace \decCoef\subscriptexe: \, \, \eone = 0,\dots, \numEventTypes-1 , \, \, \xone=0,\dots,\numStates-1\rbrace$. Then, we can write 
 \begin{equation*}
  \ell(0,\timeHorizon)
  =
  \sum_{e=0}^{\numEventTypes-1}
  \Big(
  \leplus(0,\timeHorizon) - \leminus(0,\timeHorizon)
  \Big),
 \end{equation*}
where for every $e=0,\dots, \numEventTypes-1$ the summands $\leplus(0,\timeHorizon)$ and $\leminus(0,\timeHorizon)$ are independent of $\lbrace \nu_{\eone}, \, \impCoef\subscriptdotseone , \, \decCoef\subscriptdotseone : \, \, \eone \ne e\rbrace$. As a consequence, minimising 
\begin{equation*}
 (\baseRate,\impCoef,\decCoef) \quad \longmapsto \quad 
- \ell(0,\timeHorizon)
\end{equation*}
is equivalent to the $\numEventTypes$ independent minimisations
\begin{equation*}
 (\baseRate_e,\impCoef\subscriptdotse,\decCoef\subscriptdotse) \quad \longmapsto \quad 
 -\leplus(0,\timeHorizon) + \leminus(0,\timeHorizon),
\end{equation*}
for $e=0,\dots,\numEventTypes-1$.
\end{prop}
\begin{proof}
 Immediate. 
\end{proof}

Proposition \ref{prop.partial_hawkes_components} suggests the following parallelisation. Rather than targeting  $(\baseRate,\impCoef,\decCoef) \mapsto -\ell(0,\timeHorizon)$ directly, we independently and asynchronously apply a gradient descent algorithm to $(\baseRate_e,\impCoef\subscriptdotse,\decCoef\subscriptdotse) \mapsto  
 -\leplus(0,\timeHorizon) + \leminus(0,\timeHorizon)$, for $e=0,\dots,\numEventTypes-1$, and a posteriori we join the minimisers $\lbrace(\hat{\baseRate}_e,\hat{\impCoef}\subscriptdotse,\hat{\decCoef}\subscriptdotse): \, e=0,\dotsm \numEventTypes -1 \rbrace $ into the estimate $(\hat{\baseRate},\hat{\impCoef},\hat{\decCoef})$. Notice that the decomposition also benefits from the reduction of dimensionality in the impact coefficients and decay coefficients. We will refer to this parallelisation as \emph{higher parallelisation}, in order to distinguish it from a further parallelisation that takes place at a lower level in the algorithm and will be discussed later.
 
 We implement the parallelisation suggested by Proposition \ref{prop.partial_hawkes_components} by submitting an array job of length $\numEventTypes$ to the hpc cluster. This type of parallelism is often referred to as \emph{data parallelism}, as opposed to \emph{thread-level parallelism} (which we will also deploy below). An array job of length $n$ is a shell script that instructs the PBS to execute $n$ instances of itself, independently and in parallel on different nodes of the cluster. Each instance is aware of its own ``identity'' thanks to the value of the environment variable \texttt{PBS\_ARRAY\_INDEX}, which we assume ranging from $0$ to $n-1$. Hence, a prototype for distributing the minimisation of $-\ell(0,\timeHorizon)$ into the $\numEventTypes$ parallel minimisations of $ -\leplus(0,\timeHorizon) + \leminus(0,\timeHorizon)$ is provided below in Listings \ref{listing.launch}, \ref{listing.wrappersubmission}, and \ref{listing.main_testestim}.
 
Listing \ref{listing.launch} reports the shell script \texttt{launch.sh}, which submits the array job by invoking the PBS command \texttt{qsub} on a second shell script, named \texttt{wrappersubmission.sh}. The latter script is reported in Listing \ref{listing.wrappersubmission}.
In \texttt{launch.sh}, the command \texttt{qsub} is actually invoked twice. The first is on line 6 in order to submit the array job. It specifies the walltime request, the request for the computational resources, the name of the job, the range of the array index (\texttt{ -J 0-}$(\numEventTypes -1)$ ), and it passes the variables \texttt{a} and \texttt{p} to \texttt{wrappersubmission.sh}. The variable \texttt{a} specifies the type of action to perform; initially this action is \texttt{-e} for ``estimation''. The variable \texttt{p} specifies the type of parallelism to be used within every execution of the array job; it happens at a lower level than the distribution of the minimisation of $-\ell(0,\timeHorizon)$, and we will discuss it later. 

The second call of \texttt{qsub} is on line 9, and it is used to submit the job responsible for merging the partial results from the distributed minimisations. Notice that the action variable \texttt{a} is accordingly passed with the value \texttt{-m}, standing for ``merge'', and we set the job dependency \texttt{-W depend=afterok:\$jobid\_e}, in order to have the job executed only after the distributed minimisations have successfully returned. 

The shell script \texttt{wrappersubmission.sh} reported in Listing \ref{listing.wrappersubmission} is responsible for loading anaconda and for activating the python environment \texttt{pipest}, containing the libraries used by our module. Having received the variables \texttt{a} and \texttt{p} from \texttt{launch.sh}, it executes the python script \texttt{main.py} as specified by these variables.

Listing \ref{listing.main_testestim} reports a prototype for this python script. The actual script is longer and more nuanced, but the reported prototype better conveys the essence of the deployed data parallelism. Depending on the string at \texttt{argv[1]} (namely, the value of the variable \texttt{a} in the shell script), it either calls \texttt{estimate} or \texttt{merge\_from\_partial}. The former performs the actual minimisation, the latter merges the partial results. 

Recall that, in view of the array job, the execution of \texttt{>>python main.py -e ``\$p''} happens $\numEventTypes$ times, and hence \texttt{estimate} is actually called $\numEventTypes$ times. When \texttt{estimate} is called, it loads a previously saved instance of \texttt{model}, which is assumed to be initialised with model parameters $\numEventTypes$ and $\numStates$, and which is supposed to store a sample of $(\arrivalTimes[]_n, \event[n], \stateVariable_n)$ on which the user wishes to calibrate the model. It then selects on which component it performs the minimisation by scanning the range $e=0,\dots, \numEventTypes -1$ and  selecting the value $e$ equal to the environment variable \texttt{PBS\_ARRAY\_INDEX}; this identifies the execution of \texttt{>>python main.py -e ``\$p''} within which \texttt{estimate} has been called (lines 9-11 in Listing \ref{listing.main_testestim}). After this selection, the methods of the class \texttt{model} are called to initialise the subclass \texttt{mle\_estim} (line 13), to set its hyper-parameters (among which the type of lower parallelisation to be discussed below, line 15), and to launch the gradient descent for the minimisation of  $ -\leplus(0,\timeHorizon) + \leminus(0,\timeHorizon)$, where $e$ is as selected (line 17). 

When \texttt{merge\_from\_partial} is called, it loads the instances of \texttt{model} with the stored partial minimisers $\lbrace(\hat{\baseRate}_e,\hat{\impCoef}\subscriptdotse,\hat{\decCoef}\subscriptdotse): \, e=0,\dotsm \numEventTypes -1 \rbrace $, and it joins them in the estimate $(\hat{\baseRate},\hat{\impCoef},\hat{\decCoef})$.

\lstinputlisting[language=Scilab,caption=launch.sh, label=listing.launch]{./snippets/launch.sh}
\lstinputlisting[language=Scilab,caption=wrappersubmission.sh, label=listing.wrappersubmission]{./snippets/wrappersubmission.sh}
\lstinputlisting[language=Python,caption=main.py, label=listing.main_testestim]{./snippets/main.py}

As seen in Listing \ref{listing.main_testestim}, the class \texttt{model} handles the mle estimation through its subclass \texttt{mle\_estim}. This is imported from the resource \texttt{mle\_estimation.pyx} (where the name of the class is \texttt{EstimProcedure}), and it is responsible for preparing the sample data to be analysed and for setting the hyper-parameters of the minimisation algorithm. These hyper-parameters are: the type of lower parallelisation, the list of initial positions where the gradient descent starts, the size of the data batches to feed to the gradient descent, the number of runs for each batch, the maximum number of iterations for each run, and the learning rate in the gradient descent.
The first hyper-parameter (type of lower parallelisation) is the way in which the computations associated with the gradient descent search are distributed among the available cores. Notice that this happens at a lower level compared to the parallelisation suggested by Proposition \ref{prop.partial_hawkes_components}, and hence justifies its name of \emph{lower} parallelisation. 

Given the component $e$ of the $\numEventTypes$-dimensional Hawkes process, the target of the minimisation algorithm is $-\leplus(0,\timeHorizon) + \leminus(0,\timeHorizon)$, and this function is non-convex in the variable $(\baseRate_e,\impCoef\subscriptdotse,\decCoef\subscriptdotse)$. Hence, the gradient descent is sensitive to the initial position, it might converge to a local minimum and/or the convergence might require a long iteration. Moreover, the (possibly high) number of states $\numStates$ is associated to phenomena similar to curse of dimensionality. To mitigate these issues, we run the gradient descent with several initial positions (also referred to as ``initial guesses'' and listed among the hyper-parameters of the gradient descent), and we keep the best minimiser among the results of those runs.

Beside tuning initial guesses by hand or randomly generating them, two more grounded methods to fill the list of initial guesses are implemented. One is the preliminary estimation of an ordinary Hawkes process to fit the one-dimensional counting process $(\arrivalTimes[e]_j)_j$, the other is the non-parametric estimation of marked Hawkes processes introduced in \citealp{BM16fir}, \citealp{BM14haw}, and \citealp{BJM16est}. The former is implemented from within the file \texttt{mle\_estimation.pyx} by calling the same minimisation algorithm used for the state-dependent version, where however we set the number of states equal to 1. This means in particular that the assumption of power-law kernels is kept in place for the ordinary Hawkes processes; hence, for every $e$, the estimation will yield a base rate $n_e$, and $\numEventTypes$ impact coefficients $a\subscriptee$ and $\numEventTypes$ decay coefficients $b\subscriptee$, where $\eone$ is ranging from $0$ to $\numEventTypes-1$.  When this minimisation returns the estimators $\lbrace (n_e, a_{\eone,e}, b_{\eone,e}): \,\, \eone = 0, \dots,\numEventTypes-1\rbrace$ of the ordinary Hawkes process, the initial guess for the main gradient descent is set as
\begin{equation*}
 \baseRate_e := n_e
 \qquad
 \impCoef\subscriptexe := a_{\eone,e}
 \qquad
 \decCoef\subscriptexe := b_{\eone,e}
\end{equation*}
for all $\xone=0,\dots, \numStates -1$ and all $\eone=0,\dots, \numEventTypes -1$.

The production of an initial guess based on non-parametric estimation of marked Hawkes processes is instead implemented separately in the resource \texttt{nonparam\_estimation.pyx}, and it will be described in Section \ref{sec.nonparam_estim}.

The class \texttt{EstimProcedure} in \texttt{mle\_estimation.pyx} launches the minimisation by calling the resource \texttt{minimisation\_algo.pyx}, which implements the gradient descent. A prototype of the latter resource is given below in Listing \ref{listing.minimisation}. It shows the main skeleton of the actual file in the module, and we use it to discuss first the gradient descent algorithm and its call to the resource \texttt{computation.pyx} which evaluates the log-likelihood $\loglikelihood$, and secondly we use it to discuss the two alternatives for the lower parallelisation in the gradient descent search. 

The code snippet in Listing \ref{listing.minimisation} comprises the class \texttt{MinimisationProcedure} and the function \texttt{grad\_descent}. The former handles the call to the latter, in particular with regard to the type of lower parallelisation, and it stores the result of the gradient descent. 

\begin{figure}
\centering
 \includegraphics[width=.6\textwidth]{arch_minim.png}
 \caption{Prototype calls to the methods in the class \texttt{MinimisationProcedure} of \texttt{minimisation\_algo.pyx}.}
 \label{fig.callgraph_minim}
\end{figure}

The function \texttt{grad\_descent} presents the algorithm for the approximation to the minimiser. It is a classical gradient descent where however we implemented the following two variants. The first variant is the rejection of the update \texttt{x1 = x-a*gradf} if it does not yield a lower value for the objective function (Lines 39-42 in Listing \ref{listing.minimisation}). If the rejection happens, the learning rate is (randomly) renewed, and the search continues with this renewed learning rate. This variant has proven useful in preventing the algorithm from drifting away and diverging when it explores oscillatory regions of the objective function.

The second variant is the simultaneous evaluation of the objective function and its gradient via the call to the appropriate function in the resource \texttt{computation.pyx} (Line 32 in Listing \ref{listing.minimisation}). This responds to the need of optimising the number of operations associated with each step of the gradient descent: although it does not decrease the computational complexity of the algorithm (which remains $O(N\squared)$), it halves the number of computations by merging the evaluation of the objective function and its gradient in one single function call. Building on the formulae in Propositions \ref{prop.intensity_and_compensator_powerlaw} and \ref{prop.derivatives_of_loglikelihood}, it is based on the observation that the evaluation of  $\leplus$ and its gradient $\nabla \leplus$ can be nested in fewer loops than the formulae in equations \eqref{eq.intensity_powerlaw} and \eqref{eq.derivatives_of_lplus} may indicate at a first glance. The crucial steps in this implementation are reported in Listing \ref{listing.lplus}. A similar nesting is also possible for the  evaluation of  $\leminus$ and its gradient $\nabla \leminus$, starting from equations \eqref{eq.eval_compensator_powerlaw} and \eqref{eq.derivatives_of_lminus}, and the crucial steps in its implementation are reported in Listing \ref{listing.lminus}.

A diagram of the function calls within \texttt{minimisation\_algo.pyx} and from \texttt{minimisation\_algo.pyx} to \texttt{computation.pyx} is shown in Figure \ref{fig.callgraph_minim}. 

The first alternative for the lower parallelisation is to run asynchronous gradient descents, one for each initial guess. Hence, it is inscribed in the paradigm of data parallelism, as the higher parallelisation was. This first alternative is implemented through the usage of the python package \texttt{multiprocessing}, and its class \texttt{Pool} in particular.\footnote{The reference documentation for \texttt{multiprocessing.Pool} is available at \url{https://docs.python.org/3/library/multiprocessing.html\#multiprocessing.pool.Pool}} Lines 16-22 in Listing \ref{listing.minimisation} exhibits the basic asynchronous calls of \texttt{grad\_descent} via \texttt{multiprocessing.Pool}.   

The second alternative for the lower parallelisation is to use a thread-level parallelism in the evaluation of the objective function. It thus brings the parallelisation from the resource \texttt{minimisation\_algo.pyx } to the resource \texttt{computation.pyx}. We implement this parallelisation by applying the function \texttt{prange} of the module \texttt{cython.parallel}\footnote{The reference documentation for \texttt{cython.parallel} is available at \url{https://cython.readthedocs.io/en/latest/src/userguide/parallelism.html}.} to the outermost loops in the functions \texttt{compute\_partial\_at\_arrival\_times}, \texttt{compute\_sum\_of\_S3} and \texttt{compute\_sum\_of\_S2} of \texttt{computation.pyx}. Since \texttt{cython.parallel.prange} is able to automatically infer thread-locality and reductions, and since our implementations of the resources in \texttt{computation.pyx} already allowed for the GIL to be released, the application of \texttt{prange} is as straightforward as replacing lines 54-55 in Listing \ref{listing.lplus} with
\begin{verbatim}
for t in prange(num_arrival_times, nogil=True): 
\end{verbatim}
and lines 31-32 and 53-54 in Listing \ref{listing.lminus} with 
\begin{verbatim}
for e1 in prange(num_event_types, nogil=True):
\end{verbatim}

The module \texttt{cython.parallel} supports the backend OpenMP,\footnote{OpenMP is an application programming interface that supports multi-platform shared-memory multiprocessing programming; it is managed by the consortium \emph{OpenMP Architecture Review Board} -- see \url{https://www.openmp.org/}.} and due to the restrictions of the latter the functionality in the former can only be used from the main parallel region. Therefore, although the two types of lower parallelisation are not logically exclusive, we need to choose one or the other. A comparison of the speedups associated with either choice will be provided in Section \ref{sec.performance_tests}; the recommendation emerging from this comparison will be in favour of the thread-level parallelism. 

\lstinputlisting[language=Python,caption=Prototype for minimisation\_algo.pyx, label=listing.minimisation]{./snippets/minimisation.py}
\lstinputlisting[language=Python,caption=computation.compute\_$\leplus$\_and\_$\nabla\leplus$, label=listing.lplus]{./snippets/lplus.pyx}
\lstinputlisting[language=Python,caption=computation.compute\_$\leminus$\_and\_$\nabla\leminus$, label=listing.lminus]{./snippets/lminus.pyx}


\newpage 

\subsection{Non-parametric estimation of marked Hawkes processes}\label{sec.nonparam_estim}

We deal with a $\numEventTypes$-dimensional state-dependent Hawkes process $\multiCountingProc=(\countingProc[0], \dots, \countingProc[\numEventTypes-1])$, where the state space is finite and of cardinality $\numStates$. Hence, the process is described by the $\numEventTypes$ intensities
\begin{equation*}
 \intensity(t) = \baseRate_e + \sum_{\eone =0}^{\numEventTypes - 1}\int_{[0,t)} \hawkesKernel\subscriptee (t-s,\stateVariable(s))d\countingProc[\eone](s),
\end{equation*}
where $\stateVariable(s)$ is the state variable at time $s$. Notice that, in order to keep consistency with the cython code, we choose our indexes to range from $0$ to $n-1$ if they have to span a set of cardinality $n$ (in other words, we start counting from 0 rather than from 1). 

Since the state space is finite we can rewrite the intensities as 
\begin{equation*}
 \intensity(t) = \baseRate_e + \sum_{\eone =0}^{\numEventTypes - 1}\sum_{\xone=0}^{\numStates-1} \int_{[0,t)} \hawkesKernel\subscriptexe (t-s)d\countingProc[\eone, \xone](s),
\end{equation*}
where $\countingProc[\eone, \xone]$ counts the events of type $\eone$ that are followed by a state of type $\xone$, and $\hawkesKernel\subscriptexe(t) = \hawkesKernel\subscriptee(t,\xone)$. In the case of exponential kernels we have $\hawkesKernel\subscriptexe(t) = \impCoef\subscriptexe \exp(-\decCoef\subscriptexe \, t)$, whereas in the case of power-law kernels we have $\hawkesKernel\subscriptexe(t) = \impCoef\subscriptexe (t+1)^{-\decCoef\subscriptexe}$. 

In order to reconcile state-dependent Hawkes processes with the non-parametric estimation procedure outlined in \citealp{BM16fir}, we place the following two assumptions. Firstly, we assume that the point process is stationary. This allows in particular to define the expected intensity of the $e$-th component as $\Lambda_e = \Expectation[\intensity(t)]$, where the right hand side does not in fact depends on $t$ because of stationarity. Secondly, we assume that the states are i.i.d. with distribution only dependent on the event type that immediately precedes the update of the state variable, in the following sense: if $\arrivalTimes[]_n$ denotes the sequence of jump times and $\event[n]$ denotes the sequence of event labels (i.e. $\event$ is the type of event that happened at $\arrivalTimes[]_n$), then we assume that
there exist $\numEventTypes$ probabilities $\transProb_e$, $e=0,\dots,\numEventTypes - 1$,  on the state space such that 
 \begin{equation*}
   \Prob\left( \stateVariable(\arrivalTimes[ ]_{n})=x \,  \vert \,  E_n, \, \internalHistory_{\arrivalTimes[ ]_{n}-} \right)
   =
   \Prob\left( \stateVariable(\arrivalTimes[ ]_{n})=x \,  \vert \,  E_n\right)
   =
   \transProb_{E_n}\left(x\right),
   \qquad
   x=0, \dots, \numStates-1,
\end{equation*}
where $\internalHistory_{\arrivalTimes[ ]_{n}-} = \bigvee_{\epsilon>0} \internalHistory_{\arrivalTimes[ ]_{n} - \epsilon}$ represents the information available immediately before the arrival of the $n$-th event.


Following \citealp{BM16fir}, we define $g\subscriptexe$ as the density of the non-singular component of the conditional law of $\countingProc$ given that the current state is $\xone$ and the previous event was of type $\eone$. The defining formula for $g\subscriptexe$ is 
\begin{equation*}
 g\subscriptexe(t)dt = \Expectation[d\countingProc[e](t)=1 \vert d\countingProc[\eone](0)=1, \stateVariable(0)=\xone] - \Lambda_e dt, \qquad t>0.
\end{equation*}
Then, the classical Wiener-Hopf equation generalises to the case of marked point processes as stated in Proposition \ref{prop.wienerhopf}.
\begin{prop}[{\citealp[Proposition 3]{BM16fir}}]\label{prop.wienerhopf}
 Let $\hawkesKernel\subscriptexe$ and $g\subscriptexe$ be as above. Then, for all $e,\eone = 0,\dots,\numEventTypes -1$, all $\xone = 0,\dots,\numStates-1$ and all  $t>0$ it holds
 \begin{equation}\label{eq.wienerhopf}
  g\subscriptexe(t) = \hawkesKernel\subscriptexe(t) 
  +
  \sum_{\epsilon = 0}^{\numEventTypes -1}\sum_{y=0}^{\numStates-1} \int_{0}^{\infty} \hawkesKernel\subscriptepsye (s) \WHconvkernel_{\xone,y}^{\eone,\epsilon}(t-s)ds,
 \end{equation}
where $\WHconvkernel_{\xone,y}^{\eone,\epsilon}(t) = \transProb_{\epsilon}(y) g_{\eone,\xone,\epsilon}(t) \one\lbrace t\geq 0\rbrace + (\Lambda_{\epsilon}/\Lambda_{\eone})\transProb_{\epsilon}(y) g_{\epsilon,y,\eone}(-t)\one\lbrace t<0\rbrace$.
\end{prop}
Notice that in the non-marked case (i.e. when $\numStates = 1$), equation \eqref{eq.wienerhopf} reduces to the classical Wiener-Hopf equation. Furthermore, if we take $\numEventTypes = \numStates = 1$ (as we shall in the test below), equation \eqref{eq.wienerhopf} reduces to the particularly simple expression
\begin{equation}\label{eq.wienerhopf_onedim}
 g(t) = \hawkesKernel(t) + \Big(\hawkesKernel \star g\Big) (t),
\end{equation}
where we suppressed the subscripts $\eone$, $\xone$ and $e$ from $g$ and from $\hawkesKernel$, because in this case they can only take the value $0$.

The non-parametric estimation introduced in \citealp{BM16fir} consists in  1. estimating $g\subscriptexe$ via empirical averages; 2. discretising equation \eqref{eq.wienerhopf} using quadrature formulae;  and 3. solving the resulting linear system for the values $\hawkesKernel\subscriptexe(s_n)$ of the Hawkes kernels at quadrature points $s_n$, $n=0,1,2,\dots$. We describe this in detail following the more recent approach to quadrature suggested in \citealp{BJM16est}, which was introduced to deal with power-law kernels. Notice that \citealp{BJM16est} does not treat the case of marked Hawkes processes, but the generalisation to the case of marks is (at least on paper) straightforward. 

\subsubsection{Estimation of $g\subscriptexe$}
We follow Appendix A in \citealp{BJM16est} and we discretise $g\subscriptexe$ on a grid $\partition$ with two scales: linear from $0$ to $h_{\text{min}}$, and logarithmic from $h_{\text{min}}$ to $h_{\text{max}}$. More precisely, let $0<h_{\text{min}}<h_{\text{max}}$, fix an increment $\delta$, and let $\partition$ be defined by 
\begin{equation*}
 \begin{split}
  \partition \cap [0,h_{\text{min}}] =& \quad 
  \lbrace 0<\delta<2\delta< \dots< (n-1)\delta < h_{\text{min}} \rbrace,
  \\
  \partition \cap [h_{\text{min}},h_{\text{max}}] =& \quad 
 \lbrace h_{\text{min}}<h_{\text{min}}e^{\delta}<h_{\text{min}}e^{2\delta}< \dots< h_{\text{min}}e^{(n-1)\delta} < h_{\text{max}} \rbrace.
\end{split}
\end{equation*}
We will denote by $\tau_n$ the points of this partition. 

The stepsize $\delta$ must be chosen in a way that, on  the one hand, the partition is sufficiently fine to allow the value $g\subscriptexe(\tau_n)$  to be approximated by the average of $g\subscriptexe$ on the subinterval $[\tau_{n-1},\tau_{n+1}]$, namely 
\begin{equation*}
 g\subscriptexe (\tau_n) \approx \frac{1}{\tau_{n+1}-\tau_{n-1}} \int_{\tau_{n-1}}^{\tau_{n+1}} g\subscriptexe (s) ds;
\end{equation*}
on the other hand, the stepsize $\delta$ must be chosen so that every subinterval $[\tau_{n-1},\tau_{n+1}]$ contains a sufficiently large number of arrival times.

Then, we define the estimator $\hat{g}\subscriptexe(\tau_n)$ of the value $g\subscriptexe(\tau_n)$ that $g\subscriptexe$ takes at the partition point $\tau_n$ as 
\begin{equation*}
 \hat{g}\subscriptexe(\tau_n)
 :=
 \frac{1}{\tau_{n+1}-\tau_{n-1}}
 \frac{1}{\# \lbrace \arrivalTimes[\eone,\xone]_k\rbrace}
 \sum_{\arrivalTimes[e]_j} \sum_{\arrivalTimes[\eone,\xone]_k}
 \one\left\lbrace \tau_{n-1} \leq \arrivalTimes[e]_j - \arrivalTimes[\eone,\xone]_k < \tau_{n+1} \right\rbrace
 - \Lambda_e,
\end{equation*}
where $\arrivalTimes[e]_j$ are the jump times of the $e$-th component of the counting process,  where $\arrivalTimes[\eone, \xone]_k$ are the times when the $\eone$-th component jumps and it is followed by a state $\xone$, and where $\Lambda_e$ is the expected intensity $\Expectation[\intensity(t)]$.  
We define $\hat{g}\subscriptexe$ to be the linear interpolation of the points $\lbrace \hat{g}\subscriptexe(\tau_n): \, n=0,1,\dots\rbrace$.

\subsubsection{Discretisation of the Wiener-Hopf equation in \eqref{eq.wienerhopf}}
As in the previous paragraph, we start by introducing a two-scale partition. Given $0<t_{\text{min}} < t_{\text{max}}$ and some (small) $\epsilon>0$,  we let $\lbrace s_n: \, n=0,1,\dots\rbrace$ be the partition defined by
\begin{equation*}
 \begin{split}
  &0< \epsilon < 2\epsilon < \dots <  (m-1)\epsilon < t_{\text{min}} \\
  &t_{\text{min}}< t_{\text{min}}e^{\epsilon} < t_{\text{min}}e^{2\epsilon} < \dots <  t_{\text{min}}e^{(m-1)\epsilon} < t_{\text{max}}.
 \end{split}
\end{equation*}
The so-introduced partition $\lbrace s_n: \, n=0,1,\dots\rbrace$ takes the role that Gaussian quadrature points had in the first  version of the non-parametric estimation (\citealp{BM16fir}). Indeed, we assume that the Hawkes kernels $\hawkesKernel\subscriptexe$ are piecewise linear functions, and that their points of non-differentiability are (a subset of) $\lbrace s_n: \, n=0,1,\dots\rbrace$. In this way, the convolution integral in equation \eqref{eq.wienerhopf} can be broken down into the integration on the subintervals $[s_n, s_{n+1}]$, where the primitive of $\hawkesKernel\subscriptexe$ can be calculated exactly. We proceed by describing this strategy in detail.


We assume that 
\begin{equation*}
 \hawkesKernel\subscriptexe (t)
 =
 \sum_{n} \left(
 \hawkesKernel\subscriptexe (s_n)
 +
 \frac{\hawkesKernel\subscriptexe (s_{n+1}) - \hawkesKernel\subscriptexe (s_n)}{s_{n+1}-s_n}\Big(t-s_n\Big)
 \right)\one\lbrace s_n \leq t < s_{n+1}\rbrace.
\end{equation*}
With this assumption we can write 
\begin{equation}\label{eq.piecewise_WHintegral}
 \begin{split}
 \int_{0}^{\infty} \hawkesKernel\subscriptepsye (s) \WHconvkernel_{\xone,y}^{\eone,\epsilon}(t-s)ds
 =&
 \sum_{n} \hawkesKernel\subscriptepsye (s_n) \int_{s_n}^{s_{n+1}}\WHconvkernel_{\xone,y}^{\eone,\epsilon}(t-s)ds
 \\
 &+
 \sum_{n} \frac{\hawkesKernel\subscriptexe (s_{n+1}) - \hawkesKernel\subscriptexe (s_n)}{s_{n+1}-s_n}
 \int_{s_n}^{s_{n+1}}(s-s_n)\WHconvkernel_{\xone,y}^{\eone,\epsilon}(t-s)ds.
 \end{split}
\end{equation}


We replace $g$ with $\hat{g}$ in the definition of the convolution kernel $\WHconvkernel_{\xone,y}^{\eone,\epsilon}$ of Proposition \ref{prop.wienerhopf}, so that we can approximate 
\begin{equation}\label{eq.approxWHconvkernel}
 \int_{s_{n}}^{s_{n+1}} \WHconvkernel_{\xone,y}^{\eone,\epsilon} (s_m -s )ds
 \approx
 \hat{\WHconvkernel}_{\xone,y}^{\eone,\epsilon}(m,n) 
 :=
 \begin{cases}
  \transProb_{\epsilon} (y) (s_{n+1}-s_n) \hat{g}_{\eone,\xone,\epsilon}^{m,n} & \text{ if } n+1\leq m
  \\
  \frac{\Lambda_\epsilon}{\Lambda_{\eone}} \transProb_{\epsilon} (y) (s_{n+1}-s_n) \hat{g}_{\epsilon,y,\eone}^{m,n} & \text{ if } n\geq m , 
 \end{cases}
\end{equation}
where 
\begin{equation*}
 \hat{g}\subscriptexe^{m,n} := \frac{1}{s_{n+1}-s_{n}}\int_{s_n}^{s_{n+1}} \hat{g}\subscriptexe (\abs{s_m -s})ds.
\end{equation*}

In the same way we can approximate
\begin{equation}\label{eq.approxWHconvkernel_one}
 \frac{1}{s_{n+1}-s_{n}}\int_{s_{n}}^{s_{n+1}} (s-s_n)\WHconvkernel_{\xone,y}^{\eone,\epsilon} (s_m -s )ds
 \approx
 \hat{\WHconvkernel}_{\xone,y}^{(1) \, \, \eone,\epsilon}(m,n) 
 :=
 \begin{cases}
  \transProb_{\epsilon} (y) \hat{g}_{\eone,\xone,\epsilon}^{(1) \, \, m,n} & \text{ if } n+1\leq m
  \\
  \frac{\Lambda_\epsilon}{\Lambda_{\eone}} \transProb_{\epsilon} (y) \hat{g}_{\epsilon,y,\eone}^{(1) \, \, m,n} & \text{ if } n\geq m , 
 \end{cases}
\end{equation}
where 
\begin{equation*}
 \hat{g}\subscriptexe^{(1) \, \, m,n} := \frac{1}{s_{n+1}-s_{n}}\int_{s_n}^{s_{n+1}} (s-s_n)\hat{g}\subscriptexe (\abs{s_m -s})ds.
\end{equation*}

The values $\hat{g}\subscriptexe^{m,n}$ and  $\hat{g}\subscriptexe^{(1) \, \, m,n}$ can be computed exactly because $\hat{g}$ by construction is piecewise linear with points of non-differentiability given by the partition $\lbrace \tau_k : k=0,1,\dots\rbrace$ introduced in the previous paragraph. 

Finally, by combining equations \eqref{eq.piecewise_WHintegral}, \eqref{eq.approxWHconvkernel} and \eqref{eq.approxWHconvkernel_one}, we get to our discretisation of the Wiener-Hopf equation. We state this  in form of a proposition.

\begin{prop}\label{prop.discretised_wienerhopf}
 With the notation introduced so far,  equation \eqref{eq.wienerhopf} is discretised as per the formula
 \begin{equation}\label{eq.discretised_wienerhopf}
 \hat{g}\subscriptexe (s_m)
 =
 \hawkesKernel\subscriptexe (s_m)
 +\sum_{\epsilon = 0}^{\numEventTypes -1}\sum_{y=0}^{\numStates-1}
 \sum_{n}
 \Big(
 \hat{\WHconvkernel}_{\xone,y}^{\eone,\epsilon}(m,n)
 -\hat{\WHconvkernel}_{\xone,y}^{(1) \, \, \eone,\epsilon}(m,n)
 -\hat{\WHconvkernel}_{\xone,y}^{(1) \, \, \eone,\epsilon}(m,n-1)
 \Big)
 \hawkesKernel\subscriptepsye(s_n).
 \end{equation}
\end{prop}

\subsubsection{Linear system for the estimation of $\hawkesKernel\subscriptexe$}
We rephrase equation \eqref{eq.discretised_wienerhopf} in terms of a linear system.

Recall that the indexes $e, \eone, \epsilon$ range from $0$ to $\numEventTypes -1$, and that the indexes $x,\xone, y$ range from $0$ to $\numStates -1$.  Let $Q$ be the number of points in the partition $[0,t_{\text{max}}] \cap \lbrace s_n: \, n=0,1,\dots\rbrace$, and let $m$, $n$ range from $0$ to $Q-1$. 
We define the index $i$ and the index $j$ as
\begin{equation*}
 \begin{split}
  i =& (\eone \numStates + \xone)Q +m\\
  j =& (\epsilon \numStates + y)Q +n
 \end{split}
\end{equation*}
We set 
\begin{equation*}
 \begin{split}
  A_{i,j} = & \quad 
  A_{(\eone \numStates + \xone)Q +m, \, \, (\epsilon \numStates + y)Q +n}
  = \hat{\WHconvkernel}_{\xone,y}^{\eone,\epsilon}(m,n)
 -\hat{\WHconvkernel}_{\xone,y}^{(1) \, \, \eone,\epsilon}(m,n)
 -\hat{\WHconvkernel}_{\xone,y}^{(1) \, \, \eone,\epsilon}(m,n-1)\one_{n\geq 1},
 \\
 b^{e}_{i} = & \quad 
 b^{e}_{(\eone \numStates + \xone)Q +m}
 =
 \hat{g}\subscriptexe (s_m),
 \\
 x^{e}_{i} = & \quad 
 x^{e}_{(\eone \numStates + \xone)Q +m}
 =
 \hawkesKernel\subscriptexe(s_m).
 \end{split}
\end{equation*}
The final step in the estimation procedure is therefore the solution of the following $\numEventTypes$ linear systems in the unknowns $x^{e}$:
\begin{equation*}
 (I + A) x^{e} = b^{e}, \qquad e=0,\dots, \numEventTypes-1,
\end{equation*}
where $I$ is the identity matrix.

\subsubsection{Test of the estimation procedure}
In the following test we consider a one-dimensional ordinary Hawkes process, whose kernel has  $\Lone$ norm just below 1.
We initialise the model using the python module pipest; we simulate a realisation of the stochastic process, and we perform the non-parametric estimation described above, in the attempt to retrieve a kernel close to the one that we started with. Unfortunately, this fails.

\begin{tcolorbox}[breakable, size=fbox, boxrule=1pt, pad at break*=1mm,colback=cellbackground, colframe=cellborder]
\prompt{In}{incolor}{2}{\boxspacing}
\begin{Verbatim}[commandchars=\\\{\}]
\PY{n}{list\PYZus{}of\PYZus{}n\PYZus{}states}\PY{p}{,} \PY{n}{n\PYZus{}events}\PY{p}{,} \PY{n}{n\PYZus{}levels} \PY{o}{=} \PY{p}{[}\PY{l+m+mi}{1}\PY{p}{,}\PY{l+m+mi}{1}\PY{p}{]}\PY{p}{,} \PY{l+m+mi}{1}\PY{p}{,} \PY{l+m+mi}{2}
\PY{n}{model} \PY{o}{=} \PY{n}{sdmodel}\PY{o}{.}\PY{n}{SDHawkes}\PY{p}{(}\PY{n}{number\PYZus{}of\PYZus{}event\PYZus{}types}\PY{o}{=}\PY{n}{n\PYZus{}events}\PY{p}{,}
                 \PY{n}{number\PYZus{}of\PYZus{}lob\PYZus{}levels}\PY{o}{=}\PY{n}{n\PYZus{}levels}\PY{p}{,}
                 \PY{n}{volume\PYZus{}imbalance\PYZus{}upto\PYZus{}level}\PY{o}{=}\PY{l+m+mi}{2}\PY{p}{,}
                 \PY{n}{list\PYZus{}of\PYZus{}n\PYZus{}states}\PY{o}{=}\PY{n}{list\PYZus{}of\PYZus{}n\PYZus{}states}
                \PY{p}{)}
\PY{n}{tot\PYZus{}n\PYZus{}states}\PY{o}{=}\PY{n}{model}\PY{o}{.}\PY{n}{state\PYZus{}enc}\PY{o}{.}\PY{n}{tot\PYZus{}n\PYZus{}states}

\PY{n}{nus} \PY{o}{=} \PY{l+m+mf}{0.015}\PY{o}{*}\PY{n}{np}\PY{o}{.}\PY{n}{random}\PY{o}{.}\PY{n}{randint}\PY{p}{(}\PY{n}{low}\PY{o}{=}\PY{l+m+mi}{1}\PY{p}{,}\PY{n}{high}\PY{o}{=}\PY{l+m+mi}{2}\PY{p}{,}\PY{n}{size}\PY{o}{=}\PY{n}{n\PYZus{}events}\PY{p}{)}
\PY{n}{a}\PY{o}{=}\PY{l+m+mf}{4.90}
\PY{n}{alphas} \PY{o}{=} \PY{n}{np}\PY{o}{.}\PY{n}{random}\PY{o}{.}\PY{n}{uniform}\PY{p}{(}\PY{n}{a}\PY{p}{,}\PY{n}{a}\PY{o}{+}\PY{l+m+mf}{0.01}\PY{p}{,}\PY{n}{size}\PY{o}{=}\PY{p}{(}\PY{n}{n\PYZus{}events}\PY{p}{,} \PY{n}{tot\PYZus{}n\PYZus{}states}\PY{p}{,} \PY{n}{n\PYZus{}events}\PY{p}{)}\PY{p}{)}
\PY{n}{betas} \PY{o}{=} \PY{n}{np}\PY{o}{.}\PY{n}{random}\PY{o}{.}\PY{n}{uniform}\PY{p}{(}\PY{l+m+mf}{5.9}\PY{p}{,} \PY{l+m+mf}{6.1}\PY{p}{,}\PY{n}{size}\PY{o}{=}\PY{p}{(}\PY{n}{n\PYZus{}events}\PY{p}{,} \PY{n}{tot\PYZus{}n\PYZus{}states}\PY{p}{,} \PY{n}{n\PYZus{}events}\PY{p}{)}\PY{p}{)}
\PY{n}{model}\PY{o}{.}\PY{n}{set\PYZus{}hawkes\PYZus{}parameters}\PY{p}{(}\PY{n}{nus}\PY{p}{,}\PY{n}{alphas}\PY{p}{,}\PY{n}{betas}\PY{p}{)}
\PY{n}{phis} \PY{o}{=} \PY{n}{model}\PY{o}{.}\PY{n}{state\PYZus{}enc}\PY{o}{.}\PY{n}{generate\PYZus{}random\PYZus{}transition\PYZus{}prob}\PY{p}{(}\PY{n}{n\PYZus{}events}\PY{o}{=}\PY{n}{n\PYZus{}events}\PY{p}{)}\PY{o}
\PY{n}{model}\PY{o}{.}\PY{n}{set\PYZus{}transition\PYZus{}probabilities}\PY{p}{(}\PY{n}{phis}\PY{p}{)}
\PY{n}{kappas} \PY{o}{=} \PY{n}{np}\PY{o}{.}\PY{n}{random}\PY{o}{.}\PY{n}{lognormal}\PY{p}{(}\PY{n}{size}\PY{o}{=}\PY{p}{(}\PY{n}{tot\PYZus{}n\PYZus{}states}\PY{p}{,}\PY{l+m+mi}{2}\PY{o}{*}\PY{n}{n\PYZus{}levels}\PY{p}{)}\PY{p}{)}
\PY{n}{model}\PY{o}{.}\PY{n}{set\PYZus{}dirichlet\PYZus{}parameters}\PY{p}{(}\PY{n}{kappas}\PY{p}{)}
\end{Verbatim}
\end{tcolorbox}

    \begin{Verbatim}[commandchars=\\\{\}]
Hawkes parameters have been set
SDHawkes: asymmetry in transition\_probabilities = 0.9962891407667446
Transition probabilities have been set
Dirichlet parameters have been set
    \end{Verbatim}
    
The Hawkes kernel $\impCoef (t+1)^{-\decCoef}$ is summarised as follows:     
   \begin{Verbatim}[commandchars=\\\{\}]
imp\_coef=4.901719822309374
dec\_coef=5.995184006603476
spectral\_radii: [0.98128914]
expected intensity: [10.09547108]
    \end{Verbatim}

\begin{center}
    \adjustimage{max size={0.9\linewidth}{0.9\paperheight}}{hknptest.png}
    \end{center}
    { \hspace*{\fill} \\}

    \begin{tcolorbox}[breakable, size=fbox, boxrule=1pt, pad at break*=1mm,colback=cellbackground, colframe=cellborder]
\prompt{In}{incolor}{3}{\boxspacing}
\begin{Verbatim}[commandchars=\\\{\}]
\PY{n+nb}{print}\PY{p}{(}\PY{l+s+s2}{\PYZdq{}}\PY{l+s+se}{\PYZbs{}n}\PY{l+s+s2}{SIMULATION}\PY{l+s+se}{\PYZbs{}n}\PY{l+s+s2}{\PYZdq{}}\PY{p}{)}
\PY{n}{time\PYZus{}start} \PY{o}{=} \PY{l+m+mf}{0.0}
\PY{n}{time\PYZus{}end} \PY{o}{=} \PY{n}{time\PYZus{}start} \PY{o}{+} \PY{l+m+mi}{9000} \PY{c+c1}{\PYZsh{}150 minutes}
\PY{n}{max\PYZus{}number\PYZus{}of\PYZus{}events} \PY{o}{=} \PY{l+m+mi}{20000}
\PY{n}{model}\PY{o}{.}\PY{n}{simulate}\PY{p}{(}
    \PY{n}{time\PYZus{}start}\PY{p}{,} \PY{n}{time\PYZus{}end}\PY{p}{,}\PY{n}{max\PYZus{}number\PYZus{}of\PYZus{}events}\PY{o}{=}\PY{n}{max\PYZus{}number\PYZus{}of\PYZus{}events}\PY{p}{,}
    \PY{n}{add\PYZus{}initial\PYZus{}cond}\PY{o}{=}\PY{k+kc}{True}\PY{p}{,}
    \PY{n}{store\PYZus{}results}\PY{o}{=}\PY{k+kc}{True}\PY{p}{,} \PY{n}{report\PYZus{}full\PYZus{}volumes}\PY{o}{=}\PY{k+kc}{False}\PY{p}{)}
\PY{n}{time\PYZus{}end}\PY{o}{=}\PY{n+nb}{float}\PY{p}{(}\PY{n}{model}\PY{o}{.}\PY{n}{simulated\PYZus{}times}\PY{p}{[}\PY{o}{\PYZhy{}}\PY{l+m+mi}{1}\PY{p}{]}\PY{p}{)}
\end{Verbatim}
\end{tcolorbox}

    \begin{Verbatim}[commandchars=\\\{\}]

SIMULATION

Simulation is being performed on the following machine:
 posix.uname\_result(sysname='Linux', nodename='hpelitebook850g3',
release='5.3.0-53-generic', version='\#47\textasciitilde{}18.04.1-Ubuntu SMP Thu May 7 13:10:50
UTC 2020', machine='x86\_64')
SDHawkes.simulate: initial conditions have been acknowledged
I am adding initial conditions on the negative time axis
sd\_hawkes\_powerlaw\_simulation.simulate: start of initialisation
   Number of levels in the order book: 2
   number\_of\_initial\_events=2
sd\_hawkes\_powerlaw\_simulation: simulate: labelled\_times and count have been
initialised.
sd\_hawkes\_powerlaw\_simulation: simulate: intensities have been initialised.
  intensities at start: [0.015]
  intensity\_overall at start: 0.015000000002927241
sd\_hawkes\_powerlaw\_simulation.simulate: start of simulation
  time\_start=0.0,  time at start =0.0
 Simulation terminates. time at end =9059.296543498653,  num\_of\_event=7441
sd\_hawkes\_powerlaw\_simulation: simulate. run\_time=3.9 seconds
    \end{Verbatim}

We perform goodness-of-fit diagnostic to check that the simulated realisation of the counting process has distributional features representative of the distribution that we are targeting. The following confirms that the this is indeed the case and our simulation algorithm works well.

    \begin{tcolorbox}[breakable, size=fbox, boxrule=1pt, pad at break*=1mm,colback=cellbackground, colframe=cellborder]
\prompt{In}{incolor}{4}{\boxspacing}
\begin{Verbatim}[commandchars=\\\{\}]
\PY{n}{model}\PY{o}{.}\PY{n}{create\PYZus{}goodness\PYZus{}of\PYZus{}fit}\PY{p}{(}\PY{n}{type\PYZus{}of\PYZus{}input}\PY{o}{=}\PY{l+s+s1}{\PYZsq{}}\PY{l+s+s1}{simulated}\PY{l+s+s1}{\PYZsq{}}\PY{p}{)}
\end{Verbatim}
\end{tcolorbox}

    \begin{Verbatim}[commandchars=\\\{\}]
I am computing residuals in parallel. num\_process: 1
goodness\_of\_fit.compute\_residuals. component\_e: 0; process\_id: pid4568
Computation of residuals terminates. run\_time=4.129154443740845
    \end{Verbatim}

    \begin{tcolorbox}[breakable, size=fbox, boxrule=1pt, pad at break*=1mm,colback=cellbackground, colframe=cellborder]
\prompt{In}{incolor}{5}{\boxspacing}
\begin{Verbatim}[commandchars=\\\{\}]
\PY{n}{model}\PY{o}{.}\PY{n}{goodness\PYZus{}of\PYZus{}fit}\PY{o}{.}\PY{n}{qq\PYZus{}plot\PYZus{}residuals}\PY{p}{(}\PY{p}{)}
\PY{n+nb}{print}\PY{p}{(}\PY{n}{model}\PY{o}{.}\PY{n}{goodness\PYZus{}of\PYZus{}fit}\PY{o}{.}\PY{n}{ks\PYZus{}test\PYZus{}on\PYZus{}residuals}\PY{p}{(}\PY{p}{)}\PY{p}{)}
\PY{n+nb}{print}\PY{p}{(}\PY{n}{model}\PY{o}{.}\PY{n}{goodness\PYZus{}of\PYZus{}fit}\PY{o}{.}\PY{n}{ad\PYZus{}test\PYZus{}on\PYZus{}residuals}\PY{p}{(}\PY{p}{)}\PY{p}{)}
\end{Verbatim}
\end{tcolorbox}
\begin{figure}[H]
 \includegraphics[width=.4\textwidth]{qqnptest.png}
\end{figure}

    \begin{Verbatim}[commandchars=\\\{\}]
Kolmogorov-Smirnov test to check that the residuals are iid with distribution
Exp(1)
event type=0, ks\_answer: KstestResult(statistic=0.009633092232216589,
pvalue=0.4948586813932767)
None
Anderson-Darling test to check distribution of residuals
Null hypothesis is "expon"
Significance levels: [15.  10.   5.   2.5  1. ]
Critical values: [0.922 1.078 1.341 1.606 1.957]
event type=0, ad\_stat: 0.7789373159594106
None
    \end{Verbatim}
Finally, we deploy the non-parametric estimation. Notation is in line with that in the paragraphs above. 

    \begin{tcolorbox}[breakable, size=fbox, boxrule=1pt, pad at break*=1mm,colback=cellbackground, colframe=cellborder]
\prompt{In}{incolor}{6}{\boxspacing}
\begin{Verbatim}[commandchars=\\\{\}]
\PY{n+nb}{print}\PY{p}{(}\PY{l+s+s2}{\PYZdq{}}\PY{l+s+se}{\PYZbs{}n}\PY{l+s+s2}{NON\PYZhy{}PARAMETRIC ESTIMATION}\PY{l+s+se}{\PYZbs{}n}\PY{l+s+s2}{\PYZdq{}}\PY{p}{)}
\PY{n}{tmax}\PY{o}{=}\PY{l+m+mf}{1.0e+02}
\PY{n}{tmin}\PY{o}{=}\PY{l+m+mf}{1.0e\PYZhy{}03}
\PY{n}{hmax}\PY{o}{=}\PY{l+m+mf}{5.0e+01}
\PY{n}{hmin}\PY{o}{=}\PY{l+m+mf}{1.0e\PYZhy{}02}
\PY{n}{num\PYZus{}quadpnts} \PY{o}{=} \PY{l+m+mi}{200}
\PY{n}{num\PYZus{}gridpnts}\PY{o}{=} \PY{l+m+mi}{30}
\PY{n}{model}\PY{o}{.}\PY{n}{create\PYZus{}nonparam\PYZus{}estim}\PY{p}{(}\PY{n}{type\PYZus{}of\PYZus{}input}\PY{o}{=}\PY{l+s+s1}{\PYZsq{}}\PY{l+s+s1}{simulated}\PY{l+s+s1}{\PYZsq{}}\PY{p}{,}
                            \PY{n}{num\PYZus{}quadpnts} \PY{o}{=} \PY{n}{num\PYZus{}quadpnts}\PY{p}{,}
                            \PY{n}{quad\PYZus{}tmax} \PY{o}{=} \PY{n}{tmax}\PY{p}{,}
                            \PY{n}{quad\PYZus{}tmin} \PY{o}{=} \PY{n}{tmin}\PY{p}{,}
                            \PY{n}{num\PYZus{}gridpnts} \PY{o}{=} \PY{n}{num\PYZus{}gridpnts}\PY{p}{,}
                            \PY{n}{grid\PYZus{}tmax} \PY{o}{=} \PY{n}{hmax}\PY{p}{,}
                            \PY{n}{grid\PYZus{}tmin} \PY{o}{=} \PY{n}{hmin}\PY{p}{,}
                            \PY{n}{two\PYZus{}scales}\PY{o}{=}\PY{k+kc}{True}\PY{p}{,}
                            \PY{n}{tol}\PY{o}{=}\PY{l+m+mf}{1.0e\PYZhy{}6}
                           \PY{p}{)}
\end{Verbatim}
\end{tcolorbox}

    \begin{Verbatim}[commandchars=\\\{\}]

NON-PARAMETRIC ESTIMATION

nonparam\_estimation.EstimProcedure is being initialised
I am storing distribution of marks
I am storing transition probabilities
I am storing expected intensities
EstimProcedure has been successfully initialised
    \end{Verbatim}
    
    \begin{tcolorbox}[breakable, size=fbox, boxrule=1pt, pad at break*=1mm,colback=cellbackground, colframe=cellborder]
\prompt{In}{incolor}{8}{\boxspacing}
\begin{Verbatim}[commandchars=\\\{\}]
\PY{n}{model}\PY{o}{.}\PY{n}{nonparam\PYZus{}estim}\PY{o}{.}\PY{n}{estimate\PYZus{}hawkes\PYZus{}kernel}\PY{p}{(}\PY{p}{)}
\end{Verbatim}
\end{tcolorbox}

    \begin{Verbatim}[commandchars=\\\{\}]
I am preparing estimation of hawkes kernel
I am storing non-singular expected jumps in parallel. num\_processes: 1
I am estimating non-singular expected jumps for the component e=0. process\_id:
pid4574
Execution of non-singular expected jumps terminates. run\_time=4.6215198040008545
I am storing convolution kernels
I am storing matrix A
Estimation of hawkes kernel is now ready
I am performing estimation of hawkes kernel serially
I am solving for the component e=0. process\_id: pid4550
Estimation of hawkes kernel terminates. run\_time=0.006167888641357422
    \end{Verbatim}

    \begin{tcolorbox}[breakable, size=fbox, boxrule=1pt, pad at break*=1mm,colback=cellbackground, colframe=cellborder]
\prompt{In}{incolor}{9}{\boxspacing}
\begin{Verbatim}[commandchars=\\\{\}]
\PY{n+nb}{print}\PY{p}{(}\PY{l+s+s1}{\PYZsq{}}\PY{l+s+s1}{Estimated values of the conditional law g: }\PY{l+s+se}{\PYZbs{}n}\PY{l+s+s1}{ }\PY{l+s+si}{\PYZob{}\PYZcb{}}\PY{l+s+s1}{\PYZsq{}}\PY{o}{.}\PY{n}{format}\PY{p}{(}\PY{n}{model}\PY{o}{.}\PY{n}{nonparam\PYZus{}estim}\PY{o}{.}\PY{n}{g\PYZus{}hat\PYZus{}at\PYZus{}gridpnts}\PY{p}{[}\PY{l+m+mi}{0}\PY{p}{,}\PY{l+m+mi}{0}\PY{p}{,}\PY{l+m+mi}{0}\PY{p}{,}\PY{p}{:}\PY{p}{]}\PY{p}{)}\PY{p}{)}
\end{Verbatim}
\end{tcolorbox}

    \begin{Verbatim}[commandchars=\\\{\}]
Estimated values of the conditional law g:
 [ 3.10925937e+03  1.10469045e+02  1.04017432e+02  9.15174321e+01
  9.19206579e+01  1.13291626e+02  9.43400128e+01  9.87754967e+01
  8.50658192e+01  9.35335612e+01  1.00791626e+02  9.59529160e+01
  9.95819483e+01  1.02001303e+02  9.99851741e+01  1.02001303e+02
  9.39367870e+01  1.07646464e+02  8.50658192e+01  9.99851741e+01
  8.70819483e+01  1.04420658e+02  8.99045289e+01  1.05227110e+02
  1.02404529e+02  9.19206579e+01  1.08452916e+02  9.91787225e+01
  9.19206579e+01  1.10469045e+02  1.02202148e+02  9.95962194e+01
  1.00722721e+02  9.96047203e+01  1.01488185e+02  1.01154555e+02
  9.88723115e+01  1.00275814e+02  9.86755256e+01  9.81297425e+01
  9.78360460e+01  9.64058587e+01  9.59755750e+01  9.36547389e+01
  9.20145834e+01  9.07046387e+01  8.84691015e+01  8.54821032e+01
  8.30872900e+01  7.79086416e+01  7.08548899e+01  6.38042176e+01
  5.04249155e+01  3.39963697e+01  1.89693698e+01  7.44676724e+00
  3.38936178e-02 -7.06587241e+00 -9.55169218e+00 -9.78435985e+00]
    \end{Verbatim}

    \begin{tcolorbox}[breakable, size=fbox, boxrule=1pt, pad at break*=1mm,colback=cellbackground, colframe=cellborder]
\prompt{In}{incolor}{10}{\boxspacing}
\begin{Verbatim}[commandchars=\\\{\}]
\PY{n}{fig}\PY{o}{=}\PY{n}{plt}\PY{o}{.}\PY{n}{figure}\PY{p}{(}\PY{n}{figsize}\PY{o}{=}\PY{p}{(}\PY{l+m+mi}{15}\PY{p}{,}\PY{l+m+mi}{5}\PY{p}{)}\PY{p}{)}
\PY{n}{ax}\PY{o}{=}\PY{n}{fig}\PY{o}{.}\PY{n}{add\PYZus{}subplot}\PY{p}{(}\PY{l+m+mi}{121}\PY{p}{)}
\PY{n}{t}\PY{o}{=}\PY{n}{model}\PY{o}{.}\PY{n}{nonparam\PYZus{}estim}\PY{o}{.}\PY{n}{grid}\PY{o}{.}\PY{n}{partition}\PY{p}{[}\PY{p}{:}\PY{o}{\PYZhy{}}\PY{l+m+mi}{1}\PY{p}{]}
\PY{n}{ghat}\PY{o}{=}\PY{n}{model}\PY{o}{.}\PY{n}{nonparam\PYZus{}estim}\PY{o}{.}\PY{n}{g\PYZus{}hat\PYZus{}at\PYZus{}gridpnts}\PY{p}{[}\PY{l+m+mi}{0}\PY{p}{,}\PY{l+m+mi}{0}\PY{p}{,}\PY{l+m+mi}{0}\PY{p}{,}\PY{p}{:}\PY{p}{]}
\PY{n}{ax}\PY{o}{.}\PY{n}{plot}\PY{p}{(}\PY{n}{t}\PY{p}{,}\PY{n}{ghat}\PY{p}{)}
\PY{n}{ax}\PY{o}{.}\PY{n}{set\PYZus{}title}\PY{p}{(}\PY{l+s+s1}{\PYZsq{}}\PY{l+s+s1}{g\PYZus{}hat on the full integral where defined}\PY{l+s+s1}{\PYZsq{}}\PY{p}{)}
\PY{n}{ax1}\PY{o}{=}\PY{n}{fig}\PY{o}{.}\PY{n}{add\PYZus{}subplot}\PY{p}{(}\PY{l+m+mi}{122}\PY{p}{)}
\PY{n}{t}\PY{o}{=}\PY{n}{model}\PY{o}{.}\PY{n}{nonparam\PYZus{}estim}\PY{o}{.}\PY{n}{grid}\PY{o}{.}\PY{n}{partition}\PY{p}{[}\PY{l+m+mi}{2}\PY{p}{:}\PY{o}{\PYZhy{}}\PY{l+m+mi}{21}\PY{p}{]}
\PY{n}{ghat}\PY{o}{=}\PY{n}{model}\PY{o}{.}\PY{n}{nonparam\PYZus{}estim}\PY{o}{.}\PY{n}{g\PYZus{}hat\PYZus{}at\PYZus{}gridpnts}\PY{p}{[}\PY{l+m+mi}{0}\PY{p}{,}\PY{l+m+mi}{0}\PY{p}{,}\PY{l+m+mi}{0}\PY{p}{,}\PY{l+m+mi}{2}\PY{p}{:}\PY{o}{\PYZhy{}}\PY{l+m+mi}{20}\PY{p}{]}
\PY{n}{ax1}\PY{o}{.}\PY{n}{plot}\PY{p}{(}\PY{n}{t}\PY{p}{,}\PY{n}{ghat}\PY{p}{)}
\PY{n}{ax1}\PY{o}{.}\PY{n}{set\PYZus{}title}\PY{p}{(}\PY{l+s+s1}{\PYZsq{}}\PY{l+s+s1}{g\PYZus{}hat on a subinterval of its definition grid}\PY{l+s+s1}{\PYZsq{}}\PY{p}{)}
\PY{n}{plt}\PY{o}{.}\PY{n}{show}\PY{p}{(}\PY{p}{)}
\end{Verbatim}
\end{tcolorbox}

    \begin{center}
    \adjustimage{max size={0.9\linewidth}{0.9\paperheight}}{ghatnptest.png}
    \end{center}
    { \hspace*{\fill} \\}
    
    \begin{tcolorbox}[breakable, size=fbox, boxrule=1pt, pad at break*=1mm,colback=cellbackground, colframe=cellborder]
\prompt{In}{incolor}{11}{\boxspacing}
\begin{Verbatim}[commandchars=\\\{\}]
\PY{n}{fig}\PY{o}{=}\PY{n}{plt}\PY{o}{.}\PY{n}{figure}\PY{p}{(}\PY{n}{figsize}\PY{o}{=}\PY{p}{(}\PY{l+m+mi}{15}\PY{p}{,}\PY{l+m+mi}{5}\PY{p}{)}\PY{p}{)}
\PY{n}{ax}\PY{o}{=}\PY{n}{fig}\PY{o}{.}\PY{n}{add\PYZus{}subplot}\PY{p}{(}\PY{l+m+mi}{121}\PY{p}{)}
\PY{n}{t}\PY{o}{=}\PY{n}{model}\PY{o}{.}\PY{n}{nonparam\PYZus{}estim}\PY{o}{.}\PY{n}{quadrature}\PY{o}{.}\PY{n}{partition}\PY{p}{[}\PY{p}{:}\PY{o}{\PYZhy{}}\PY{l+m+mi}{51}\PY{p}{]}
\PY{n}{k}\PY{o}{=}\PY{n}{model}\PY{o}{.}\PY{n}{nonparam\PYZus{}estim}\PY{o}{.}\PY{n}{hawkes\PYZus{}kernel}\PY{o}{.}\PY{n}{values\PYZus{}at\PYZus{}quadpnts}\PY{p}{[}\PY{l+m+mi}{0}\PY{p}{,}\PY{l+m+mi}{0}\PY{p}{,}\PY{l+m+mi}{0}\PY{p}{,}\PY{p}{:}\PY{o}{\PYZhy{}}\PY{l+m+mi}{50}\PY{p}{]}
\PY{n}{ax}\PY{o}{.}\PY{n}{plot}\PY{p}{(}\PY{n}{t}\PY{p}{,}\PY{n}{k}\PY{p}{)}
\PY{n}{ax}\PY{o}{.}\PY{n}{set\PYZus{}title}\PY{p}{(}\PY{l+s+s1}{\PYZsq{}}\PY{l+s+s1}{Estimated hawkes kernel}\PY{l+s+s1}{\PYZsq{}}\PY{p}{)}
\PY{n}{ax1}\PY{o}{=}\PY{n}{fig}\PY{o}{.}\PY{n}{add\PYZus{}subplot}\PY{p}{(}\PY{l+m+mi}{122}\PY{p}{)}
\PY{n}{t}\PY{o}{=}\PY{n}{model}\PY{o}{.}\PY{n}{nonparam\PYZus{}estim}\PY{o}{.}\PY{n}{quadrature}\PY{o}{.}\PY{n}{partition}\PY{p}{[}\PY{l+m+mi}{30}\PY{p}{:}\PY{o}{\PYZhy{}}\PY{l+m+mi}{101}\PY{p}{]}
\PY{n}{k}\PY{o}{=}\PY{n}{model}\PY{o}{.}\PY{n}{nonparam\PYZus{}estim}\PY{o}{.}\PY{n}{hawkes\PYZus{}kernel}\PY{o}{.}\PY{n}{values\PYZus{}at\PYZus{}quadpnts}\PY{p}{[}\PY{l+m+mi}{0}\PY{p}{,}\PY{l+m+mi}{0}\PY{p}{,}\PY{l+m+mi}{0}\PY{p}{,}\PY{l+m+mi}{30}\PY{p}{:}\PY{o}{\PYZhy{}}\PY{l+m+mi}{100}\PY{p}{]}
\PY{n}{ax1}\PY{o}{.}\PY{n}{plot}\PY{p}{(}\PY{n}{t}\PY{p}{,}\PY{n}{k}\PY{p}{)}
\PY{n}{ax1}\PY{o}{.}\PY{n}{set\PYZus{}title}\PY{p}{(}\PY{l+s+s1}{\PYZsq{}}\PY{l+s+s1}{Estimated hawkes kernel restricted to a subinterval}\PY{l+s+s1}{\PYZsq{}}\PY{p}{)}
\PY{n}{plt}\PY{o}{.}\PY{n}{show}\PY{p}{(}\PY{p}{)}
\end{Verbatim}
\end{tcolorbox}

    \begin{center}
    \adjustimage{max size={0.9\linewidth}{0.9\paperheight}}{resnptest.png}
    \end{center}
    { \hspace*{\fill} \\}

\newpage 

\subsection{Rejection sampling}\label{sec.pipest.rejection_sampling}

Let $n$ denote the number of levels in the limit order book. Let $\dirparam=(\dirparam_{0},\dots,\dirparam_{2n-1})$ be a $2n$-dimensional vector with strictly positive components; it accounts for the Dirichlet parameter of the distribution of the vector $(\nthBestAskVolume[1],\nthBestBidVolume[1],\dots, \nthBestAskVolume[n],\nthBestBidVolume[n])$ of normalised ask/bid volumes on the first $n$ levels. Hence, $\dirparam_{2k}$ refers to the ask volume on the $k+1$-th level, for $k=0,\dots,n-1$, and 
$\dirparam_{2k+1}$ refers to the bid volume on the $k+1$-th level, for $k=0,\dots,n-1$. 

Recall that a $2n$-dimensional Dirichlet distribution is concentrated on the set $\lbrace v \in [0,1]^{2n}: \, \, \sum_{i=0}^{2n-1} v_i = 1 \rbrace$. Moreover, given $v=(v_0, \dots,v_{2n-1})$ with  $v_0 + \dots + v_{2n-1} = 1$, the value $\dirichletDensity(v)$ in $v$  of the Dirichlet distirbution with parameter $\dirparam$ is $\dirichletDensity(v) \propto v_{0}^{\dirparam_{0}} \cdot \dots \cdot v_{2n-1}^{\dirparam_{2n-1}}$, where  the normalisation constant is the multivariate beta function $B(\dirparam)$.



Let $V=(V_{0}, \dots,V_{2n-1})$ be a Dirichlet random variable with parameter $\dirparam$. Let $\dirichletLaw$ be the law of $V$, and let $\Expectation_{\dirparam}$ denote expectation under this law. Let $\dirichletDensity$ be the Dirichlet density.  Given the lowerbound $\ell$ and the upperbound $u$ in $-1\leq \ell < u \leq 1$, let $Q=Q_{\ell, u}$ be the set 
\begin{equation}\label{eq.volimb_constraint}
 Q_{\ell,u}:= \left\lbrace
 \ell \leq \sum_{k=0}^{n-1} \left( V_{2k+1} - V_{2k} \right)
 < u
 \right\rbrace. 
\end{equation}
Then, the target distribution for the rejection sampling is $\dirichletLaw(\cdot \vert Q_{\ell,u})$, and the target density is 
\begin{equation}\label{eq.target_density}
 \targetDensity(v) = 
 \frac{\dirichletDensity(v)\one _{Q_{\ell,u}} (v)}{\dirichletLaw\left(Q_{\ell,u}\right)}. 
\end{equation}


\begin{figure}\label{fig.dirichletAccept}
 \centering
 \includegraphics[width=0.4\textwidth]{dirichletExample.png}
\end{figure}



The following proposition casts the general rejection principle to the case where the target density is $f$ and the proposal density is the Dirichlet density $\dirichletDensity[\rho]$ with parameter $\rho\in\R_{+}^{2n}$.  


\begin{prop}\label{prop.rejection_sampling}
 Let $\dirparam$, $\rho$ in $\R_{+}^{2n}$ and $-1\leq \ell < u \leq 1$ be as above. Assume that $\rho_i\leq \dirparam_i$ for all $i=0,\dots,2n-1$. 
 Let $\lbrace V(t): \, \, t=1,2,\dots \rbrace$ be a sequence of i.i.d. random variables with distribution $\text{Dir}_{\rho}$. Let $\lbrace U(t): \, \, t=1,2,\dots \rbrace$ be a sequence of i.i.d. random variables with uniform distribution on the unit interval. Define the stopping time 
 \begin{equation}\label{eq.stoppingTimeReject}
  \tau := \inf \left\lbrace 
  t=1,2,\dots : \, \, 
  U(t)<\frac{1}{M} \prod_{i=0}^{2n-1} V_{i}^{\dirparam_i - \rho_i}(t)
  \one_{Q_{\ell,u}}\left\lbrace V(t)\right\rbrace
  \right\rbrace,
 \end{equation}
where $Q_{\ell,u}$ is as in equation \eqref{eq.volimb_constraint} and $M$ is an upperbound for $\prod_i v_{i}^{\dirparam_i - \rho_i}\one_{Q\subscriptellu}(v)$, with $v$ in the the support of $\dirichletDensity$.
Then, $V(\tau)$ has absolutely continuous distribution with density $\targetDensity$ given in equation \eqref{eq.target_density}.
\end{prop}
\begin{proof}
 Let $U$ and $V$ be independent random variables such that  $U$ is uniformly distributed in $[0,1]$, and $V \sim \text{Dir}_{\rho}$. Let $A$ be the event 
 \begin{equation*}
  A  =
  \left\lbrace
  U < \frac{1}{M} \prod_{i=0}^{2n-1} V_{i}^{\dirparam_i - \rho_i} \one_{Q_{\ell,u}}(V)  
  \right\rbrace,
 \end{equation*}
 where $M$ is as in the statement. Let $\tilde{M}:= B(\rho) M / (\Prob_{\dirparam}(Q_{\ell, u}) B(\dirparam) )$, where $B(\dirparam)$ and  $B(\rho)$ are the normalisation constants of $\dirichletDensity[\dirparam]$ and $\dirichletDensity[\rho]$ respectively.  For $0\leq s \leq 1$ and $v=(v_0,\dots,v_{2n-1}) \in (0,1)^{2n}$ such that $v_0 + \dots, + v_{2n-1} = 1 $, the condition $ M s < \prod_i v_{i}^{\dirparam_i - \rho_i} \one_{Q}(v)$ is equivalent to $\dirichletDensity[\rho](v) \tilde{M} s  < \targetDensity(v)$. Hence, 
 \begin{equation*}
  \begin{split}
   \Prob(A) = &
   \int dv \, \, \dirichletDensity[\rho](v)
   \int_{0}^{1}ds\, \, 
   \one\left\lbrace
   M s < \prod_i v_{i}^{\dirparam_i - \rho_i} \one_{Q_{\ell,u}}(v)
   \right\rbrace
   \\
   = &
   \int dv \,\, \dirichletDensity[\rho](v)
   \int_{0}^{1}ds \, \, 
   \one\left\lbrace
   \dirichletDensity[\rho](v) \tilde{M} s  < \frac{\dirichletDensity(v)\one _{Q_{\ell,u}} (v)}{\dirichletLaw\left(Q_{\ell,u}\right)}
   \right\rbrace
   \\
   = &
   \frac{1}{\tilde{M}}.
  \end{split}
 \end{equation*}
Therefore, for any arbitrary continuous and bounded real function $\phi$ defined on $[0,1]^{2n}$, we have
\begin{equation*}
 \begin{split}
  \Expectation\left[ \phi\left( V(\tau) \right)\right]
  = &
  \sum_{t=1}^{\infty} \Expectation\left[ \phi\left( V(t)\right) \one \lbrace \tau = t \rbrace \right]
  \\
  = & 
  \sum_{t=1}^{\infty} \big(1-\Prob(A)\big)^{t-1}\Expectation\left[ \phi(V)\one_{A}\right]
  \\
  = &
  \frac{1}{\Prob(A)}
  \int dv \,\,\phi(v) \dirichletDensity[\rho](v)
   \int_{0}^{1}ds \, \, 
   \one\left\lbrace
   \dirichletDensity[\rho](v) \tilde{M} s  < \targetDensity(v)
   \right\rbrace
   \\
   = &
   \int dv \,\, \phi(v)  \targetDensity(v).
 \end{split}
\end{equation*}
\end{proof}



\begin{prop}\label{prop.expectedWaitingTime}
 Let $\dirparam$, $\rho$, $\ell$, $u$, and $M$ be as in Proposition \ref{prop.rejection_sampling}. Let $\tau$ be as in equation \eqref{eq.stoppingTimeReject}.  Let $V\sim\dirichlet[\rho]$, and let $U$ be independent from $V$ and uniformly distributed on the unit interval.
 Then, 
 \begin{equation}\label{eq.expectedWaitingTime}
  \Expectation[\tau]
  =
  \frac{1}{\Prob(A)},
 \end{equation}
where $A:=\lbrace U\leq \frac{1}{M}\prod_{i=0}^{2n-1}V_{i}^{\dirparam_i - \rho_i}\one_{Q\subscriptellu}(V)\rbrace$.
\end{prop}
\begin{proof}
We can compute
 \begin{equation*}
  \Expectation[\tau]
  =
  \sum_{t=1}^{\infinity}t\Prob(A)\big(1-\Prob(A)\big)^{t-1}
  =
  -\Prob(A)\frac{d}{dx}\Big\vert_{x=\Prob(A)} \sum_{t=0}^{\infinity} (1-x)^{t}.
 \end{equation*}
\end{proof}

Our task is to optimally choose the parameter $\rho$  of the proposal density $\dirichletDensity[\rho]$  and the bound $M$ in Proposition \ref{prop.rejection_sampling} in order to minimise the expected waiting time in equation \eqref{eq.expectedWaitingTime}. 

\begin{lemma}\label{lemma.equivalence_expected_constraint}
 Let $V$ in $\R^{2n}$ be a random vector with Dirichlet distribution $\text{Dir}_{\dirparam}$ of parameter $\dirparam \in \R^{2n}_{+}$. Let  $-1\leq \ell < u \leq 1$. Then, the following are equivalent 
 \begin{enumerate}[label={\roman{*}.} , ref={\ref{lemma.equivalence_expected_constraint}.\roman{*}}]
  \item \label{lemma.equivalence_expected_constraint.expectation}$
  \ell \leq \Expectation_{\dirparam} \sum_{k=0}^{n-1} \left( V_{2k+1} - V_{2k} \right) < u $;
  \item \label{lemma.equivalence_expected_constraint.parameter} $\ell \sum_{i=0}^{2n-1}\dirparam_i 
  \leq 
  \sum_{k=0}^{n-1} (\dirparam_{2k+1} - \dirparam_{2k} ) 
  < u \sum_{i=0}^{2n-1}\dirparam_i
  $.
 \end{enumerate}
\end{lemma}
\begin{proof}
 Straightforward from $\Expectation_{\dirparam}[V_{k}] = \dirparam_{k}/\sum_{i=0}^{2n-1} \dirparam_{i}$.
\end{proof}

\begin{lemma}\label{lemma.bound_on_dirratio}
 Let  $-1\leq \ell < u \leq 1$. Let $\dirparam$ and $\rho$ be in $\R_{+}^{2n}$ and consider the Drichlet densities $\dirichletDensity[\dirparam]$ and $\dirichletDensity[\rho]$. Assume that $\rho_i < \dirparam_i$ for all $i=0,\dots,2n-1$ and set $\alpha=\dirparam-\rho$.  Then,
\begin{enumerate}[label={\roman{*}.}, ref={\ref{lemma.proposal_dirparam}.\roman{*}}]
  \item the maximiser $\hat{v}$ of $v\mapsto \dirichletDensity[\dirparam](v)/\dirichletDensity[\rho](v)$ is given by 
 $\hat{v} = (\hat{v}_0 , \dots,  \hat{v}_{2n-1})$, where $\hat{v}_{i} = \alpha_{i}/\sum_{k} \alpha_k$ for all $i=0,\dots,2n-1$;
 \item the maximum $M\subscriptellu$ of $v\mapsto \dirichletDensity[\dirparam](v)\one_{Q\subscriptellu}(v)/\dirichletDensity[\rho](v)$ is given by 
 \begin{equation}\label{eq.Mellu}
  M\subscriptellu
  =
  \exp\left(
  \max\left\lbrace
  \sum_{i=0}^{2n-1} \alpha_i \log v_i^{(\ell)},
  \sum_{i=0}^{2n-1} \alpha_i \log v_i^{(u)}
  \right\rbrace
  \right),
 \end{equation}
where for any real number $c\neq 0$ we have set
\begin{equation*}
 \begin{split}
    K_c :=& \frac{\sum \alpha_i - \frac{1}{c}\sum(\alpha_{2i+1} - \alpha_{2i})}{\sum (\alpha_{2i+1}-\alpha_{2i}) - \frac{1}{c} \sum \alpha_i},
   \\
   Z_c :=& (1-K_c) \sum \alpha_{2i+1} + (1+K_c)\sum \alpha_{2i},
   \\
   v_{2i+1}^{(c)}:=& (1-K_c)\alpha_{2i+1}/Z_c, \qquad i=0,\dots,n-1,
   \\
   v_{2i}^{(c)}:=& (1+K_c)\alpha_{2i}/Z_c, \qquad i=0,\dots,n-1.
 \end{split}
\end{equation*}
\end{enumerate}
\end{lemma}
\begin{proof}
As for the first claim, notice that $(\dirichletDensity[\dirparam]/\dirichletDensity[\rho]) (x_0,\dots,x_{2n-2},v_{2n-1})$ is proportional to $\phi:\,  [0,1]^{2n-1} \mapsto \R$ given by 
\begin{equation*}
 \phi(x_0,\dots,x_{2n-2}) = 
 \left( \prod_{i=0}^{2n-2} x_{i}^{\alpha_i}\right)
 \left( 1-\sum_{i=0}^{2n-2} x_i \right)^{\alpha_{2n-1}}
 \one\left\lbrace \sum_{i=0}^{2n-2} x_i < 1 \right\rbrace,
\end{equation*}
where $\alpha_i = \dirparam_i  - \rho_i$ for all $i=0,\dots,2n-1$. Within the open set $\lbrace x \in [0,1]^{2n-1}: \, \, \sum_{i=0}^{2n-2} x_i < 1 \rbrace$ the $k$-th partial derivative of $\phi$ is 
\begin{equation*}
 \partial_k \phi(x) = 
 x_{k}^{\alpha_k -1 } 
 \left( \prod_{i\neq k } x_{i}^{\alpha_i}\right)
 \left( 1-\sum_{i=0}^{2n-2} x_i \right)^{\alpha_{2n-1}-1}
 \left[
 \alpha_k \left( 1-\sum_{i=0}^{2n-2} x_i \right) - \alpha_{2n-1} x_k
 \right],
\end{equation*}
and thus $x$ in $[0,1]^{2k-1}$ is stationary for $\phi$ if and only if 
\begin{equation*}
 \alpha_k \left( 1-\sum_{i=0}^{2n-2} x_i \right) - \alpha_{2n-1} x_k = 0
\end{equation*}
for all $k=0,\dots,2n-2$, or equivalently if and only if  
\begin{equation*}
 \sum_{i\neq k} x_i + \frac{\alpha_k + \alpha_{2k-1}}{\alpha_k}x_k = 1
\end{equation*}
for all $k=0,\dots,2n-2$. The latter yields the $(2n-1)\times(2n-1)$-linear system $\Gamma x = \underbar{1}$, where $\underbar{1} = (1,\cdots,1)\transpose$ and 
 \begin{equation*}
  \Gamma = 
  \begin{pmatrix}
   \frac{\alpha_{0} + \alpha_{2n-1}}{\alpha_0}
   &   1   &   \cdots   &   1
   \\
   1 &
   \frac{\alpha_{1} + \alpha_{2n-1}}{\alpha_1}
   & \cdots & 1 
   \\
   \vdots & & \ddots & \vdots\\
   1 &\cdots & 1 &
   \frac{\alpha_{2n-2} + \alpha_{2n-1}}{\alpha_{2n-2}}
  \end{pmatrix},
 \end{equation*}
 where $\alpha_i = \dirparam_i  - \rho_i$ for all $i=0,\dots,2n-1$. Solving this system of linear equations gives the claimed maximiser.
 
As for the second claim, we discuss 
\begin{equation*}
 \begin{split}
  &
  \text{maximise }\quad f(v)=\prod_{i=0}^{2n-1}v_{i}^{\alpha_i}
  \\
  &
  \text{subject to}\quad 
  \begin{cases}
   &
   \sum_{i=0}^{2n-1} v_i = 1
   \\
   &
   \sum_{i=0}^{n-1}(v_{2i+1}-v_{2i}) \leq u
   \\
   &
   \sum_{i=0}^{n-1}(v_{2i+1}-v_{2i}) \geq \ell
  \end{cases}
 \end{split}
\end{equation*}
using Lagrangian multipliers. Let $\mu > 0 $ and $\lambda_1, \lambda_1 \geq 0$ and define
\begin{equation*}
 \begin{split}
  L(v_0,\dots,v_{2n-1},\mu,\lambda_1,\lambda_2)
  :=
  &
  \sum_{i=0}^{2n-1} \alpha_{i}\log v_{i} - \mu \left(\sum_{i=0}^{2n-1} v_{i} -1\right)
  \\
  &
  -\lambda_1\left(\sum_{i=0}^{n-1}(v_{2i+1}-v_{2i}) - u\right)
  -\lambda_2\left(\sum_{i=0}^{n-1}(v_{2i}-v_{2i+1}) - \ell\right). 
 \end{split}
\end{equation*}
From the stationarity condition $\nabla_{v} L =0$ it follows
\begin{equation}\label{eq.conditionLagrangianPnt}
 \begin{split}
  &
  \alpha_{2i+1} - (\mu+\lambda_1 - \lambda_2)v_{2i+1} = 0
  \\
  &
  \alpha_{2i} - (\mu - \lambda_1 + \lambda_2)v_{2i} = 0
 \end{split}
\end{equation}
for $i=0,\dots,n-1$. From this, by using the equality constraint $\sum_i v_i = 1$ we derive the condition 
\begin{equation}\label{eq.conditionLagrangianMultipl}
 \mu\squared - \mu \sum_{i=0}^{2n-1}\alpha_i 
 +(\lambda_1-\lambda_2)\sum_{i=0}^{n-1}(\alpha_{2i+1}-\alpha_{2i})
 -(\lambda_1-\lambda_2)\squared=0
\end{equation}
For $c>0$, let $A_c$ be as in the statement. Observe that $A_c =1$ if and only if $\sum_{i=0}^{n-1}\alpha_{2i} =0$, and $A_c =-1$ if and only if $\sum_{i=0}^{n-1}\alpha_{2i+1} =0$. Since $\alpha_i>0$ for all $i=0,\dots,2n-1$ we conclude tha $A_c\squared \neq 1$. 

Assume that the first inequality constraint 
   $\sum_{i=0}^{n-1}(v_{2i+1}-v_{2i}) \leq u$
is binding, namely that equality holds. Then, necessarily the second inequality constraint is non-binding and $\lambda_2=0$. Hence, we compute
\begin{equation*}
 u=\sum_{i=0}^{n-1}(v_{2i+1}-v_{2i})
 =
 \frac{1}{\mu+\lambda_1}\sum_{i=0}^{n-1}\alpha_{2i+1}
 -\frac{1}{\mu-\lambda_1}\sum_{i=0}^{n-1}\alpha_{2i},
\end{equation*}
which yields
\begin{equation*}
 \mu\squared - \frac{\mu}{u}\sum_{i=0}^{n-1}(\alpha_{2i+1}-\alpha_{2i})
 +\frac{\lambda_1}{u}\sum_{i=0}^{2n-1}\alpha_i - \lambda_{1}\squared=0.
\end{equation*}
Putting the latter in system with equation \eqref{eq.conditionLagrangianMultipl} gives
\begin{equation*}
 \begin{cases}
  &
  (1-A_{u}\squared)\mu=\sum_{i=0}^{2n-1}\alpha_i - A_{u}\sum_{i=0}^{n-1}(\alpha_{2i+1}-\alpha_{2i})
  \\
  &
  \lambda_1 = A_u \mu.
 \end{cases}
\end{equation*}
Using these values for the multipliers (and $\lambda_2=0$) in equation \eqref{eq.conditionLagrangianPnt} we derive the formulae for $v^{(u)}_{i}$, $i=0,\dots,2n-1$. The case where instead the second inequality is binding is treated in the same way and gives the values of $v^{(\ell)}_{i}$, for $i=0,\dots,2n-1$.
\end{proof}

\begin{prop}\label{prop.proposal_param}
 Let $U$ in $\R$ and $V$ in $\R^{2n}$ be independent random variables. Assume that $U$ is uniformly distributed in the unit interval.  Let $\dirichletLaw[\rho]$ be the joint law of $(U,V)$ when $V$ has Dirichlet distirbution with parameter $\rho$. Let $A$ be the event
 \begin{equation*}
  A:=\left\lbrace
  U
  \leq
  \frac{1}{M\subscriptellu}
  \prod V_{i}^{\dirparam_i - \rho_i}\one_{Q\subscriptellu}(V)
  \right\rbrace,
 \end{equation*}
 where $M\subscriptellu$ is as in equation \eqref{eq.Mellu}. Consider $M\subscriptellu = M\subscriptellu(\rho)$ as a function of $\rho$.  Then,
 \begin{equation*}
  \argmax\left\lbrace
  \dirichletLaw[\rho](A): \, 0<\rho_i\leq \dirparam_i
  \right\rbrace
  =
  \argmin\left\lbrace
  \log B(\rho)
  + \log M\subscriptellu (\rho)
  : \, 0<\rho_i\leq \dirparam_i
  \right\rbrace,
 \end{equation*}
 where $B=B(\rho)$ is the multivariate beta function. 
\end{prop}
\begin{proof}
It suffices to notice that
 \begin{equation*}
  \dirichletLaw[\rho](A)
  =
     \int dv \, \, \dirichletDensity[\rho](v)
   \int_{0}^{1}ds\, \, 
   \one\left\lbrace
   M\subscriptellu(\rho) \,  s < \prod_i v_{i}^{\dirparam_i - \rho_i} \one_{Q_{\ell,u}}(v)
   \right\rbrace
  =
  \frac{B(\dirparam)}{B(\rho)}
  \frac{\dirichletLaw(Q\subscriptellu)}{M\subscriptellu(\rho)}.
 \end{equation*}
\end{proof}
\begin{remark}
 Notice that $\log M\subscriptellu (\dirparam)=0$ and $\dirichletLaw(A)=\dirichletLaw(Q\subscriptellu)$, guaranteeing consistency with the naive samplig procedure from $f$. 
\end{remark}


\begin{algorithm}[h]
 \caption{Definition of the parameter $\rho$ of the proposal density $\proposalDensity = \dirichletDensity[\rho]$}
 \label{algo.set_proposal_param}
 \begin{algorithmic}
  \REQUIRE $\ell$, $u$, $(\dirparam_i)_{i=0,\dots,2n-1}$ 
  \IF { \begin{equation*}\ell \sum_{i=0}^{2n-1}\dirparam_i \leq \sum_{k=0}^{n-1} (\dirparam_{2k+1} - \dirparam_{2k} ) < u \sum_{i=0}^{2n-1}\dirparam_i
  \qquad\qquad\qquad\qquad\qquad\qquad
  \end{equation*} }
  \STATE \begin{equation*}
  \begin{split}
  &\text{set } \quad \rho_i = \dirparam_i \qquad \text{for all } i=0,\dots,2n-1;
  \qquad\qquad\qquad\qquad\qquad\qquad
  \\
  &\text{set } \quad M:=1;
 \end{split}  
  \end{equation*}
  \ELSE
  \STATE 
   \begin{equation*}
   \begin{split}
  &\text{set } \quad \rho = \argmin\lbrace \log B(r) + \log M\subscriptellu(r): r \in \R^{2n}, \, 0<r_i\leq \dirparam_i \forall i \rbrace
  \\
  &
  \text{set } \quad M =  M\subscriptellu(\rho)
  \end{split}
  \end{equation*}
  \ENDIF
  \RETURN $M$, $(\rho_i)_{i=0,\dots,2n-1}$.
 \end{algorithmic}
\end{algorithm}

\begin{algorithm}[h!]
 \caption{Accept-reject algorithm for drawing from $\targetDensity$ of equation \eqref{eq.target_density}: case 1}
 \label{algo.rejection_sampling_1}
 \begin{algorithmic}[5]
  \REQUIRE $\ell$, $u$, $(\dirparam_i)_{i=0,\dots,2n-1}$
  \STATE set $\xi = 1$
  \WHILE{$\xi = 1$}
  \STATE draw $V\sim\dirichlet[\dirparam]$
  \IF {$\ell \leq \sum_{k=0}^{n-1} (V_{2k+1} - V_{2k}) < u$}
  \STATE set $\xi=0$
  \ENDIF
  \ENDWHILE
  \RETURN $V$.
 \end{algorithmic}
\end{algorithm}



\begin{algorithm}[h!]
 \caption{Accept-reject algorithm for drawing from $\targetDensity$ of equation \eqref{eq.target_density}: case 2}
 \label{algo.rejection_sampling_2}
 \begin{algorithmic}[5]
  \REQUIRE $\ell$, $u$,  $(\rho_i)_{i=0,\dots,2n-1}$, $M$
  \STATE set $\xi = 1$
  \WHILE{$\xi = 1$}
  \STATE draw $V\sim\dirichlet[\rho]$
  \IF {$\ell \leq \sum_{k=0}^{n-1} (V_{2k+1} - V_{2k}) < u$}
  \STATE draw $U\sim \text{Unif}[0,1]$
  \IF {$U \leq  \prod_{i} V_{i}^{\dirparam_i-\rho_i}/M$}
  \STATE set $\xi = 0$
  \ENDIF
  \ENDIF
  \ENDWHILE
  \RETURN $V$.
 \end{algorithmic}
\end{algorithm}


\newpage

\subsection{One-sided impact profile}
\begin{lemma}\label{lemma.implementation_of_one-sided_impact}
 In the setting of Proposition \ref{prop.impact_poisson_and_constant-size}, the difference $\weakDeflCompensator(t) - \intzerot \Expectation[\weakDeflIntensity(s)\,\vert\,\sigmaAlgebraG_{0}\orthogonal(s)]ds$ can be evaluatated as per the formula
 \begin{equation}\label{eq.evaluation_of_one-sided_impact}
 \begin{split}
  (\terminationTime\wedge t) &\baseRate_{0}
  - \left\lceil \frac{\metaorderSize}{q_{M}}\right\rceil
  + \sum_{n=0}^{\lceil\metaorderSize/q_{M}\rceil -1}
  \left( \left\lceil \frac{\metaorderSize}{q_{M}}\right\rceil - n\right)\frac{(\baseRate_{0} t)^n}{n!} \exp \Big( - \baseRate_{0} t \Big)
  \\
  &+
  \sum_{x\in\weakDeflSpace}\sum_{e=1}^{4}\sum_{\xone=1}^{3K}
  \frac{\impCoef_{1,\xone,e}}{\decCoef_{1,\xone,e} -1}
  \sum_{0\leq \arrivalTimes[]_n < t}
  \Big(
  f^{(n)}_{\xone,e,x}(t) - g^{(n)}_{\xone,e,x}(t) \one\lbrace \arrivalTimes[]_n >0\rbrace
  \Big),
  \end{split}
 \end{equation}
where 
\begin{equation*}
\begin{split}
 f^{(n)}_{\xone,e,x} (t)
 =&
 \transProb_{e} (\stateVariable(\arrivalTimes[]_n),x)
 \Bigg\lbrace
 \sum_{0\leq \arrivalTimes[0,\xone]_j < \arrivalTimes[]_n}
 \left[
 (\arrivalTimes[]_n - \arrivalTimes[0,\xone]_j +1)^{1-\decCoef_{1,\xone,e}}
 -
 (\arrivalTimes[]_{n+1}\wedge t - \arrivalTimes[0,\xone]_j +1)^{1-\decCoef_{1,\xone,e}}
 \right]
 \\
 &-
 \baseRate_{0} \sum_{0\leq \arrivalTimes[]_j < \arrivalTimes[]_n}
 \transProb_{0}(\stateVariable(\arrivalTimes[]_j,\xone)
 \int_{\arrivalTimes[]_j}^{\arrivalTimes[]_{j+1}}
 du \, \,  \Prob(\terminationTime>u)
 \left[
 (\arrivalTimes[]_n - u +1)^{1-\decCoef_{1,\xone,e}}
 -
 (\arrivalTimes[]_{n+1}\wedge t - u +1)^{1-\decCoef_{1,\xone,e}}
 \right]
 \Bigg\rbrace
 \end{split}
\end{equation*}
and 
\begin{equation*}
 g^{(n)}_{\xone,e,x} (t) = 
 \baseRate_{0} \transProb_{0} (\stateVariable(\arrivalTimes[]_{n}),\xone)\transProb_{e} (\stateVariable(\arrivalTimes[]_{n-1}),x)
 \int_{\arrivalTimes[]_n}^{\arrivalTimes[]_{n+1}\wedge t}
 du \Prob(\terminationTime>u)
 \left[
 1 -
 (\arrivalTimes[]_{n+1} - u +1)^{1-\decCoef_{1,\xone,e}}
 \right].
\end{equation*}
\end{lemma}
\begin{proof}
We start by re-writing the right hand side of equation \eqref{eq.explicit_numerator_of_impProfile} as follows
\begin{equation}\label{eq.proof_implementation_one-sided_impact}
 \weakDeflCompensator(t) - \intzerot \Expectation[\weakDeflIntensity(s)\,\vert\,\sigmaAlgebraG_{0}\orthogonal(s)]ds
 = I + II + III,
\end{equation}
where
\begin{equation*}
\begin{split}
 I =& 
 \big(t\wedge \terminationTime - \Expectation [t\wedge\terminationTime] \big) \baseRate_{0} \, ;
 \\
 II =&
 \sum_{x\in\weakDeflSpace}\sum_{e=1}^{4}\sum_{\xone}
 \intzerot  ds \, \transProb_{e}(\stateVariable(s),x)
 \int_{0}^{s}\hawkesKernel_{1,e}(s-u,\xone) 
 d\hybridHawkes_{0,\xone}(u)\,\, ;
 \\
 III = &
 -\baseRate_{0}\sum_{x\in\weakDeflSpace}\sum_{e=1}^{4}\sum_{\xone}
 \intzerot ds \, \transProb_{e}(\stateVariable(s),x)
 \int_{[0,s)}
 \hawkesKernel_{1,e} (s-u,\xone) \transProb_0(\stateVariable(u),\xone)\Prob(\terminationTime>u) du.
\end{split}
\end{equation*}

Summand $I$ is evaluated by relying on equation \eqref{eq.expected_terminationTime} in  Proposition \ref{prop.terminationTime_poisson_and_constant-size}. 

As for summand $II$, apply Fubini to swap the order of integration and obtain 
\begin{equation*}
 II =
 \sum_{x\in\weakDeflSpace}\sum_{e=1}^{4}\sum_{\xone}
 \intzerot d\hybridHawkes_{0,\xone}(u) 
 \int_{u}^{t}ds \, \transProb_{e}(\stateVariable(s),x)\hawkesKernel_{1,e}(s-u,\xone) .
\end{equation*}
Now, consider the innermost integration with respect to $ds$. By partitioning the integration interval $[u,t]$ into the interarrival sub-intervals $[\arrivalTimes[]_{n}, \arrivalTimes[]_{n+1}]$, we can perform exact integration by computing primitives of $t\mapsto\hawkesKernel_{1,e}(t,\xone)$. This yields
\begin{equation*}
\begin{split}
 \int_{u}^{t} ds \, \, \transProb_{e} (\stateVariable(s),x) \hawkesKernel_{1,e}(s-u,\xone)
 = &
 \sum_{u\leq \arrivalTimes[]_n < t} \transProb_{e} (\stateVariable_n, x)
 \int_{\arrivalTimes[]_n}^{\arrivalTimes[]_{n+1}\wedge t} ds \,\, \hawkesKernel_{1,e}(s-u,\xone)
 \\
 &+
 \transProb_{e} (\stateVariable(u-), x)
 \int_{u}^{t_u \wedge t} ds \,\, \hawkesKernel_{1,e}(s-u,\xone)
 \\
 =&
 \sum_{u\leq \arrivalTimes[]_n < t} \transProb_{e}(\stateVariable_n, x) \frac{\impCoef_{1,\xone,e}}{\decCoef_{1,\xone,e} - 1}
 \Big(
 (\arrivalTimes[]_n - u +1)^{1-\decCoef_{1,\xone,e}}
  - (\arrivalTimes[]_{n+1}\wedge t - u +1)^{1-\decCoef_{1,\xone,e}}  
 \Big)
 \\
 &+
 \transProb_{e}(\stateVariable(u-), x) \frac{\impCoef_{1,\xone,e}}{\decCoef_{1,\xone,e} - 1}
 \Big(
 1  - (t_u\wedge t - u +1)^{1-\decCoef_{1,\xone,e}}
 \Big),
 \end{split}
\end{equation*}
where $t_u:= \inf\lbrace \arrivalTimes[]_n \geq u\rbrace$ is the first arrival time greater than or equal to $u$. When the latter is integrated in $d\hybridHawkes_{0,\xone}(u)$ from $0$ to $t$, we have that the second summand on the right hand side gives no contribution because $\hybridHawkes_{0,\xone}$ is constant from $u$ to $t_u$. Hence, we are left with 
\begin{equation*}
\begin{split}
 \frac{\impCoef_{1,\xone,e}}{\decCoef_{1,\xone,e} - 1}
 \intzerot d\hybridHawkes_{0,\xone}(u)
 &
 \sum_{u\leq \arrivalTimes[]_n < t} \transProb_{e}(\stateVariable_n, x) 
 \Big(
 (\arrivalTimes[]_n - u +1)^{1-\decCoef_{1,\xone,e}}
  - (\arrivalTimes[]_{n+1}\wedge t - u +1)^{1-\decCoef_{1,\xone,e}}  
 \Big)
 \\
 =&
 \frac{\impCoef_{1,\xone,e}}{\decCoef_{1,\xone,e} - 1}
 \sum_{0\leq \arrivalTimes[]_n < t}
 \int_{[0,\arrivalTimes[]_n)} d\hybridHawkes_{0,\xone}(u)
 \transProb_{e}(\stateVariable_n, x) 
 \\
 &
 \cdot 
 \Big(
 (\arrivalTimes[]_n - u +1)^{1-\decCoef_{1,\xone,e}}
  - (\arrivalTimes[]_{n+1}\wedge t - u +1)^{1-\decCoef_{1,\xone,e}}  
 \Big)
 \\
 =&
 \frac{\impCoef_{1,\xone,e}}{\decCoef_{1,\xone,e} - 1}
 \sum_{0\leq \arrivalTimes[]_n < t}
 \transProb_{e}(\stateVariable_n, x)
 \\
 &
 \cdot
 \sum_{0\leq \arrivalTimes[0,\xone]_j < \arrivalTimes[]_n}  
 \Big(
 (\arrivalTimes[]_n - \arrivalTimes[0,\xone]_j +1)^{1-\decCoef_{1,\xone,e}}
  - (\arrivalTimes[]_{n+1}\wedge t - \arrivalTimes[0,\xone]_j +1)^{1-\decCoef_{1,\xone,e}}  
 \Big),
 \end{split}
\end{equation*}
where we used Fubini theorem on the first step. 

As for summand $III$, we follow a computation similar to $II$. Indeed, we first swap integration in $ds$ and in $du$, we partition the integration in $ds$ over $[0,t]$ into the integration over the subintervals $[\arrivalTimes[]_n , \arrivalTimes[]_{n+1}]$, and we get 
\begin{equation*}
 \begin{split}
   \frac{\impCoef_{1,\xone,e}}{\decCoef_{1,\xone,e} - 1}
   \intzerot du & \transProb_{0}(\stateVariable(u),\xone) \Prob(\terminationTime > u)
   \\
   & \cdot
   \Bigg[
   \transProb_{e}(\stateVariable(u-),x) \Big( 1- (t_u - u +1)^{1-\decCoef_{1,\xone,e}}\Big)
   \\
   & \qquad 
   + \sum_{u\leq \arrivalTimes[]_n < t} \transProb_{e}(\stateVariable_n,x)\Big( (\arrivalTimes[]_n - u +1)^{1-\decCoef_{1,\xone,e}} - (\arrivalTimes[]_{n+1}\wedge t - u +1)^{1-\decCoef_{1,\xone,e}}\Big)
   \Bigg]
   \\
   =&
   \frac{\impCoef_{1,\xone,e}}{\decCoef_{1,\xone,e} - 1}
   \sum_{0\leq \arrivalTimes[]_n < t} \transProb_{e}(\stateVariable_n, x) \sum_{0\leq \arrivalTimes[]_j < \arrivalTimes[]_n} \transProb_{0}(\stateVariable_j, \xone) \int_{\arrivalTimes[]_{j}}^{\arrivalTimes[]_{j+1}}du \, \Prob(\terminationTime>u)
   \\
   & \qquad \qquad \qquad \qquad \qquad \cdot 
   \Big( (\arrivalTimes[]_n - u +1)^{1-\decCoef_{1,\xone,e}} - (\arrivalTimes[]_{n+1}\wedge t - u +1)^{1-\decCoef_{1,\xone,e}}\Big)
   \\
   & + 
   \frac{\impCoef_{1,\xone,e}}{\decCoef_{1,\xone,e} - 1}
   \sum_{0<  \arrivalTimes[]_n < t} \transProb_{e}(\stateVariable_{n-1}, x)  \transProb_{0}(\stateVariable_n, \xone)\int_{\arrivalTimes[]_n}^{\arrivalTimes[]_{n+1}\wedge t} du \, \Prob(\terminationTime>u)
   \\
   & \qquad \qquad \qquad \qquad \qquad \cdot 
   \Big( 1- (\arrivalTimes[]_{n+1} - u +1)^{1-\decCoef_{1,\xone,e}}\Big).
 \end{split}
\end{equation*}
\end{proof}


\subsection{Performance tests}\label{sec.performance_tests}

\subsubsection{Rejection sampling}
    \begin{tcolorbox}[breakable, size=fbox, boxrule=1pt, pad at break*=1mm,colback=cellbackground, colframe=cellborder]
\prompt{In}{incolor}{8}{\boxspacing}
\begin{Verbatim}[commandchars=\\\{\}]
\PY{n}{num\PYZus{}levels}\PY{p}{,} \PY{n}{l}\PY{p}{,} \PY{n}{u} \PY{o}{=} \PY{l+m+mi}{2}\PY{p}{,} \PY{o}{\PYZhy{}}\PY{l+m+mf}{0.85}\PY{p}{,} \PY{o}{\PYZhy{}}\PY{l+m+mf}{0.65}
\PY{n}{gamma}\PY{o}{=}\PY{n}{np}\PY{o}{.}\PY{n}{random}\PY{o}{.}\PY{n}{uniform}\PY{p}{(}\PY{n}{low}\PY{o}{=}\PY{l+m+mf}{0.0}\PY{p}{,} \PY{n}{high}\PY{o}{=}\PY{l+m+mf}{10.0}\PY{p}{,} \PY{n}{size}\PY{o}{=}\PY{p}{(}\PY{l+m+mi}{2}\PY{o}{*}\PY{n}{num\PYZus{}levels}\PY{p}{,}\PY{p}{)}\PY{p}{)}
\PY{n}{avgimb}\PY{o}{=}\PY{n}{np}\PY{o}{.}\PY{n}{sum}\PY{p}{(}\PY{n}{gamma}\PY{p}{[}\PY{l+m+mi}{1}\PY{p}{:}\PY{p}{:}\PY{l+m+mi}{2}\PY{p}{]}\PY{o}{\PYZhy{}}\PY{n}{gamma}\PY{p}{[}\PY{l+m+mi}{0}\PY{p}{:}\PY{p}{:}\PY{l+m+mi}{2}\PY{p}{]}\PY{p}{)}\PY{o}{/}\PY{n}{np}\PY{o}{.}\PY{n}{sum}\PY{p}{(}\PY{n}{gamma}\PY{p}{)}
\PY{n}{avgconstr}\PY{o}{=}\PY{p}{(}\PY{n}{l}\PY{o}{\PYZlt{}}\PY{o}{=}\PY{n}{avgimb}\PY{p}{)}\PY{o}{\PYZam{}}\PY{p}{(}\PY{n}{avgimb}\PY{o}{\PYZlt{}}\PY{n}{u}\PY{p}{)}
\PY{n}{res} \PY{o}{=} \PY{n}{find\PYZus{}rho}\PY{p}{(}\PY{n}{gamma}\PY{p}{,}\PY{n}{l}\PY{p}{,}\PY{n}{u}\PY{p}{)}
\PY{n}{rho} \PY{o}{=} \PY{n}{np}\PY{o}{.}\PY{n}{array}\PY{p}{(}\PY{n}{res}\PY{p}{[}\PY{l+s+s1}{\PYZsq{}}\PY{l+s+s1}{x}\PY{l+s+s1}{\PYZsq{}}\PY{p}{]}\PY{p}{,}\PY{n}{copy}\PY{o}{=}\PY{k+kc}{True}\PY{p}{)}
\PY{n}{alpha} \PY{o}{=} \PY{n}{gamma}\PY{o}{\PYZhy{}}\PY{n}{rho}
\PY{n}{M} \PY{o}{=} \PY{n}{produce\PYZus{}M}\PY{p}{(}\PY{n}{alpha}\PY{p}{,} \PY{n}{l}\PY{p}{,} \PY{n}{u}\PY{p}{)}
\end{Verbatim}
\end{tcolorbox}

\begin{tcolorbox}[breakable, size=fbox, boxrule=1pt, pad at break*=1mm,colback=cellbackground, colframe=cellborder]
\prompt{In}{incolor}{9}{\boxspacing}
\begin{Verbatim}[commandchars=\\\{\}]
\PY{n}{num\PYZus{}reject}\PY{o}{=}\PY{p}{[}\PY{p}{]}
\end{Verbatim}
\end{tcolorbox}

\begin{tcolorbox}[breakable, size=fbox, boxrule=1pt, pad at break*=1mm,colback=cellbackground, colframe=cellborder]
\prompt{In}{incolor}{10}{\boxspacing}
\begin{Verbatim}[commandchars=\\\{\}]
\PY{o}{\PYZpc{}\PYZpc{}time}it \PYZhy{}n 10 \PYZhy{}r 10
\PY{n}{num\PYZus{}reject}\PY{o}{.}\PY{n}{append}\PY{p}{(}
    \PY{n}{simulation}\PY{o}{.}\PY{n}{naive\PYZus{}sample}\PY{p}{(}\PY{n}{num\PYZus{}levels}\PY{p}{,}\PY{n}{gamma}\PY{p}{,}\PY{n}{l}\PY{p}{,}\PY{n}{u}\PY{p}{)}\PY{p}{)}
\end{Verbatim}
\end{tcolorbox}
\begin{tcolorbox}[breakable, size=fbox, boxrule=.5pt, pad at break*=1mm, opacityfill=0]
\prompt{Out}{outcolor}{10}{\boxspacing}
\begin{Verbatim}[commandchars=\\\{\}]
436 ms +/- 132 ms per loop (mean +/- std. dev. of 10 runs, 10 loops each)
\end{Verbatim}
\end{tcolorbox}

\begin{tcolorbox}[breakable, size=fbox, boxrule=1pt, pad at break*=1mm,colback=cellbackground, colframe=cellborder]
\prompt{In}{incolor}{11}{\boxspacing}
\begin{Verbatim}[commandchars=\\\{\}]
\PY{n+nb}{print}\PY{p}{(}\PY{l+s+s1}{\PYZsq{}}\PY{l+s+s1}{avg num of rejections: }\PY{l+s+si}{\PYZob{}\PYZcb{}}\PY{l+s+s1}{\PYZsq{}}
      \PY{o}{.}\PY{n}{format}\PY{p}{(}\PY{n}{np}\PY{o}{.}\PY{n}{mean}\PY{p}{(}\PY{n}{np}\PY{o}{.}\PY{n}{array}\PY{p}{(}\PY{n}{num\PYZus{}reject}\PY{p}{)}\PY{p}{)}\PY{p}{)}\PY{p}{)}
\PY{n+nb}{print}\PY{p}{(}\PY{l+s+s1}{\PYZsq{}}\PY{l+s+s1}{median num of rejections: }\PY{l+s+si}{\PYZob{}\PYZcb{}}\PY{l+s+s1}{\PYZsq{}}
      \PY{o}{.}\PY{n}{format}\PY{p}{(}\PY{n}{np}\PY{o}{.}\PY{n}{median}\PY{p}{(}\PY{n}{np}\PY{o}{.}\PY{n}{array}\PY{p}{(}\PY{n}{num\PYZus{}reject}\PY{p}{)}\PY{p}{)}\PY{p}{)}\PY{p}{)}
\end{Verbatim}
\end{tcolorbox}
\begin{tcolorbox}[breakable, size=fbox, boxrule=.5pt, pad at break*=1mm, opacityfill=0]
\prompt{Out}{outcolor}{11}{\boxspacing}
\begin{Verbatim}[commandchars=\\\{\}]
avg num of rejections: 18378.23
median num of rejections: 12710.0
\end{Verbatim}
\end{tcolorbox}

\begin{tcolorbox}[breakable, size=fbox, boxrule=1pt, pad at break*=1mm,colback=cellbackground, colframe=cellborder]
\prompt{In}{incolor}{12}{\boxspacing}
\begin{Verbatim}[commandchars=\\\{\}]
\PY{n}{num\PYZus{}reject}\PY{o}{=}\PY{p}{[}\PY{p}{]}
\end{Verbatim}
\end{tcolorbox}

\begin{tcolorbox}[breakable, size=fbox, boxrule=1pt, pad at break*=1mm,colback=cellbackground, colframe=cellborder]
\prompt{In}{incolor}{13}{\boxspacing}
\begin{Verbatim}[commandchars=\\\{\}]
\PY{o}{\PYZpc{}\PYZpc{}time}it \PYZhy{}n 10 \PYZhy{}r 10
\PY{n}{num\PYZus{}reject}\PY{o}{.}\PY{n}{append}\PY{p}{(}
    \PY{n}{simulation}\PY{o}{.}\PY{n}{rejection\PYZus{}sample}\PY{p}{(}\PY{n}{num\PYZus{}levels}\PY{p}{,}\PY{n}{rho}\PY{p}{,}\PY{n}{alpha}\PY{p}{,}\PY{n}{l}\PY{p}{,}\PY{n}{u}\PY{p}{,}\PY{n}{M}\PY{p}{)}\PY{p}{)}
\end{Verbatim}
\end{tcolorbox}
\begin{tcolorbox}[breakable, size=fbox, boxrule=.5pt, pad at break*=1mm, opacityfill=0]
\prompt{Out}{outcolor}{13}{\boxspacing}
\texttt{4.74 ms +/- 1.99 ms per loop (mean +/- std. dev. of 10 runs, 10 loops each)}
\end{tcolorbox}

\begin{tcolorbox}[breakable, size=fbox, boxrule=1pt, pad at break*=1mm,colback=cellbackground, colframe=cellborder]
\prompt{In}{incolor}{14}{\boxspacing}
\begin{Verbatim}[commandchars=\\\{\}]
\PY{n+nb}{print}\PY{p}{(}\PY{l+s+s1}{\PYZsq{}}\PY{l+s+s1}{avg num of rejections: }\PY{l+s+si}{\PYZob{}\PYZcb{}}\PY{l+s+s1}{\PYZsq{}}
      \PY{o}{.}\PY{n}{format}\PY{p}{(}\PY{n}{np}\PY{o}{.}\PY{n}{mean}\PY{p}{(}\PY{n}{np}\PY{o}{.}\PY{n}{array}\PY{p}{(}\PY{n}{num\PYZus{}reject}\PY{p}{)}\PY{p}{)}\PY{p}{)}\PY{p}{)}
\PY{n+nb}{print}\PY{p}{(}\PY{l+s+s1}{\PYZsq{}}\PY{l+s+s1}{median num of rejections: }\PY{l+s+si}{\PYZob{}\PYZcb{}}\PY{l+s+s1}{\PYZsq{}}
      \PY{o}{.}\PY{n}{format}\PY{p}{(}\PY{n}{np}\PY{o}{.}\PY{n}{median}\PY{p}{(}\PY{n}{np}\PY{o}{.}\PY{n}{array}\PY{p}{(}\PY{n}{num\PYZus{}reject}\PY{p}{)}\PY{p}{)}\PY{p}{)}\PY{p}{)}
\end{Verbatim}
\end{tcolorbox}
\begin{tcolorbox}[breakable, size=fbox, boxrule=.5pt, pad at break*=1mm, opacityfill=0]
\prompt{Out}{outcolor}{14}{\boxspacing}
    \begin{Verbatim}[commandchars=\\\{\}]
avg num of rejections: 178.57
median num of rejections: 121.0
    \end{Verbatim}
\end{tcolorbox}

\subsubsection{Evaluation of loglikelihood}

\subsubsection*{On a laptop}
   \begin{tcolorbox}[breakable, size=fbox, boxrule=1pt, pad at break*=1mm,colback=cellbackground, colframe=cellborder]
\prompt{In}{incolor}{2}{\boxspacing}
\begin{Verbatim}[commandchars=\\\{\}, fontsize=\small]
\PY{n}{date\PYZus{}time}\PY{o}{=}\PY{l+s+s2}{\PYZdq{}}\PY{l+s+s2}{2020\PYZhy{}04\PYZhy{}12\PYZus{}0738}\PY{l+s+s2}{\PYZdq{}}
\PY{k}{with} \PY{n+nb}{open}\PY{p}{(}\PY{n}{path\PYZus{}saved\PYZus{}tests}\PY{o}{+}\PY{l+s+s1}{\PYZsq{}}\PY{l+s+s1}{/perf\PYZhy{}test\PYZus{}}\PY{l+s+si}{\PYZob{}\PYZcb{}}\PY{l+s+s1}{/}\PY{l+s+s1}{\PYZsq{}}\PY{o}{.}\PY{n}{format}\PY{p}{(}\PY{n}{date\PYZus{}time}\PY{p}{)}\PY{o}{+}\PY{l+s+s1}{\PYZsq{}}\PY{l+s+s1}{test\PYZus{}model\PYZus{}}\PY{l+s+s1}{\PYZsq{}}\PY{o}{+}\PY{n}{date\PYZus{}time}\PY{p}{,} \PY{l+s+s1}{\PYZsq{}}\PY{l+s+s1}{rb}\PY{l+s+s1}{\PYZsq{}}\PY{p}{)} \PY{k}{as} \PY{n}{source}\PY{p}{:}
    \PY{n}{model}\PY{o}{=}\PY{n}{pickle}\PY{o}{.}\PY{n}{load}\PY{p}{(}\PY{n}{source}\PY{p}{)}
\PY{n}{meas} \PY{o}{=} \PY{n}{measure\PYZus{}exectime}\PY{o}{.}\PY{n}{PerformanceMeasure}\PY{p}{(}\PY{n}{model}\PY{p}{)}
\PY{n+nb}{print}\PY{p}{(}\PY{l+s+s2}{\PYZdq{}}\PY{l+s+s2}{Number of cpus: }\PY{l+s+si}{\PYZob{}\PYZcb{}}\PY{l+s+s2}{\PYZdq{}}\PY{o}{.}\PY{n}{format}\PY{p}{(}\PY{n}{os}\PY{o}{.}\PY{n}{cpu\PYZus{}count}\PY{p}{(}\PY{p}{)}\PY{p}{)}\PY{p}{)}
\PY{n+nb}{print}\PY{p}{(}\PY{l+s+s2}{\PYZdq{}}\PY{l+s+se}{\PYZbs{}n}\PY{l+s+s2}{Model}\PY{l+s+s2}{\PYZsq{}}\PY{l+s+s2}{s key features:}\PY{l+s+s2}{\PYZdq{}}\PY{p}{)}
\PY{n+nb}{print}\PY{p}{(}\PY{l+s+s2}{\PYZdq{}}\PY{l+s+s2}{d\PYZus{}E=}\PY{l+s+si}{\PYZob{}\PYZcb{}}\PY{l+s+s2}{; d\PYZus{}S=}\PY{l+s+si}{\PYZob{}\PYZcb{}}\PY{l+s+s2}{\PYZdq{}}\PY{o}{.}\PY{n}{format}\PY{p}{(}\PY{n}{meas}\PY{o}{.}\PY{n}{model}\PY{o}{.}\PY{n}{number\PYZus{}of\PYZus{}event\PYZus{}types}\PY{p}{,} \PY{n}{meas}\PY{o}{.}\PY{n}{model}\PY{o}{.}\PY{n}{number\PYZus{}of\PYZus{}states}\PY{p}{)}\PY{p}{)}
\PY{n+nb}{print}\PY{p}{(}\PY{l+s+s2}{\PYZdq{}}\PY{l+s+s2}{Number of simulated LOB events: }\PY{l+s+si}{\PYZob{}\PYZcb{}}\PY{l+s+s2}{\PYZdq{}}\PY{o}{.}\PY{n}{format}\PY{p}{(}\PY{n+nb}{len}\PY{p}{(}\PY{n}{meas}\PY{o}{.}\PY{n}{model}\PY{o}{.}\PY{n}{simulated\PYZus{}events}\PY{p}{)}\PY{p}{)}\PY{p}{)}
\end{Verbatim}
\end{tcolorbox}
\begin{tcolorbox}[breakable, size=fbox, boxrule=.5pt, pad at break*=1mm, opacityfill=0]
\prompt{Out}{outcolor}{2}{\boxspacing}\begin{Verbatim}[commandchars=\\\{\}, fontsize=\small]
PerformanceMeasure is being initialised on
model.name\_of\_model=test\_model\_2020-04-12\_0738
posix.uname\_result(sysname='Linux', nodename='claudio-HP-EliteBook-850-G3',
release='5.3.0-46-generic', version='\#38\textasciitilde{}18.04.1-Ubuntu SMP Tue Mar 31 04:17:56
UTC 2020', machine='x86\_64')
Number of cpus: 4

Model's key features:
d\_E=4; d\_S=15
Number of simulated LOB events: 19958
    \end{Verbatim}
\end{tcolorbox}
\begin{tcolorbox}[breakable, size=fbox, boxrule=1pt, pad at break*=1mm,colback=cellbackground, colframe=cellborder]
\prompt{In}{incolor}{3}{\boxspacing}
\begin{Verbatim}[commandchars=\\\{\}, fontsize=\small]
\PY{c+c1}{\PYZsh{}select component of multivariate Hawkes process e=0,...,d\PYZus{}E\PYZhy{}1}
\PY{n}{e} \PY{o}{=} \PY{l+m+mi}{0}
\end{Verbatim}
\end{tcolorbox}
\begin{tcolorbox}[breakable, size=fbox, boxrule=1pt, pad at break*=1mm,colback=cellbackground, colframe=cellborder]
\prompt{In}{incolor}{4}{\boxspacing}
\begin{Verbatim}[commandchars=\\\{\}, fontsize=\small]
\PY{o}{\PYZpc{}\PYZpc{}timeit} \PYZhy{}n 5 \PYZhy{}r 3
\PY{n}{meas}\PY{o}{.}\PY{n}{target\PYZus{}loglikelihood}\PY{p}{(}\PY{n}{e}\PY{p}{,} \PY{n}{use\PYZus{}prange}\PY{o}{=}\PY{k+kc}{False}\PY{p}{)}
\end{Verbatim}
\end{tcolorbox}
\begin{tcolorbox}[breakable, size=fbox, boxrule=.5pt, pad at break*=1mm, opacityfill=0]
\prompt{Out}{outcolor}{4}{\boxspacing}
\begin{Verbatim}[commandchars=\\\{\}, fontsize=\small]
3.37 s +/- 50.8 ms per loop (mean +/- std. dev. of 3 runs, 5 loops each)
    \end{Verbatim}
\end{tcolorbox}
\begin{tcolorbox}[breakable, size=fbox, boxrule=1pt, pad at break*=1mm,colback=cellbackground, colframe=cellborder]
\prompt{In}{incolor}{5}{\boxspacing}
\begin{Verbatim}[commandchars=\\\{\}, fontsize=\small]
\PY{o}{\PYZpc{}\PYZpc{}timeit} \PYZhy{}n 5 \PYZhy{}r 3
\PY{n}{meas}\PY{o}{.}\PY{n}{target\PYZus{}loglikelihood}\PY{p}{(}\PY{n}{e}\PY{p}{,} \PY{n}{use\PYZus{}prange}\PY{o}{=}\PY{k+kc}{True}\PY{p}{)}
\end{Verbatim}
\end{tcolorbox}
\begin{tcolorbox}[breakable, size=fbox, boxrule=.5pt, pad at break*=1mm, opacityfill=0]
\prompt{Out}{outcolor}{5}{\boxspacing}
\begin{Verbatim}[commandchars=\\\{\}, fontsize=\small]
1.26 s +/- 5.1 ms per loop (mean +/- std. dev. of 3 runs, 5 loops each)
    \end{Verbatim}
\end{tcolorbox}
\subsubsection*{On a cluster hpc}    
\begin{tcolorbox}[breakable, size=fbox, boxrule=1pt, pad at break*=1mm,colframe=cellborder]    
\begin{Verbatim}[commandchars=\\\{\}, fontsize=\small]
 $python test_perf.py --loglikelihood
this_test_model_name: test_model_2020-04-12_0738
date_time of model: 2020-04-12_0738
Date and time of test: 2020-04-12 at 11:50

PerformanceMeasure is being initialised on model.name_of_model=test_model_2020-04-12_0738
posix.uname_result(sysname='Linux', nodename='r5i2n0', release='3.10.0-514.26.2.el7.x86_64', version='#1 SMP Tue Jul 4 15:04:05 UTC 2017', machine='x86_64')
Number of cpus: 8

Model's key features:
d_E=4; d_S=15
Number of simulated LOB events: 19958

Execution times for the function 'computation.compute_event_loglikelihood_partial_and_gradient_partial' with 'plain' for-loops (no prange):
[3.7853622250258923, 4.9914959750603884, 4.556901058880612, 4.488794992910698]
Execution times for the function 'computation.compute_event_loglikelihood_partial_and_gradient_partial' with 'prange' in outermost for-loop:
[0.25286552612669766, 0.33203633199445903, 0.30347337294369936, 0.2986017200164497]

measure_loglikelihood() terminates on 2020-04-12 at 11:53
============================================
        Job resource usage summary 
                 Memory (GB)    NCPUs
 Requested  :        10             8
 Used       :         2 (peak)   1.16 (ave)
============================================
\end{Verbatim}
\end{tcolorbox}


\subsubsection{Goodness-of-fit}

\begin{tcolorbox}[breakable, size=fbox, boxrule=1pt, pad at break*=1mm,colback=cellbackground, colframe=cellborder]
\prompt{In}{incolor}{5}{\boxspacing}
\begin{Verbatim}[commandchars=\\\{\}, , fontsize=\small]
\PY{n}{date\PYZus{}time} \PY{o}{=} \PY{l+s+s2}{\PYZdq{}}\PY{l+s+s2}{2020\PYZhy{}04\PYZhy{}12\PYZus{}0738}\PY{l+s+s2}{\PYZdq{}}
\PY{n}{name\PYZus{}of\PYZus{}test} \PY{o}{=} \PY{l+s+s2}{\PYZdq{}}\PY{l+s+s2}{prange\PYZus{}test\PYZus{}model\PYZus{}}\PY{l+s+s2}{\PYZdq{}}\PY{o}{+}\PY{n}{date\PYZus{}time}
\end{Verbatim}
\end{tcolorbox}
\begin{tcolorbox}[breakable, size=fbox, boxrule=1pt, pad at break*=1mm,colback=cellbackground, colframe=cellborder]
\prompt{In}{incolor}{6}{\boxspacing}
\begin{Verbatim}[commandchars=\\\{\}, fontsize=\small]
\PY{k}{with} \PY{n+nb}{open}\PY{p}{(}\PY{n}{path\PYZus{}tests}\PY{o}{+}\PY{l+s+s1}{\PYZsq{}}\PY{l+s+s1}{/saved\PYZus{}tests/perf\PYZhy{}test\PYZus{}}\PY{l+s+s1}{\PYZsq{}}\PY{o}{+}\PY{n}{date\PYZus{}time}\PY{o}{+}\PY{l+s+s1}{\PYZsq{}}\PY{l+s+s1}{/}\PY{l+s+s1}{\PYZsq{}}\PY{o}{+}\PY{n}{name\PYZus{}of\PYZus{}test}\PY{p}{,}\PY{l+s+s1}{\PYZsq{}}\PY{l+s+s1}{rb}\PY{l+s+s1}{\PYZsq{}}\PY{p}{)} \PY{k}{as} \PY{n}{source}\PY{p}{:}
    \PY{n}{model}\PY{o}{=}\PY{n}{pickle}\PY{o}{.}\PY{n}{load}\PY{p}{(}\PY{n}{source}\PY{p}{)}
\PY{n+nb}{print}\PY{p}{(}\PY{n}{model}\PY{o}{.}\PY{n}{name\PYZus{}of\PYZus{}model}\PY{p}{)}
\PY{n+nb}{print}\PY{p}{(}\PY{l+s+s1}{\PYZsq{}}\PY{l+s+s1}{Date and time of initilisation: }\PY{l+s+si}{\PYZob{}\PYZcb{}}\PY{l+s+s1}{\PYZsq{}}\PY{o}{.}\PY{n}{format}\PY{p}{(}\PY{n}{model}\PY{o}{.}\PY{n}{datetime\PYZus{}of\PYZus{}initialisation}\PY{p}{)}\PY{p}{)}
\PY{n+nb}{print}\PY{p}{(}\PY{l+s+s1}{\PYZsq{}}\PY{l+s+s1}{number\PYZus{}of\PYZus{}event\PYZus{}types: }\PY{l+s+si}{\PYZob{}\PYZcb{}}\PY{l+s+s1}{; number\PYZus{}of\PYZus{}states: }\PY{l+s+si}{\PYZob{}\PYZcb{}}\PY{l+s+s1}{\PYZsq{}}\PY{o}{.}\PY{n}{format}\PY{p}{(}
    \PY{n}{model}\PY{o}{.}\PY{n}{number\PYZus{}of\PYZus{}event\PYZus{}types}\PY{p}{,} \PY{n}{model}\PY{o}{.}\PY{n}{number\PYZus{}of\PYZus{}states}\PY{p}{)}\PY{p}{)}
\end{Verbatim}
\end{tcolorbox}
\begin{tcolorbox}[breakable, size=fbox, boxrule=.5pt, pad at break*=1mm, opacityfill=0]
\prompt{Out}{outcolor}{6}{\boxspacing}
\begin{Verbatim}[commandchars=\\\{\}, fontsize=\small]
prange\_test\_model\_2020-04-12\_0738
Date and time of initilisation: 2020-04-12 07:38:26.601783
number\_of\_event\_types: 4; number\_of\_states: 15
    \end{Verbatim}
\end{tcolorbox}

The following three figures will show QQ plots that assess goodness of
fit. The first figure refers to the coefficients inserted by the user;
it asseses the reliability of the simulation and it is expected to show
good results. The second picture refers to the coefficients obtained via
non-parametric estimation. The results can be mixed since the procedure
assumes stationarity of the point process and i.i.d. sequences of marks
(which may not hold). The third figure refers to the parameters obtained
via maximum likelihood estimation and it is expected to be an
improvement over the non-parametric estimation; this is because on the
one hand the mle estimation does not assume stationarity, and on the
other hand the results of the non-parametric estimations are used as one
of the initilisations for the gradient descent of the mle estimation.

\begin{tcolorbox}[breakable, size=fbox, boxrule=1pt, pad at break*=1mm,colback=cellbackground, colframe=cellborder]
\prompt{In}{incolor}{7}{\boxspacing}
\begin{Verbatim}[commandchars=\\\{\}, fontsize=\small]
\PY{n}{model}\PY{o}{.}\PY{n}{goodness\PYZus{}of\PYZus{}fit}\PY{o}{.}\PY{n}{qq\PYZus{}plot\PYZus{}residuals}\PY{p}{(}\PY{n}{index\PYZus{}of\PYZus{}first\PYZus{}event\PYZus{}type}\PY{o}{=}\PY{l+m+mi}{0}\PY{p}{)}
\PY{n+nb}{print}\PY{p}{(}\PY{n}{model}\PY{o}{.}\PY{n}{goodness\PYZus{}of\PYZus{}fit}\PY{o}{.}\PY{n}{ad\PYZus{}test\PYZus{}on\PYZus{}residuals}\PY{p}{(}\PY{p}{)}\PY{p}{)}
\end{Verbatim}
\end{tcolorbox}
\begin{tcolorbox}[breakable, size=fbox, boxrule=.5pt, pad at break*=1mm, opacityfill=0]
\prompt{Out}{outcolor}{7}{\boxspacing}
\begin{center}
    \adjustimage{max size={0.9\linewidth}{0.9\paperheight}}{gfit_2020-04-12_0738.png}
    \end{center}
    { \hspace*{\fill} \\}
    
    \begin{Verbatim}[commandchars=\\\{\}, fontsize=\small]
Anderson-Darling test to check distribution of residuals
Null hypothesis is "expon"
Significance levels: [15.  10.   5.   2.5  1. ]
Critical values: [0.922 1.078 1.341 1.606 1.957]
event type=0, ad\_stat: 0.2141966471563137
event type=1, ad\_stat: 0.49801130759806256
event type=2, ad\_stat: 0.45093652879040746
event type=3, ad\_stat: 0.8266097663436085
    \end{Verbatim}
\end{tcolorbox}
\begin{tcolorbox}[breakable, size=fbox, boxrule=1pt, pad at break*=1mm,colback=cellbackground, colframe=cellborder]
\prompt{In}{incolor}{8}{\boxspacing}
\begin{Verbatim}[commandchars=\\\{\}, fontsize=\small]
\PY{n}{model}\PY{o}{.}\PY{n}{nonparam\PYZus{}estim}\PY{o}{.}\PY{n}{goodness\PYZus{}of\PYZus{}fit}\PY{o}{.}\PY{n}{qq\PYZus{}plot\PYZus{}residuals}\PY{p}{(}\PY{n}{index\PYZus{}of\PYZus{}first\PYZus{}event\PYZus{}type}\PY{o}{=}\PY{l+m+mi}{0}\PY{p}{)}
\PY{n+nb}{print}\PY{p}{(}\PY{n}{model}\PY{o}{.}\PY{n}{nonparam\PYZus{}estim}\PY{o}{.}\PY{n}{goodness\PYZus{}of\PYZus{}fit}\PY{o}{.}\PY{n}{ad\PYZus{}test\PYZus{}on\PYZus{}residuals}\PY{p}{(}\PY{p}{)}\PY{p}{)}
\end{Verbatim}
\end{tcolorbox}
\begin{tcolorbox}[breakable, size=fbox, boxrule=.5pt, pad at break*=1mm, opacityfill=0]
\prompt{Out}{outcolor}{8}{\boxspacing}
\begin{center}
    \adjustimage{max size={0.9\linewidth}{0.9\paperheight}}{gfit_2020-04-12_0738_nonparam.png}
    \end{center}
    { \hspace*{\fill} \\}
    
    \begin{Verbatim}[commandchars=\\\{\}, fontsize=\small]
Anderson-Darling test to check distribution of residuals
Null hypothesis is "expon"
Significance levels: [15.  10.   5.   2.5  1. ]
Critical values: [0.922 1.078 1.341 1.606 1.957]
event type=0, ad\_stat: 120.51717269351502
event type=1, ad\_stat: 54.94009196285606
event type=2, ad\_stat: 0.8662745583551441
event type=3, ad\_stat: 222.09860859998298
    \end{Verbatim}
\end{tcolorbox}
\begin{tcolorbox}[breakable, size=fbox, boxrule=1pt, pad at break*=1mm,colback=cellbackground, colframe=cellborder]
\prompt{In}{incolor}{9}{\boxspacing}
\begin{Verbatim}[commandchars=\\\{\}, fontsize=\small]
\PY{n}{model}\PY{o}{.}\PY{n}{mle\PYZus{}estim}\PY{o}{.}\PY{n}{goodness\PYZus{}of\PYZus{}fit}\PY{o}{.}\PY{n}{qq\PYZus{}plot\PYZus{}residuals}\PY{p}{(}\PY{n}{index\PYZus{}of\PYZus{}first\PYZus{}event\PYZus{}type}\PY{o}{=}\PY{l+m+mi}{0}\PY{p}{)}
\PY{n+nb}{print}\PY{p}{(}\PY{n}{model}\PY{o}{.}\PY{n}{mle\PYZus{}estim}\PY{o}{.}\PY{n}{goodness\PYZus{}of\PYZus{}fit}\PY{o}{.}\PY{n}{ad\PYZus{}test\PYZus{}on\PYZus{}residuals}\PY{p}{(}\PY{p}{)}\PY{p}{)}
\end{Verbatim}
\end{tcolorbox}
\begin{tcolorbox}[breakable, size=fbox, boxrule=.5pt, pad at break*=1mm, opacityfill=0]
\prompt{Out}{outcolor}{9}{\boxspacing}
\begin{center}
    \adjustimage{max size={0.9\linewidth}{0.9\paperheight}}{gfit_2020-04-12_0738_mle.png}
    \end{center}
    { \hspace*{\fill} \\}
    
    \begin{Verbatim}[commandchars=\\\{\}, fontsize=\small]
Anderson-Darling test to check distribution of residuals
Null hypothesis is "expon"
Significance levels: [15.  10.   5.   2.5  1. ]
Critical values: [0.922 1.078 1.341 1.606 1.957]
event type=0, ad\_stat: 0.2137307976381635
event type=1, ad\_stat: 0.8740122294875619
event type=2, ad\_stat: 1.5654262188509165
event type=3, ad\_stat: 1.3660949155273556
    \end{Verbatim}
\end{tcolorbox}

\section{Empirical analysis of LOBSTER data}
Describe how the lobster dataset is filtered. Report results of calibrationand key findings. 


\newpage

\addcontentsline{toc}{section}{References}
\bibliographystyle{apalike}
\bibliography{pipest_bib}    

\newpage


\begin{appendices}
\section{Readouts of \texttt{test 2020-04-12\_0738}}
\subsection{Execution times and resource usage summaries}
\subsubsection{Estimation of component e=0}
 \begin{tcolorbox}[breakable, size=fbox, boxrule=1pt, pad at break*=1mm,colframe=cellborder]    
  \begin{Verbatim}[commandchars=\\\{\}, fontsize=\small]
EVENT TYPE 0
this_test_model_name: test_model_2020-04-12_0738
type_of_paral: plain_
I am executing test_estim_multijob.py --mle_estimation --plain_
Date of run: 2020-04-12 at 07:43
stdout is being redirected to pipest/tests/saved_tests/plain_test_model_2020-04-12_0738_mle_readout_partial0.txt
estimate_hawkes_power_partial, event_type=0, run_time = 1009.2016847133636
Estimation of component_e 0 terminates on 2020-04-12 at 8:00
============================================
        Job resource usage summary 
                 Memory (GB)    NCPUs
 Requested  :        30             2
 Used       :         2 (peak)   1.50 (ave)
============================================
this_test_model_name: test_model_2020-04-12_0738
type_of_paral: prange_
I am executing test_estim_multijob.py --mle_estimation --prange_
Date of run: 2020-04-12 at 08:09
stdout is being redirected to pipest/tests/saved_tests/prange_test_model_2020-04-12_0738_mle_readout_partial0.txt
estimate_hawkes_power_partial, event_type=0, run_time = 94.86032009124756
Estimation of component_e 0 terminates on 2020-04-12 at 8:11
============================================
        Job resource usage summary 
                 Memory (GB)    NCPUs
 Requested  :        30             8
 Used       :         2 (peak)   4.63 (ave)
============================================
this_test_model_name: test_model_2020-04-12_0738
type_of_paral: pool_
I am executing test_estim_multijob.py --mle_estimation --pool_
Date of run: 2020-04-12 at 08:23
stdout is being redirected to pipest/tests/saved_tests/pool_test_model_2020-04-12_0738_mle_readout_partial0.txt
estimate_hawkes_power_partial, event_type=0, run_time = 208.4092574119568
Estimation of component_e 0 terminates on 2020-04-12 at 8:27
============================================
        Job resource usage summary 
                 Memory (GB)    NCPUs
 Requested  :        30             8
 Used       :        22 (peak)   3.14 (ave)
============================================
 \end{Verbatim}

\end{tcolorbox}
\subsubsection{Estimation of component e=1}
\begin{tcolorbox}[breakable, size=fbox, boxrule=1pt, pad at break*=1mm,colframe=cellborder]    
  \begin{Verbatim}[commandchars=\\\{\}, fontsize=\small]
EVENT TYPE 1
this_test_model_name: test_model_2020-04-12_0738
type_of_paral: plain_
I am executing test_estim_multijob.py --mle_estimation --plain_
Date of run: 2020-04-12 at 07:43
stdout is being redirected to pipest/tests/saved_tests/plain_test_model_2020-04-12_0738_mle_readout_partial1.txt
estimate_hawkes_power_partial, event_type=1, run_time = 1953.6043756008148
Estimation of component_e 1 terminates on 2020-04-12 at 8:16
============================================
        Job resource usage summary 
                 Memory (GB)    NCPUs
 Requested  :        30             2
 Used       :         2 (peak)   1.31 (ave)
============================================
this_test_model_name: test_model_2020-04-12_0738
type_of_paral: prange_
I am executing test_estim_multijob.py --mle_estimation --prange_
Date of run: 2020-04-12 at 08:09
stdout is being redirected to pipest/tests/saved_tests/prange_test_model_2020-04-12_0738_mle_readout_partial1.txt
estimate_hawkes_power_partial, event_type=1, run_time = 117.41784977912903
Estimation of component_e 1 terminates on 2020-04-12 at 8:11
============================================
        Job resource usage summary 
                 Memory (GB)    NCPUs
 Requested  :        30             8
 Used       :         2 (peak)   5.08 (ave)
============================================
this_test_model_name: test_model_2020-04-12_0738
type_of_paral: pool_
I am executing test_estim_multijob.py --mle_estimation --pool_
Date of run: 2020-04-12 at 08:24
stdout is being redirected to pipest/tests/saved_tests/pool_test_model_2020-04-12_0738_mle_readout_partial1.txt
estimate_hawkes_power_partial, event_type=1, run_time = 233.002938747406
Estimation of component_e 1 terminates on 2020-04-12 at 8:28
============================================
        Job resource usage summary 
                 Memory (GB)    NCPUs
 Requested  :        30             8
 Used       :        22 (peak)   3.60 (ave)
============================================
 \end{Verbatim}

\end{tcolorbox}
\subsubsection{Estimation of component e=2}
 \begin{tcolorbox}[breakable, size=fbox, boxrule=1pt, pad at break*=1mm,colframe=cellborder]    
  \begin{Verbatim}[commandchars=\\\{\}, fontsize=\small]
EVENT TYPE 2
this_test_model_name: test_model_2020-04-12_0738
type_of_paral: plain_
I am executing test_estim_multijob.py --mle_estimation --plain_
Date of run: 2020-04-12 at 07:43
stdout is being redirected to pipest/tests/saved_tests/plain_test_model_2020-04-12_0738_mle_readout_partial2.txt
estimate_hawkes_power_partial, event_type=2, run_time = 1592.5766327381134
Estimation of component_e 2 terminates on 2020-04-12 at 8:10
============================================
        Job resource usage summary 
                 Memory (GB)    NCPUs
 Requested  :        30             2
 Used       :         2 (peak)   1.33 (ave)
============================================
this_test_model_name: test_model_2020-04-12_0738
type_of_paral: prange_
I am executing test_estim_multijob.py --mle_estimation --prange_
Date of run: 2020-04-12 at 08:06
stdout is being redirected to pipest/tests/saved_tests/prange_test_model_2020-04-12_0738_mle_readout_partial2.txt
estimate_hawkes_power_partial, event_type=2, run_time = 91.04270052909851
Estimation of component_e 2 terminates on 2020-04-12 at 8:09
============================================
        Job resource usage summary 
                 Memory (GB)    NCPUs
 Requested  :        30             8
 Used       :         2 (peak)   4.39 (ave)
============================================
this_test_model_name: test_model_2020-04-12_0738
type_of_paral: pool_
I am executing test_estim_multijob.py --mle_estimation --pool_
Date of run: 2020-04-12 at 08:24
stdout is being redirected to pipest/tests/saved_tests/pool_test_model_2020-04-12_0738_mle_readout_partial2.txt
estimate_hawkes_power_partial, event_type=2, run_time = 188.75495052337646
Estimation of component_e 2 terminates on 2020-04-12 at 8:27
============================================
        Job resource usage summary 
                 Memory (GB)    NCPUs
 Requested  :        30             8
 Used       :        22 (peak)   2.98 (ave)
============================================
 \end{Verbatim}

\end{tcolorbox}
\subsubsection{Estimation of component e=3}
 \begin{tcolorbox}[breakable, size=fbox, boxrule=1pt, pad at break*=1mm,colframe=cellborder]    
  \begin{Verbatim}[commandchars=\\\{\}, fontsize=\small]
EVENT TYPE 3
this_test_model_name: test_model_2020-04-12_0738
type_of_paral: plain_
I am executing test_estim_multijob.py --mle_estimation --plain_
Date of run: 2020-04-12 at 07:42
stdout is being redirected to pipest/tests/saved_tests/plain_test_model_2020-04-12_0738_mle_readout_partial3.txt
estimate_hawkes_power_partial, event_type=3, run_time = 1069.9958610534668
Estimation of component_e 3 terminates on 2020-04-12 at 8:00
============================================
        Job resource usage summary 
                 Memory (GB)    NCPUs
 Requested  :        30             2
 Used       :         2 (peak)   1.99 (ave)
============================================
this_test_model_name: test_model_2020-04-12_0738
type_of_paral: prange_
I am executing test_estim_multijob.py --mle_estimation --prange_
Date of run: 2020-04-12 at 08:06
stdout is being redirected to pipest/tests/saved_tests/prange_test_model_2020-04-12_0738_mle_readout_partial3.txt
estimate_hawkes_power_partial, event_type=3, run_time = 98.83592748641968
Estimation of component_e 3 terminates on 2020-04-12 at 8:09
============================================
        Job resource usage summary 
                 Memory (GB)    NCPUs
 Requested  :        30             8
 Used       :         2 (peak)   4.78 (ave)
============================================
this_test_model_name: test_model_2020-04-12_0738
type_of_paral: pool_
I am executing test_estim_multijob.py --mle_estimation --pool_
Date of run: 2020-04-12 at 08:23
stdout is being redirected to pipest/tests/saved_tests/pool_test_model_2020-04-12_0738_mle_readout_partial3.txt
estimate_hawkes_power_partial, event_type=3, run_time = 135.273184299469
Estimation of component_e 3 terminates on 2020-04-12 at 8:25
============================================
        Job resource usage summary 
                 Memory (GB)    NCPUs
 Requested  :        30             8
 Used       :        21 (peak)   4.39 (ave)
============================================
 \end{Verbatim}

\end{tcolorbox}
\end{appendices}

\end{document}
